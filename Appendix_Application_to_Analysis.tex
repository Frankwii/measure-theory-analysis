%!TeX root=Final.tex

\chapter{A Brief Application to Analysis}\label{chapter:A Brief Application to Analysis}
\setcounter{section}{1}
\begin{prop}\label{proposition:integral over a simple region}
    Let \(C\subseteq\mathbb{R}^{n-1}\) be a compact set (with the usual topology). Let \(l,u\colon C\to\mathbb{R}\) be continuous functions such that \(l(x)\leq u(x)~\forall x\in C\). Consider the set
    \[
        S=\{(x,t)\in\mathbb{R}^{n-1}\times\mathbb{R}\left|x\in C, l(x)\leq t\leq u(x)\right.\}
    .\]
    Let \(f\colon S\to \mathbb{R}\) be a bounded, continuous function. Then, \(f\) is integrable on \(S\) and 
    \[
        \int_S f = \int_{x\in C}\int_{t=l(x)}^{t=u(x)} f(x,t)
    ,\]
    
    where the integration is understood to be performed in the Riemann sense.  
\end{prop}
\begin{proof}
    We know, from \Cref{theorem:Riemann and Lebesgue}, that Riemann integration coincides with Lebesgue integration (with respect to the Lebesgue measure) on ``classical'' rectangles of \(\mathbb{R}^n\) (that is, cartesian products of intervals). Riemann integration is usually defined on classical rectangles and then extended to bounded sets via indicator functions (see \cite{apuntes_Manolo}). Thus, both kinds of integration will coincide on bounded sets.

    It follows from Heine-Borel and Weierstrass Theorems that \(S\) is a bounded subset of \(\mathbb{R}^n\) (Heine-Borel guarantees the boundedness of \(C\), and Weierstrass guarantees the boundedness of limit functions \(a(x),b(x)\)). Thus, we can apply all techniques regarding Lebesgue integration.

    Additionally, \(S\) is closed because of the following reasoning:
if \(g\colon X\to \mathbb{R}\) is a continuous function, then the epigraph \(E(f)=\{(x,y)\in X\times\mathbb{R}\left|g(x)\leq y\right.\}\) is a closed set of the product topological space \(X\times \mathbb{R}\). Similarly, so is the hypograph \(H(f)=\{(x,y)\in X\times\mathbb{R}\left|g(x)\geq y\right.\}\). Then, \(S=E(l)\cap H(u)\) is closed. By the Heine-Borel theorem, \(S\) is compact.

    Extend \(f\) to \(\mathbb{R}^n\) setting its value to \(0\) outside of \(C\). Now \(I_Sf\), regarded as a function from \(\mathbb{R}^n\) to \(\mathbb{R}\), is Borel measurable (since \(S\) is closed, it is Borel measurable) and bounded because \(f\) is (by Weierstrass Theorem). Since \(S\) is also bounded, \(I_Sf\) is integrable on \(\mathbb{R}^n\).

    Now split \(f=f^{+}-f^{-}\). Use \Cref{remark:two-dim Fubini's on subsets} to see that
    \[
        \int_{S}f^{+}=\int_{\mathbb{R}^{n-1}}\left(\int_{S^{x}}f^{+}(x,t)~dt\right)dx
    .\]
    Now note that \(S^x=[l(x),u(x)]\) if \(x\in C\) and \(S^x=\emptyset\) otherwise. Therefore,
    \[
        \int_{S}f^{+}=\int_{C}\int_{l(x)}^{u(x)}f^{+}(x,t)~dt~dx
    .\]
    A similar reasoning applied to \(f^{-}\) yields the desired result expanding and regrouping the expression \(\int_{S}f=\int_{S}f^{+}-\int_{S}f^{-}\).
\end{proof}

\documentclass[english,GMAT]{TFGEPSUIB}

\usepackage[printonlyused]{acronym}
\usepackage{siunitx}

\usepackage{graphicx} % Required for inserting images
\usepackage{tikz-cd}
\usepackage{babel}
\usepackage{multicol}
\usepackage{bm}
\usepackage{mathrsfs}
\usepackage{amsmath}
\usepackage{thmtools}

\usepackage{enumitem}
\usepackage[amsmath,amsthm,thmmarks,hyperref]{ntheorem}
\usepackage[backref,hidelinks, colorlinks=true, all colors=black]{hyperref}
\usepackage[capitalize]{cleveref}
\usepackage{etoolbox}

\title{Measure theory and applications}

\author{Frank William Hammond Espinosa}

% La comanda \tutor mostra el nom del director
% a la portada interior. Si hi ha més d'un tutor
% caldrà fer un petit canvi. Demanau ajuda. 
\tutor{Bartomeu Coll Vicens i Maria Jolis Gimenez}

\date{Academic year 2023-24}

\paraulesclau{Measure Theory, Integration, Probability Theory}

%---------------------------------------------------------------------------------------------------------------------------------------------------------------

\newtheorem{thrm}{Theorem}[section]
\newtheorem{lemm}[thrm]{Lemma}
\newtheorem{prop}[thrm]{Proposition}
\newtheorem{defn}[thrm]{Definition}
\newtheorem{corl}[thrm]{Corollary}
\newtheorem{remk}[thrm]{Remark}

\Crefname{thrm}{Theorem}{Theorems}
\Crefname{thrmenumi}{Theorem}{Theorems}
\AtBeginEnvironment{thrm}{%
    \crefalias{enumi}{thrmenumi}%
    \setlist[enumerate,1]{
        label={\textit{(\roman*)}},
        ref={\thethrm.(\roman*)}
    }%
}

\Crefname{lemm}{Lemma}{Lemmas}
\Crefname{lemmenumi}{Lemma}{Lemmas}
\AtBeginEnvironment{lemm}{%
    \crefalias{enumi}{lemmenumi}%
    \setlist[enumerate,1]{
        label={\textit{(\roman*)}},
        ref={\thelemm.(\roman*)}
    }%
}

\Crefname{prop}{Proposition}{Propositions}
\Crefname{propenumi}{Proposition}{Propositions}
\AtBeginEnvironment{prop}{%
    \crefalias{enumi}{propenumi}%
    \setlist[enumerate,1]{
        label={\textit{(\roman*)}},
        ref={\theprop.(\roman*)}
    }%
}

\Crefname{defn}{Definition}{Definitions}
\Crefname{defnenumi}{Definition}{Definitions}
\AtBeginEnvironment{defn}{%
    \crefalias{enumi}{defnenumi}%
    \setlist[enumerate,1]{
        label={\textit{(\roman*)}},
        ref={\thedefn.(\roman*)}
    }%
}

\Crefname{corl}{Corollary}{Corollaries}
\Crefname{corlenumi}{Corollary}{Corollaries}
\AtBeginEnvironment{corl}{%
    \crefalias{enumi}{corlenumi}%
    \setlist[enumerate,1]{
        label={\textit{(\roman*)}},
        ref={\thecorl.(\roman*)}
    }%
}

\Crefname{remk}{Remark}{Remarks}
\Crefname{remkenumi}{Remark}{Remarks}
\AtBeginEnvironment{remk}{%
    \crefalias{enumi}{remkenumi}%
    \setlist[enumerate,1]{
        label={\textit{(\roman*)}},
        ref={\theremk.(\roman*)}
    }%
}


\setlength{\parindent}{0px}
\setlength{\parskip}{.2em}
\begin{document}

\portada
\portadainterior
\frontmatter

\cleartorecto \thispagestyle{empty}
\begin{agraiments}
Este escrito se lo dedico a mis amigos dentro y fuera del país. En especial, a mi familia y a Marga.
\end{agraiments}

\cleartorecto \tableofcontents

% Si voleu que apareguin una llista de figures
% i taules, activau les línies corresponents
%\cleartorecto \listoffigures
%\cleartorecto \listoftables 

% Si apareixen molts acrònims a la documentació
% convindrà fer-ne una llista. Podeu veure com
% crear-la consultant el fitxer 'Acronims.tex',
% que és el que s'inclou aquí.
% \include{Acronims}
% Si no usau acrònims, comentau la línia anterior

% En l'arxiu Resum.tex es posarà el resum
% del treball.
% \include{Resum}

% No toqueu la línia següent 
\mainmatter\pagestyle{ruled}

%%%%%% COS DEL TREBALL %%%%%%%%%%%%

%!TeX root=Final.tex

\chapter{INTRODUCTION}

Metric concepts such as longitudes, areas and volumes come as very natural to us. It seems almost evident that most objects appearing in our daily life can
be assigned a number, its \emph{volume}, and the same is true when imagining two-dimensional objects in a plane or one-dimensional objects in a line.
They are so intuitive that in the seventeenth century, when calculus appeared, the integral was regarded simply as ``the area under a curve'', and little 
attention was given to what the word ``area'' meant or how it related to different curves (can every curve be assigned an ``area under it''?).

The first attempt at formalising the notion of integral that is studied today was that of Bernhard Riemann. The notion of volume can then be obtained simply as the integral of the constant function \(1\). His approach is now at the core of most university courses on basic analysis, and it suffices for most real-world applications regarding the calculus of areas and volumes. Mathematically, however, Riemann integration presents some problems, which become apparent when trying to integrate non-continuous functions.

Lebesgue integration and measure theory appeared to try to solve these problems, generalising Riemann integration in almost all cases of interest, and being the foundations for more advanced mathematical theories in analysis. An example of this is functional analysis: \(L^p\) or Sobolev spaces are always defined in measure-theoretic terms \cite{brezis,rudin2006real}. 

This theory, abstract as it is, has many applications to the real world, one of which is image processing: state-of-the-art variational models think of images as the minimisers of some energy functional defined on some function space, always on measure-theoretic terms \cite{hammond2024two,duran2014nonlocal}.

Another application of measure theory is found in probability theory: in his ground-breaking work \cite{Grundbegriffe}, Kolmogorov defined probability theory in terms of the abstract integration and measure theory developed years earlier by Lebesgue, Fréchet et al. %TODO: Cita hist.
His approach has become the standard way of understanding probability theory today.

The goal of this work is to provide a solid background in abstract Measure Theory that allows the study of more complex topics in analysis, as well as developing the theory and language necessary to develop an important theorem in probability theory: the \hyperref[theorem:Kolmogorov Extension]{Kolmogorov Extension Theorem}.


The text is structured as follows: the first half of the text is composed entirely by \Cref{chapter:elementary measure theory}, which is devoted to the development of basic definitions and results in measure theory, as well as the treatment of the abstract Lebesgue integral. Then, tools required for the proof of the Kolmogorov Extension Theorem are detailed in \Cref{chapter:advanced results in measure theory}. In \Cref{chapter:product spaces}, we study a systematic way to talk about measures in higher dimensions, and obtain the Kolmogorov Extension Theorem as a final result. Finally, in \Cref{chapter:an application to real analysis and probability}, we use all the theory developed so far to obtain some applications to real analysis and probability theory. The formal dependencies between sections are represented in the following figure:

% https://q.uiver.app/#q=WzAsMTUsWzAsMSwiMi4xIl0sWzEsMSwiMi4yIl0sWzIsMCwiMi4zIl0sWzIsMSwiQSJdLFsyLDIsIjIuNCJdLFsyLDMsIkIuMSJdLFszLDMsIkIuMiJdLFszLDIsIjQuMSJdLFszLDAsIjMuMSJdLFs0LDAsIjMuMiJdLFs0LDIsIjUuMSJdLFs0LDMsIkMiXSxbNSwyLCI1LjIiXSxbNCwxLCI0LjIiXSxbMCwwLCIxIl0sWzAsMV0sWzEsMl0sWzMsMl0sWzMsNF0sWzEsNF0sWzQsNV0sWzUsNl0sWzQsN10sWzIsN10sWzQsOF0sWzIsOF0sWzgsOV0sWzEwLDEyXSxbNywxM10sWzksMTNdLFsxMywxMl0sWzcsMTBdLFsxMCwxMV1d
\begin{figure}[h!]
\centering
\begin{tikzcd}
	1 && {2.3} & {3.1} & {3.2} \\
	{2.1} & {2.2} & A && {4.2} \\
	&& {2.4} & {4.1} & {5.1} & {5.2} \\
	&& {B.1} & {B.2} & C
	\arrow[from=2-1, to=2-2]
	\arrow[from=2-2, to=1-3]
	\arrow[from=2-3, to=1-3]
	\arrow[from=2-3, to=3-3]
	\arrow[from=2-2, to=3-3]
	\arrow[from=3-3, to=4-3]
	\arrow[from=4-3, to=4-4]
	\arrow[from=3-3, to=3-4]
	\arrow[from=1-3, to=3-4]
	\arrow[from=3-3, to=1-4]
	\arrow[from=1-3, to=1-4]
	\arrow[from=1-4, to=1-5]
	\arrow[from=3-5, to=3-6]
	\arrow[from=3-4, to=2-5]
	\arrow[from=1-5, to=2-5]
	\arrow[from=2-5, to=3-6]
	\arrow[from=3-4, to=3-5]
	\arrow[from=3-5, to=4-5]
\end{tikzcd}
\caption{Formal dependencies between the sections of this work. An arrow \(X\to Y\) means that a part of section \(X\) is needed to study \(Y\).}
\end{figure}

Most of the content of the text was written following the excellent structure, explanation and exposition of \cite{ash1972real}. Any praise on the quality of said book would be an understatement; simply put, it is the perfect material for an undergraduate student who wants to gain some insight into measure theory, real analysis and probability theory. Sometimes, however, the approach followed was original; in those cases, it has been indicated in the text.

It should be noted that the Axiom of Choice is used indiscriminately throughout the text. Two logical equivalences used are Zorn's Lemma (in the proof of the \hyperref[theorem:Radon-Nikodym]{Radon-Nikodým Theorem}) and the statement that every (infinite) cartesian product of nonempty set is nonempty (in \cref{section:infinite products}).

%!TeX root=Final.tex

\chapter{BASIC MEASURE THEORY}\label{chapter:elementary measure theory}

\section{Notation}

Let \(\mathbb{R}\) be the set of real numbers. We will define the
\textbf{extended real line} in the usual way; that is, as the set
\(\overline{\mathbb{R}}=\mathbb{R}\cup\left\{+\infty\right\}\cup\left\{-\infty\right\}\), where
\(+\infty\) and \(-\infty\) are formal symbols for which we extend the order and arithmetic of \(\mathbb{R}\) in the following manner:
\[
\begin{array}{l}
		\forall a\in\mathbb{R}\colon-\infty<a<+\infty\\
		+\infty+\left(+\infty\right)=+\infty,~~~~~ -\infty+(-\infty)=-\infty
		,~~~~~ 0\cdot\left(+\infty\right)=\left(+\infty\right)\cdot0=0\cdot(-\infty)=(-\infty)\cdot0=0\\
		\forall a\in\mathbb{R}\colon -\infty+a=a+(-\infty)=-\infty,~~~~~+\infty+a=a+(+\infty)=+\infty,~~~~~\frac{a}{+\infty}=\frac{a}{-\infty}=0\\
		\forall b\in\overline{\mathbb{R}}\setminus\left\{0\right\}\colon \left(+\infty\right)\cdot b=b\cdot\left(+\infty\right)=\left\{
		\begin{array}{rl} +\infty,&b>0\\ -\infty,&b<0
		\end{array} \right.,  (-\infty)\cdot b=b\cdot(-\infty)=\left\{
		\begin{array}{rl} -\infty,&b>0\\ +\infty,&b<0
		\end{array} \right.
\end{array}
\]
We say that \(a<\infty\) whenever \(a\neq-\infty\) and \(a\neq+\infty\). If
there is no confusion with the previous notation, the symbol \(+\infty\) is
often also denoted as just \(\infty\). The expressions
\(+\infty+(-\infty), -\infty+(+\infty), \frac{+\infty}{+\infty}, \frac{-\infty}{+\infty}, \frac{+\infty}{-\infty}\)
and \(\frac{-\infty}{-\infty}\) are not defined. We say that an expression or
arithmetic operation in \(\overline{\mathbb{R}}\) is \textbf{well-defined} (or, simply,
\textbf{defined}), if it is not one of the above undefined expressions.  We also
extend the notion of \textbf{interval} to \(\overline{\mathbb{R}}\) in an
intuitive way: given \(a,b\in\overline{\mathbb{R}}\), define
\(\left[a,b\right]=\left\{x\in\overline{\mathbb{R}}\colon a\leq x\leq b\right\}\),
\(\left(a,b\right)=\left\{x\in\overline{\mathbb{R}}\colon a<x<b\right\}\) and
other kinds of intervals similarly. The class of all intervals of the
form \(\left(a,b\right)\), \(\left[-\infty,b\right)\) or \(\left(a,+\infty\right]\) with
\(a,b\in\overline{\mathbb{R}}\), \(a\leq b\), forms the basis of a topology on
\(\overline{\mathbb{R}}\), which we will call the \textbf{standard topology} on
\(\overline{\mathbb{R}}\). Intervals of the form \((a,b]\) or
\([a,b]\) with \(a,b\in\overline{\mathbb{R}}\), \(a\leq b\), will be of special interest,
and they will be called \textbf{right-semiclosed} intervals.

Let \(A, A_1,\dots,A_n,\dots\) be subsets of some nonempty set \(\Omega\). We say
that the sets \(A_n\) form an \textbf{increasing} sequence of sets whenever
\(A_1\subseteq\dots\subseteq A_n\subseteq\dots\) If \(A=\bigcup_nA_n\), we
denote it by \(A_n\uparrow A\).  We say that the sets \(A_n\) form a
\textbf{decreasing} sequence of sets whenever
\(A_1\supseteq\dots\supseteq A_n\supseteq\dots\) If \(A=\bigcap_nA_n\), we
denote it by \(A_n\downarrow A\).	We denote their \textbf{upper} and \textbf{lower limits} as
\(\limsup_{n}A_{n}=\bigcap_{n}\bigcup_{k\geq n}A_{k}\) and
\(\liminf_{n}A_{n}=\bigcup_{n}\bigcap_{k\geq n}A_{k}\), respectively.

Similarly, if \(f_1,f_2,\dots\) form an increasing (decreasing) sequence of (extended) real-valued functions
with limit \(f\), we may write \(f_n\uparrow f\) (\(f_n\downarrow f\)). The same
convention is used for monotone sequences of (extended) real numbers.
	
Sometimes, if a sequence of sets, functions or numbers has more than
one index, (i.e., \(\{A_{nm}\}_{n,m\in\mathbb{N}}\)) and is monotone with
respect to one, the index is specified as a subindex of the arrow (i.e.,
\(A_{nm}\uparrow_{n}B_{m}, \) where \(B_{m}=\bigcup_{n}A_{nm}\)).
	
If \(f\) and \(g\) are functions from some set \(\Omega\) to \(\overline{\mathbb{R}}\),
statements such as \(f\leq g\) or \(f=g\) are always interpreted pointwise, that
is, \(f(\omega)\leq g(\omega)\) or \(f(\omega)=g(\omega)\) for all
\(\omega\in\Omega\). The same is true for expressions like the limit of a
sequence of functions or its supremum. We say that a function \(f\) is
\textbf{positive} (resp., \textbf{negative}) if \(f>0\) (\(f<0\)), and
\textbf{nonnegative} (\textbf{nonpositive}) if \(f\geq0\) (\(f\leq0\)). If \(f\colon\Omega\to\overline{\mathbb{R}}\), its \textbf{positive}
and \textbf{negative parts} are defined by \(f^+(\omega)=\max(f(\omega),0)\),
\(f^-(\omega)=-\min(f(\omega),0)=\max(-f(\omega),0)\).

The notation for \(\left\{f\geq 0\right\}\) (and similar expressions) is interpreted likewise:
\(\left\{f\geq 0\right\}=\left\{\omega\in\Omega\left|f(\omega)\geq 0\right.\right\}\). Also, if \(B\subseteq\overline{\mathbb{R}}\), we write \(\left\{f\in B\right\}=\left\{\omega\in\Omega\left|f(\omega)\in B\right.\right\}=f^{-1}(B)\).
	
Finally, if \(A\subseteq\Omega\), the \textbf{indicator function} of \(A\) is defined
as \(I_A(\omega)=1\) if \(\omega\in A\) and \(I_A(\omega)=0\) otherwise.


\section{Introductory results and definitions}\label{section:introductory results and definitions}
	
In this section, we introduce the basic concepts regarding measure theory:
fields, \(\sigma\)-fields and measures, among others. As we will see, there is a strong algebraic component in measure theory. For instance, it is not possible
to assign an \emph{intuitive}\footnote{More concretely, if using the axiom of choice, it is not possible to define a measure on the power set \(\mathcal{P}(\mathbb{R})\) that assigns its length to each interval.} measure every subset of \(\mathbb{R}\) (see 2.18 in \cite{axler}), so it is necessary to find 
suitable structures on which to develop the theory. This will be the role played
by the concepts introduced here:
	
\begin{defn}
Let \(\Omega\) be an arbitrary set and let \(\mathcal{F}\) be a class of subsets of ~\(\Omega\). We say that \(\mathcal{F}\) is
\begin{itemize}
		\item A \textbf{field} or \textbf{algebra} if \(\emptyset\in\mathcal{F}\) and
				it is closed under finite unions and complementation, that is, if we consider any two 
				\(A,B\in\mathcal{F}\), then \(A^c\in\mathcal{F}\) and \(A\cup B\in\mathcal{F}\).
		\item 
				A \(\bm{\sigma}\)\textbf{-field} or \(\bm{\sigma}\)\textbf{-algebra} if
				\(\emptyset\in\mathcal{F}\) and it is closed under countable unions and complementation,
				that is, if we consider any sequence of sets \(A_1,A_{2},\dots\in\mathcal{F}\), then 
				\(A_1^c\in\mathcal{F}\) and \(\bigcup_nA_n\in\mathcal{F}\).
		\item \textbf{Monotone} if it is closed under monotone sequences, that is, if 
				\(A_n\in\mathcal{F}\) and \(A_n\downarrow A\) or \(A_n\uparrow A\), then
				\(A\in\mathcal{F}\).
\end{itemize}
\end{defn}

From this definition in follows easily that a field is closed under finite
intersection, a \(\sigma\)-field is closed under countable intersection and that
a class of sets is a \(\sigma\)-field if, and only if, it is both monotone and a
field.
	
It is easy to check that the arbitrary intersection of fields,
\(\sigma\)-fields or monotone classes is, respectively, a field,
\(\sigma\)-field or monotone class.  This allows us to ensure the existence of the
respective structure containing a given class
\(\mathcal{S}\) of subsets of \(\Omega\). We shall call them the
\textbf{minimal} field, \(\sigma\)-field or monotone class over \(\mathcal{S}\)
or say that they are \textbf{generated by}
\(\mathcal{S}\). They will be written, respectively, as
\(\mathcal{F}(\mathcal{S}), \sigma(\mathcal{S})\) and \(\mathcal{C}(\mathcal{S})\). We
say that \(\mathcal{S}\) \textbf{generates}
\(\mathcal{F}(\mathcal{S}), \sigma(\mathcal{S})\) and \(\mathcal{C}(\mathcal{S})\). We will often used reasonings involving this concept. One that is almost self-evident - but very useful and common - is the following:
\begin{remk}\label{remark:generated structures}
		Let \(\mathcal{S}\) be a class of subsets of ~\(\Omega\). If \(\mathcal{A}\) is another class of subsets of ~\(\Omega\) that has a given structure and contains \(\mathcal{S}\), then the respective structure generated by \(\mathcal{S}\) is also contained in \(\mathcal{A}\); say, \(\mathcal{A}\) is a \(\sigma\)-field. Then, \(\mathcal{S}\subseteq\mathcal{A}\) implies \(\sigma(\mathcal{S})\subseteq\mathcal{A}\).
\end{remk}
	
Let \(X\) be a topological space. The class of \textbf{Borel sets} of \(X\),
denoted by \(\mathscr{B}(X)\), is the smallest \(\sigma\)-field containing all
open sets of \(X\). The Borel sets of \(\mathbb{R}\) and
\(\overline{\mathbb{R}}\) are of special interest, and having a small enough class of generators will be very convenient.
	
\begin{prop}\label{proposition:Borel sets on R} The class of
Borel sets of ~\(\mathbb{R}\) is generated by the class of all intervals of a
specified form. Namely, every family of intervals of one of the following forms:
\[
  \begin{array}{llll}
\text{(i) }\left(-\infty,b\right), ~ b\in\mathbb{R}&
\text{(ii) }\left(a,+\infty\right),~ a\in\mathbb{R}&
\text{(iii) }\left(a,b\right),~ a\leq b\in\mathbb{R}&
\text{(iv) }\left[a,b\right],~ a\leq b\in\mathbb{R}\\
\text{(v) }\left(a,b\right],~ a\leq b\in\mathbb{R}&
\text{(vi) }\left[a,b\right),~ a\leq b\in\mathbb{R}&
\text{(vii) }\left(-\infty,b\right],~ b\in\mathbb{R}&
\text{(viii) }\left[a,+\infty\right),~ a\in\mathbb{R}
  \end{array}
\]
\end{prop}
\begin{proof}
We will first prove the result for open, bounded intervals. Let \(\mathcal{F}\) denote the \(\sigma\)-field generated by all open intervals, and let \(\mathcal{T}\) be the standard topology on \(\mathbb{R}\) (that is, the class of all open sets). It is known that the cartesian product of countable sets is countable. Thus, every subset of \(\mathbb{Q}\times\mathbb{Q}\) is either finite or
countable.
		
Let \(U\in\mathcal{T}\). Now note that, by the density
of the rationals and the fact that \(U\) is open,
\[
		U=\bigcup_{a,b}~(a,b)
,\]
where \(a\) and \(b\) range over the pairs of rational numbers such that \(a<b\) and \((a,b)\subseteq U\) (which there is, at most, countably many of). Hence, \(\mathcal{T}\subseteq\mathcal{F}\). Since \(\mathcal{F}\) is a \(\sigma\)-field containing \(\mathcal{T}\), by \cref{remark:generated structures}, \(\sigma(\mathcal{T})\subseteq\mathcal{F}\). But, by definition, \(\mathscr{B}\left(\mathbb{R}\right)=\sigma(\mathcal{T})\). Reciprocally, since \(\mathscr{B}\left(\mathbb{R}\right)\) contains all open intervals, the smallest \(\sigma\)-field containing all open intervals, \(\mathcal{F}\), satisfies \(\mathcal{F}\subseteq\mathscr{B}\left(\mathbb{R}\right)\).
    
This completes the proof for open, bounded intervals. To
see it for other kinds of intervals, simply note that any interval can be
expressed by finite or countable unions and intersections of any other given
kind of intervals or their complements. For instance,
\((a,b]=\bigcap_{n}{(a,b+1/n)}\) or
\(\left[a,b\right)=[a,+\infty)\cap \left([b,+\infty)\right)^{c}\).
\end{proof}
\begin{prop}\label{proposition:Borel sets on RB}
		The class of Borel sets of ~\(\overline{\mathbb{R}}\) is generated by the class of all intervals of one of these forms:
\[
  \begin{array}{llll}
\text{(i) }\left[-\infty,b\right], ~ b\in\overline{\mathbb{R}}&
\text{(ii) }\left[-\infty,b\right),~ a\in\overline{\mathbb{R}}&
\text{(iii) }\left(a,+\infty\right],~ a\in\overline{\mathbb{R}}&
\text{(iv) }\left[a,+\infty\right],~ a\in\overline{\mathbb{R}}\\
  \end{array}
\]
\end{prop}
\begin{proof}
		Note that the singletons \(\{-\infty\}\) and \(\{+\infty\}\) are closed in \(\overline{\mathbb{R}}\), since their complements \((-\infty,+\infty]\) and \([-\infty,+\infty)\) are open. It is also not hard to see that every open subset of \(\mathbb{R}\) is open if regarded as a subset of \(\overline{\mathbb{R}}\).

		Define \(\mathcal{M}=\left\{B\in\mathscr{B}\left(\overline{\mathbb{R}}\right)\left|B\setminus\left\{-\infty,+\infty\right\}\in\mathscr{B}\left(\mathbb{R}\right)\right.\right\}\). It is now easy to see that \(\mathcal{M}\) is a \(\sigma\)-field containing all open sets of \(\overline{\mathbb{R}}\), hence \(\mathscr{B}\left(\overline{\mathbb{R}}\right)\subseteq\mathcal{M}\). Since \(\mathcal{M}\subseteq\mathscr{B}\left(\overline{\mathbb{R}}\right)\) by definition, it follows that \(B\) is a Borel set of \(\overline{\mathbb{R}}\) if, and only if, \(B\setminus\left\{-\infty,+\infty\right\}\) is a Borel set of \(\mathbb{R}\); hence, \(C\in\mathscr{B}\left(\mathbb{R}\right)\) if, and only if, \(C, C\cup\left\{-\infty\right\}, C\cup\left\{+\infty\right\}\) and \(C\cup\left\{-\infty,+\infty\right\}\in\mathscr{B}\left(\overline{\mathbb{R}}\right)\).

Let \(\mathcal{I}\) be any of the classes of sets (i)-(iv), and denote \(\mathcal{I}'=\left\{B\setminus\left\{-\infty,+\infty\right\}\left|B\in\mathcal{I}\right.\right\}\). It is clear that \(\sigma(\mathcal{I})\subseteq\mathscr{B}\left(\overline{\mathbb{R}}\right)\), and by \Cref{proposition:Borel sets on R}, \(\sigma(\mathcal{I}')=\mathscr{B}\left(\mathbb{R}\right)\). Finally, define \(\mathcal{M}'=\left\{C\in\mathscr{B}\left(\mathbb{R}\right)\left|C, C\cup\left\{-\infty\right\}, C\cup\left\{+\infty\right\} \text{ and } C\cup\left\{-\infty,+\infty\right\}\in\sigma\left(\mathcal{I}\right)\right.\right\}\). It is clear, by the form of \(\mathcal{I}\), that \(\mathcal{M}'\) is a \(\sigma\)-field containing \(\mathcal{I}'\); hence, \(\mathcal{M}'=\mathscr{B}\left(\mathbb{R}\right)\). This implies, however, that \(\mathscr{B}\left(\overline{\mathbb{R}}\right)\subseteq\sigma(\mathcal{I})\).
\end{proof}

\begin{defn} Let \(\Omega\) be a nonempty set and \(\mathcal{S}\) a class of subsets of ~\(\Omega\). 
A \textbf{set function} on \(\mathcal{S}\) (or, simply, a set function) is a
mapping \(\lambda\colon\mathcal{S}\to\overline{\mathbb{R}}\).	We say that a set function \(\lambda\) is
\textbf{finitely additive}, or simply \textbf{additive}, if the values
\(+\infty\) and \(-\infty\) do not both belong to the image of \(\lambda\),
there exists some \(A\in\mathcal{S}\) such that \(\lambda(A)\) is finite\footnote{This
condition is not imposed in some literature, allowing the set functions
\(\lambda_{1}(A)=+\infty\) and \(\lambda_{2}(A)=-\infty\) to be counted as
additive, since they satisfy all other requirements, but they are degenerate 
cases and will be excluded in this text. The main reason why we consider them degenerate cases is that \(\lambda_i(\emptyset)\neq 0\). On the contrary, if there exists some \(A\) such that \(\lambda(A)<\infty\), then \(\lambda(\emptyset)=0\) (see \Cref{proposition:measure of empty set is 0}).} and
		\begin{equation}\label{equation:additivity definition} 
				\lambda\left(\bigcup_{n}A_{n}\right)=\sum_{n}\lambda(A_{n})
		\end{equation}
		
		for every finite family of disjoint sets \(A_{1},A_{2},\dotsc\in\mathcal{S}\)
such that ~\(\bigcup_{n}A_{n}\in\mathcal{S}\) (this is always the case if \(\mathcal{S}\) is a field).
If condition (\ref{equation:additivity definition}) instead holds for every countable
family of subsets whose union belongs to \(\mathcal{S}\) (this will always be the case if \(\mathcal{S}\) is a \(\sigma\)-field), we say that \(\lambda\) is \textbf{countably additive} or
\textbf{\(\bm{\sigma}\)-additive}\footnote{Note that a necessary condition for
a \(\sigma\)-additive function to be well-defined is that for every sequence of
sets \(A_{1},A_{2},\dots\) such that \(\bigcup_{n}A_{n}\in\mathcal{F}\),
\(\forall n:\lambda(A_{n})<\infty\) and
\(\lambda(\bigcup_{n}A_{n})<\infty\), the series of real numbers
\(\sum_{n}\lambda(A_{n})\) is absolutely convergent, because
\(\bigcup_{n}A_{n}\) is invariant under permutations of indices, while
the series \(\sum_{n}\lambda(A_{n})\) is only invariant under permutations of indices if it is
absolutely convergent.}.
		
If \(\mathcal{S}\) is a \(\sigma\)-field, then a nonnegative, countably additive set
function \(\mu\) is called a \textbf{measure} on \(\mathcal{S}\). A measure satisfying
\(\mu(\Omega)=1\) is called a \textbf{probability measure} or, simply, a
\textbf{probability}.
\end{defn}
\begin{defn} A \textbf{measurable space} is a pair
\(\left(\Omega,\mathcal{F}\right)\), where \(\Omega\) is a nonempty set and \(\mathcal{F}\) is a \(\sigma\)-field of subsets of
\(\Omega\). A \textbf{measure space} is a tuple \(\left(\Omega,\mathcal{F},\mu\right)\),
where \(\left(\Omega,\mathcal{F}\right)\) is a measurable space and \(\mu\) is a measure
on \(\mathcal{F}\). A \textbf{probability space} is a measure space
\(\left(\Omega,\mathcal{F},p\right)\) where \(p\) is a probability measure on \(\mathcal{F}\).
\end{defn}
\begin{prop} Let \(\lambda\) be a finitely additive set function 
on the field \(\mathcal{F}_0\). Then,
\begin{enumerate}
			\item \label{proposition:measure of empty set is 0}
\(\lambda(\emptyset)=0\)
			\item \label{proposition:inclusion-exclusion}
\(\lambda(A\cup B)+\lambda(A\cap B)=\lambda(A)+\lambda(B)\) for all
\(A, B\in\mathcal{F}_0\).
			\item \label{proposition:monotonicity of additive set functions} If
\(A, B\in\mathcal{F}_0\) and \(A\subseteq B\), then
\(\lambda(B)=\lambda(A)+\lambda(B\setminus A)\). In particular,
\(\lambda(B)\geq\lambda(A)\) if \(\lambda(B\setminus A)\geq0\) and
\(\lambda(B\setminus A)=\lambda(B)-\lambda(A)\) if \(\lambda(A)<\infty\).
			\item \label{proposition:properties of additive set functions 1-4}If
\(\lambda\) is nonnegative,
			\[ \lambda\left(\bigcup_{k=1}^{n}A_{k}\right)\leq\sum_{k=1}^{n}\lambda(A_{k}) ~~\text{
for all } A_{1},\dots,A_{n}\in\mathcal{F}_0
			\]
			\item \label{proposition:properties of additive set functions 1-5}If
\(\lambda\) is a measure,
			\[ \lambda\left(\bigcup_{n}A_{n}\right)\leq\sum_{n}\lambda(A_{n})
			\] for all \(A_{1},A_{2},\dots\in\mathcal{F}\) such that
\(\bigcup_{n}A_{n}\in\mathcal{F}\).
		\end{enumerate}
\end{prop}
\begin{proof}
		\begin{enumerate}
			\item Take any set \(A\in\mathcal{F}\) such that \(\lambda(A)<\infty\). Then,
			\[
					\lambda(A)=\lambda(A\cup\emptyset)=\lambda(A)+\lambda(\emptyset)
			,\]
			whence \(\lambda(\emptyset)=0\).
			\item Note that \(\lambda(A\cup B)=\lambda(A\setminus B)+\lambda(B\setminus A)+\lambda(A\cap B)\). Therefore,
			\[
					\lambda(A\cup B)+\lambda(A\cap B)=\left(\lambda(A\setminus B)+\lambda(A\cap B)\right)+\left(\lambda(B\setminus A)+\lambda(A\cap B)\right)=\lambda(A)+\lambda(B).
			\]
			\item Immediate by additivity.
			\item Write \(B_{n}=A_{n}\setminus \left(A_{1}\cup\dots\cup A_{n-1}\right)\in\mathcal{F}\). Since \(B_{n}\subseteq A_{n}\), by \ref{proposition:monotonicity of additive set functions}, \(\lambda(B_{n})\leq\lambda(A_{n})\). Note that the sets \(B_{n}\)
are disjoint and their union is \(\bigcup_{n}A_{n}\). Thus,
			\[ \lambda\left(\bigcup_{n}A_{n}\right)=\sum_{n}\lambda(B_{n})\leq\sum_{n}\lambda(A_{n}).
			\]
	\item The proof given for \ref{proposition:properties of additive set functions 1-4} still holds word for word (now the union is infinite, but the notation used is the same).
		\end{enumerate}
\end{proof}
\begin{defn} A set function \(\lambda\) defined on a class \(\mathcal{S}\) of
subsets of \(\Omega\) is said to be \textbf{finite} if \(\lambda(A)<\infty\) for
every \(A\in\mathcal{S}\). If \(\mathcal{F}_0\) is a field and \(\lambda\) is finitely additive, it
suffices that \(\lambda(\Omega)\) be finite, for
\(\lambda(\Omega)=\lambda(A)+\lambda(A^{c})\); and if \(\lambda(A)\) is infinite,
so is \(\lambda(\Omega)\).
		
A nonnegative, finitely additive set function \(\lambda\) on a field
\(\mathcal{F}_0\) is said to be \(\bm{\sigma}\)\textbf{-finite} whenever \(\Omega\) can be
written as \(\bigcup_{n}A_{n}\), where \(A_n\in\mathcal{F}_0\)\footnote{The condition that \(A_n\in\mathcal{F}_0\) is important. The following scenario will be quite common in the rest of the text: we have a \(\sigma\)-field \(\mathcal{F}\), and a field \(\mathcal{F}_0\) such that \(\mathcal{F}=\sigma(\mathcal{F}_0)\), and we have information on \(\mathcal{F}_0\) that we want to extend to the whole \(\sigma\)-field \(\mathcal{F}\). It is a requirement for some theorems (see the \hyperref[theorem:Caratheodory Extension]{Carathéodory Extension Theorem}) that a measure is \(\sigma\)-finite specifically over \(\mathcal{F}_0\), and not over the whole \(\sigma\)-field \(\mathcal{F}\).} and \(\lambda(A_{n})<\infty\) for every \(n\).
\end{defn}
\begin{remk}\label{remark:finiteness of subsets} Let \(\lambda\) be a
finitely additive set function on a field \(\mathcal{F}_0\). Then, consider \(A,B\in\mathcal{F}_0\) such that \(A\subseteq B\). If
\(\lambda(A)=\pm\infty\), \ref{proposition:monotonicity of additive set
functions} implies that
\(\lambda(B)=\lambda(A)+\lambda(B\setminus A)=\pm\infty+\lambda(B\setminus A)=\pm\infty\).
As a consequence, if \(\lambda(B)<\infty\), then \(\lambda(A)<\infty\).
		
		Another interesting thing to note is that, by
\cref{proposition:monotonicity of additive set functions}, every finite measure
is bounded. We will see later on that this is the case too for countably
additive set functions.
\end{remk}
One of the most common processes in analysis is taking limits. The following definition and the subsequent two propositions, which will be of great usefulness during the rest of the text, relate this process to the language we have been developing.
	
\begin{defn} A set function \(\lambda\) defined on some class of
subsets \(\mathcal{S}\) is said to be \textbf{continuous from below} at a given
\(A\in\mathcal{S}\) whenever \(\lim_{n}\lambda(A_{n})=\lambda(A)\) for every increasing
sequence of sets \(A_{n}\uparrow A\), with \(A_{n}\in\mathcal{S}\) for all \(n\). Is is
said to be \textbf{continuous from above} at a given \(A\in\mathcal{S}\) whenever
\(\lim_{n}\lambda(A_{n})=\lambda(A)\) for every decreasing sequence of sets
\(A_{n}\downarrow A\), with \(A_{n}\in\mathcal{S}\) for all \(n\).
\end{defn}
\begin{prop}\label{proposition:limits of monotone sequences of sets} Let
\(\lambda\) be a \(\sigma\)-additive set function on a field \(\mathcal{F}_0\). Then,
		\begin{enumerate}
			\item \label{proposition:limit of increasing sets} \(\lambda\) is
continuous from below at every \(A\in\mathcal{F}_0\); that is, if \(A_{n}\uparrow A\) and
\(A\in\mathcal{F}_0\), then \(\lim_{n}\lambda(A_{n})=\lambda(A)\).
			\item \label{proposition:limit of decreasing sets} \(\lambda\) is
continuous from above at every \(A\in\mathcal{F}_0\) with \(\lambda(A)<\infty\) if we only
consider decreasing sequences \(A_{1},A_{2},\dotsc\in\mathcal{F}_0\) such that
\(\lambda(A_{1})<\infty\). More concretely : if \(A_{n}\downarrow A\),
\(A\in\mathcal{F}_0\) and \(\lambda(A_{1})<\infty\), then
\(\lim_{n}\lambda(A_{n})=\lambda(A)\).
		\end{enumerate}
\end{prop}
\begin{proof}
		\begin{enumerate}
			\item Define \(B_1=A_1\),
\(B_{n}=A_{n}\setminus \left(A_{1}\cup\dotsc\cup A_{n-1}\right)=A_{n}\setminus A_{n-1}\in\mathcal{F}_0\),
so that the sets \(B_{n}\) are disjoint and \(A_{n}=B_{1}\cup\dotsc\cup B_{n}\).
Therefore, by additivity, \(\lambda(A_{n})=\sum_{k=1}^{n}\lambda(B_{k})\).
Finally, since \(A=\bigcup_{n}B_{n}\),
			\[
					\lim_{n}\lambda(A_{n})=\sum_{k=1}^{+\infty}\lambda(B_{k})=\lambda(A)
			\]
			\item Define \(C_{n}=A_{1}\setminus A_{n}\). Since
\(A_{n}\subseteq A_{1}\) and \(A\subseteq A_{1}\), by \cref{remark:finiteness of
subsets}, \(\lambda(A_{n})<\infty\) and \(\lambda(A)<\infty\). Thus, by
\cref{proposition:monotonicity of additive set functions},
\(\lambda(C_{n})=\lambda(A_{n})-\lambda(A_{1})\) and
\(\lambda(A_{1}\setminus A)=\lambda(A_{1})-\lambda(A)\). The desired result now
follows from \ref{proposition:limit of increasing sets} taking into
consideration that \(C_{n}\uparrow \left(A_{1}\setminus A\right)\).
		\end{enumerate}
\end{proof}
We can state a result which is in some way reciprocal to the previous one:
	
\begin{prop}\label{proposition:sigma-additivity and limits} Let \(\lambda\) be a finitely additive function defined
on a field \(\mathcal{F}_0\). Then,
		\begin{enumerate}
			\item \label{proposition:sigma-additivity from below} If \(\lambda\)
is continuous from below at every \(A\in\mathcal{F}_0\), then it is
\(\sigma\)-additive on \(\mathcal{F}_0\).
			\item \label{proposition:sigma-additivity from above} If \(\lambda\)
is continuous from above at the empty set, then it is
\(\sigma\)-additive on \(\mathcal{F}_0\).
		\end{enumerate}
\end{prop}
\begin{proof}
		\begin{enumerate}
			\item Let \(A_{1},A_{2},\dotsc\in\mathcal{F}_0\) be a sequence of disjoint sets
such that \(\bigcup_{n}A_{n}\in\mathcal{F}_0\). Define \(A=\bigcup_{n}A_{n}\) and
\(B_{n}=A_{1}\cup\dotsc\cup A_{n}\in\mathcal{F}_0\). Then,
\(B_{n}\uparrow \bigcup_{n}A_{n}\) and, by additivity,
\(\lambda(B_{n})=\sum_{k=1}^{n}\lambda(A_{k})\). Since \(\lambda\) is continuous
from below at \(\bigcup_{n}A_{n}\), we have
			\[ \lambda\left(\bigcup_{n}A_{n}\right)=\lim_{n}\lambda(B_{n})=\lim_{n}\sum_{k=1}^{n}\lambda(A_{k})=\sum_{n}\lambda(A_{n}).
			\]
			\item We will show that \(\lambda\) is continuous from below at
every \(A\in\mathcal{F}_0\): let \(A_{1},A_{2},\dotsc\in\mathcal{F}_0\) be a sequence of sets
increasing to \(A\in\mathcal{F}_0\). Define \(B_{n}=A\setminus A_{n}\). It is clear that
\(B_{n}\downarrow\emptyset\). Additionally,
\(\lambda(B_{n})+\lambda(A_{n})=\lambda(A)\). Since \(\lambda(B_{n})\to0\), it
must be \(\lambda(A_{n})\to\lambda(A)\). By \ref{proposition:sigma-additivity from below}, \(\lambda\) is \(\sigma\)-additive.
		\end{enumerate}
\end{proof}
\section{Extension of measures}\label{section:Extension of measures}

The goal of this section is to extend, under
certain technical hypotheses that will appear later on, a nonnegative,
\(\sigma\)-additive  set function \(\mu\) over a field \(\mathcal{F}_0\) into a measure
over a \(\sigma\)-field that contains \(\mathcal{F}_0\). Although this may seem artificial to the reader at the moment, it is very common in measure theory to find the need of extending a measure in this way. To this avail, we shall follow
the following scheme:

\begin{itemize}
	\item First, we restrict ourselves to the case where \(\mu\) is finite,
which, up to a rescaling, is equivalent to it being a probability measure. Then,
we extend \(\mu\) to the class of countable unions of sets of \(\mathcal{F}_0\), \(\mathcal{C}\),
by taking limits. This collection is closer to being a \(\sigma\)-field, but it
need not contain complements of sets in it.
	\item Secondly, we extend the function obtained to all subsets of
\(\Omega\), via approximating them by sets we can measure (sets in \(\mathcal{C}\)).
	\item This extension will turn out not to be a measure in all of
\(\mathcal{P}(\Omega)\), but we can find a subset where it \textit{is} a
measure, and said subset will turn out to be a \(\sigma\)-field containing
\(\mathcal{F}_0\).
	\item Finally, we are able to drop the finiteness restriction over \(\mu\)
and cover a more general case, by using the construction above. Moreover, the
construction made will allow us to conclude that said extension is, in fact,
unique.
\end{itemize}

Having a general overview of the ideas followed, let us now dive into the
details.

\begin{lemm}\label{lemma:inequality of limits of increasing sets} Let \(\mathcal{F}_0\)
be a field of subsets of a given set \(\Omega\), and let \(p\) be a nonnegative,
countably additive set function on \(\mathcal{F}_0\) such that \(p(\Omega)=1\). Let
\(A_1,A_2,\dots\) be family of sets that belong to \(\mathcal{F}_0\) and increase to a
limit \(A\). Take \(A_1',A_2',\dots\) and \(A'\) similarly (note that \(A\) and
\(A'\) need not belong to \(\mathcal{F}_{0}\)). If \(A\subseteq A'\), then
	
	\[\lim_np(A_n)\leq\lim_np(A_n').\]
\end{lemm}
\begin{proof} Firstly, note that both limits exist since \(\{p(A_n)\}_n\) and
\(\{p(A_n')\}_n\)  are both increasing sequences of real numbers, and bounded by
\(1\), following \cref{proposition:monotonicity of additive set functions}.
	
	Take \(m\in\mathbb{Z}^+\). Then, \(A_m\cap A_n'\uparrow_n A_m\cap A'=A_m\in\mathcal{F}\).
Thus, by \cref{proposition:limit of increasing sets},
	\[\lim_np(A_m\cap A_n')=p(A_m).\]
	
	But \(p(A_m\cap A_n')\leq p(A_n')\) (\cref{proposition:monotonicity of
additive set functions}), whence
	
	\[p(A_m)\leq\lim_n p(A_n').\]
	
	The result follows easily from taking limits when \(m\to\infty\) in the
above expression.
\end{proof}
Now we can extend \(p\) to a larger class of sets: the class of countable unions
of sets of \(\mathcal{F}_0\).

\begin{lemm}\label{lemma:extension to monotone class} Let \(\mathcal{C}\) be the class
of all countable unions of sets in \(\mathcal{F}_0\). Define
\(\mu\) on \(\mathcal{C}\) as follows: if \(A\in\mathcal{C}\), there exists a sequence of sets
\(A_n\in\mathcal{F}_0\) increasing to \(A\). Now set  \(\mu(A)=\lim_n p(A_n)\). This
limit exists since the sequence is increasing and bounded by \(1\) and \(\mu\)
is well-defined by \Cref{lemma:inequality of limits of increasing sets}.  Also,
clearly \(\mu\equiv p\) on \(\mathcal{F}_0\). Then:
	\begin{enumerate}
		\item \label{lemma:extension to monotone class
coincides}\(\emptyset,\Omega\in\mathcal{C}\), \(\mu(\emptyset)=0\), \(\mu(\Omega)=1\) and
\(0\leq\mu(A)\leq1\) for all \(A\in\mathcal{C}\)
		\item \label{lemma:extension to monotone class additivity} If
\(G_1,G_2\in\mathcal{C}\), then \(G_1\cup G_2,G_1\cap G_2\in\mathcal{C}\) and
\(\mu(G_1\cup G_2)+\mu(G_1\cap G_2)=\mu(G_1)+\mu(G_2)\).
		\item \label{lemma:extension to monotone class is monotone} If
\(G_1,G_2\in\mathcal{C}\) and \(G_1\subseteq G_2\), then \(\mu(G_1)\leq\mu(G_2)\).
		\item \label{lemma:extension to monotone class incerasing limits} If
\(G_n\in\mathcal{C}\), and \(G_n\uparrow G\), then \(G\in\mathcal{C}\) and
\(\mu(G_n)\to\mu(G)\).
	\end{enumerate}
\end{lemm}
\begin{proof}
	\begin{enumerate}
		\item Follows easily from the fact that \(\mu\equiv p\) in \(\mathcal{F}_0\),
taking into account that \(p\) is a probability measure.
		\item Let \(A_n^1\uparrow G_1\), \(A_n^1\in\mathcal{F}_0\), and
\(A_n^2\uparrow G_2\), \(A_n^2\in\mathcal{F}_0\). Then,
\(A_n^1\cup A_n^2\uparrow G_1\cup G_2\) and
\(A_n^1\cap A_n^2\uparrow G_1\cap G_2\). We have
		
		\[\forall n\in\mathbb{Z}^+:p(A_n^1\cup A_n^2)+p(A_n^1\cap A_n^2)=p(A_n^1)+p(A_n^2).\]
		
		The proof is completed by taking limits in the above expression.
		\item This follows easily from \Cref{lemma:inequality of limits of
increasing sets}.
		\item For each \(m\in\mathbb{Z}^+\), take \(A_{m,n}\uparrow_{n} G_m\),
\(A_{n,m}\in\mathcal{F}_0\). To see that \(G\in\mathcal{C}\), simply take any bijection
\(\rho\colon\mathbb{Z}^+\to\mathbb{Z}^+\times\mathbb{Z}^+\) and note that
\(G=\bigcup_{k\in\mathbb{Z}^+}A_{\rho(k)}\). Now define, for each \(m\in\mathbb{Z}^+\),
\(D_m=A_{1,m}\cup A_{2,m}\cup\dots\cup A_{m,m}\in\mathcal{F}_0\). Note that, by
definition, \(\forall n:A_{n,m}\subseteq D_m\), and
\(D_m\subseteq G_1\cup G_2\cup\dots\cup G_m=G_m\). Therefore, the following
inclusion holds for every \(n\):
		\begin{equation}\label{equation:extension to monotone class
1} A_{n,m}\subseteq D_m\subseteq G_m,
		\end{equation} and by \ref{lemma:extension to monotone class is
monotone}, we have \(\mu(A_{n,m})\leq \mu(D_m)\leq\mu(G_m)\), which can be
written as
		\begin{equation}\label{equation:extension to monotone class
2} p(A_{n,m})\leq p(D_m)\leq\mu(G_m)
		\end{equation} Let \(m\to+\infty\) in (\ref{equation:extension to monotone class 1}) to see that \(G_n\subseteq \bigcup_{m\in\mathbb{Z}^+}D_m\subseteq G\) and now
let \(n\to+\infty\) to see that \(D_m\uparrow G\), and thus
\(\mu(G)=\lim_m p(D_m)\). Finally, let \(m\to+\infty\) in (\ref{equation:extension to monotone class 2}) to see that \(\mu(G_n)\leq \mu(G)\leq\lim_m\mu(G_m)\) and
now let \(n\to+\infty\) to complete the proof.
	\end{enumerate}
\end{proof}
We now are going to extend \(\mu\) to the class of all subsets of
\(\Omega\). This construction only depends on properties \((i)-(iv)\) in
\Cref{lemma:extension to monotone class} and not on the initial definition of
\(\mu\).
\begin{lemm}\label{lemma:extension to outer measure properties} Let \(\mathcal{C}\) be a
class of subsets of a given set \(\Omega\), \(\mu\) a nonnegative real-valued
set function on \(\mathcal{C}\) such that \(\mathcal{C}\) and \(\mu\) satisfy the four conditions
\((i)-(iv)\) of \Cref{lemma:extension to monotone class}. Define, for each
\(A\subseteq\Omega\),
	
	\[\mu^*(A)=\inf\{\mu(G):G\in\mathcal{C},A\subseteq G\}.\]
	
	Then,
	\begin{enumerate}
		\item \label{lemma:extension to outer measure coincides and is bounded}
\(\mu^*\equiv\mu\) on \(\mathcal{C}\), and \(0\leq\mu^*(A)\leq1\) for all
\(A\subseteq\Omega\).
		\item \label{lemma:extension to outer measure
subadditivity}\(\mu^*(A\cup B)+\mu^*(A\cap B)\leq\mu^*(A)+\mu^*(B)\).
		\item \label{lemma:defining property of
Hcal}\(\mu^*(A)+\mu^*(A^c)\geq1\).
		\item \label{lemma:extension to outer measure monotonicity} If
\(A\subseteq B\), then \(\mu^*(A)\leq\mu^*(B)\).
		\item \label{lemma:extension to outer measure increasing limits} If
\(A_n\uparrow A\), then \(\mu^*(A_n)\to\mu^*(A)\).
	\end{enumerate}    
\end{lemm}
\begin{proof}
	\begin{enumerate}
		\item Take any set \(A\in\mathcal{C}\). Then, for all \(G\in\mathcal{C},A\subseteq G\), we
have \(\mu(A)\leq\mu(G)\) by \ref{lemma:extension to monotone class is
monotone}.  This lower bound is achieved by \(A\) itself, and thus
\(\mu^*(A)=\mu(A)\). Bounds \(0\) and \(1\) follow easily from the definition of
infimum.
		\item For any \(\varepsilon>0\), take \(G_1,G_2\in\mathcal{C}\) such that
\(A\subseteq G_1,B\subseteq G_2\) and
\(\mu(G_1)\leq\mu^*(A)+\varepsilon/2,\mu(G_2)\leq\mu^*(B)+\varepsilon/2\).
Therefore,
		\[\mu^*(A)+\mu^*(B)+\varepsilon\geq\mu(G_1)+\mu(G_2)=\mu(G_1\cup G_2)+\mu(G_1\cap G_2)\geq\mu^*(A\cup B)+\mu^*(A\cap B).\]
		
		Since \(\varepsilon>0\) is arbitrary, the result holds.
		\item Immediate consequence of \ref{lemma:extension to outer measure subadditivity}:
\(\mu^*(A)+\mu^*(A^c)\geq\mu^*(\Omega)+\mu^*(\emptyset)=1\), where
\(\mu^{*}(\emptyset)=0\) by \ref{lemma:extension to outer measure coincides and
is bounded}.
		\item Immediate by definition.
		\item First, note that \(\lim_n \mu^*(A_n)\) exists (we have an
increasing sequence of real numbers bounded by \(1\) following
\ref{lemma:extension to outer measure coincides and is bounded} and
\ref{lemma:extension to outer measure monotonicity}) and is bounded above by
\(\mu^*(A)\), because \(\forall n:\mu^{*}(A_{n})\leq \mu^{*}(A)\) by
\ref{lemma:extension to outer measure monotonicity}. Let \(\varepsilon>0\), and
define \(\varepsilon_n=\varepsilon/2^n\) (this choice is made so that
\(\sum_n\varepsilon_n=\varepsilon\)). Consider \(G_n\in\mathcal{C}\) such that
\(A_n\subseteq G_n\) and \(\mu(G_n)\leq\mu^*(A_n)+\varepsilon_n\). Now,
\(A=\bigcup_nA_n\subseteq\bigcup_nG_n\), whence
		\[\mu^*(A)\leq\mu^*\left(\bigcup_nG_n\right)=\mu\left(\bigcup_nG_n\right)=\lim_n\mu\left(\bigcup_{k=1}^nG_n\right),\]
where in the last step we used \cref{lemma:extension to monotone class incerasing
limits}. If we can show that
\(\mu\left(\bigcup_{k=1}^nG_k\right)\leq\mu^*(A_n)+\varepsilon\), the proof will
be done. With this in mind, it suffices to prove that
\(\mu\left(\bigcup_{k=1}^nG_k\right)\leq\mu^*(A_n)+\sum_{k=1}^n\varepsilon_k\)
for all \(n\in\mathbb{Z}^+\), since
\(\sum_{k=1}^n\varepsilon_k<\sum_{k}\varepsilon_{k}=\varepsilon\) . The case
\(n=1\) is true by construction. Now apply \cref{lemma:extension to monotone class additivity} to \(A=\bigcup_{k=1}^n G_k\) and \(B=G_{k+1}\):
\[
    \mu\left(\bigcup_{k=1}^{n+1}G_k\right)=\mu\left(\left(\bigcup_{k=1}^nG_k\right)\cup G_{n+1}\right)=\mu\left(\bigcup_{k=1}^nG_k\right)+\mu(G_{n+1})-\mu\left(\bigcup_{k=1}^nG_k\cap G_{n+1}\right)
.\]
Note that
\(A_n=A_n\cap A_{n+1}\subseteq G_n\cap G_{n+1}\subseteq \bigcup_{k=1}^nG_k\cap G_{n+1}\),
and that implies
\(\mu^*(A_n)\leq\mu\left(\bigcup_{k=1}^nG_k\cap G_{n+1}\right)\). Moreover,
\(\mu(G_{n+1})\leq\mu^*(A_{n+1})+\varepsilon_{n+1}\). Using both inequalities
and the induction hypothesis,
		\[\mu\left(\bigcup_{k=1}^{n+1}G_k\right)\leq \mu^*(A_n)+\sum_{k=1}^n\varepsilon_k+\mu^*(A_{n+1})+\varepsilon_{n+1}-\mu(A_n)=\mu^*(A_{n+1})+\sum_{k=1}^{n+1}\varepsilon_k,\]
thus completing the proof.
	\end{enumerate}    
\end{proof}
As discussed in the beginning of the section, this extension will, in
general, not be a measure over all of \(\mathcal{P}(\Omega)\). However, we can
find a suitable \(\sigma\)-field for it to be a measure on, following a somewhat
intuitive idea:

Were \(\mu^*\) to be a measure over a \(\sigma\)-field \(\mathcal{H}\), it would be
additive, and for each \(A\in\mathcal{H}\),
\begin{equation}\label{equation:definition of
Hcal} \mu^*(A)+\mu^*(A^c)=\mu^*(\emptyset)+\mu^*(\Omega)=1.
\end{equation}

Therefore, we can attempt to define \(\mathcal{H}\) as the class of subsets of
\(\Omega\) that satisfy (\ref{equation:definition of Hcal}).

\begin{thrm}\label{theorem:extension to Hcal} Under the hypotheses of
\cref{lemma:extension to outer measure properties}, let
	\[\mathcal{H}=\{A\subseteq\Omega:\mu^*(A)+\mu^*(A^c)=1\}\]
	
	(Following \cref{lemma:defining property of Hcal}, the defining property of
\(\mathcal{H}\) is equivalent to the weaker condition \(\mu^*(A)+\mu^*(A^c)\leq1\)).
Then, \(\mathcal{H}\) is a \(\sigma\)-field containing \(\mathcal{C}\) and \(\mu^*\) is a
probability measure on \(\mathcal{H}\).
\end{thrm}
\begin{proof} First, note that \(\mathcal{C}\subseteq \mathcal{H}\) by \cref{lemma:extension to
monotone class additivity}. We will show that \(\mathcal{H}\) is a field. Clearly, it
is closed under complementation. Let \(H_1,H_2\in\mathcal{H}\). Then, by
\cref{lemma:extension to outer measure subadditivity},
	\begin{equation}\label{equation:H inequalities}
		\begin{aligned} \mu^*(H_1\cup H_2)+\mu^*(H_1\cap H_2)\leq\mu^*(H_1)+\mu^*(H_2) \\ \mu^*(H_1^c\cup H_2^c)+\mu^*(H_1^c\cap H_2^c)\leq\mu^*(H_1^c)+\mu^*(H_2^c)
		\end{aligned}
	\end{equation}
	
	Adding both inequalities and taking into account that \(H_1,H_2\in\mathcal{H}\),
we get that
	\[\mu^*(H_1\cup H_2)+\mu^*((H_1\cup H_2)^c)+\mu^*(H_1\cap H_2)+\mu^*((H_1\cap H_2)^c)\leq 2.\]
	
	Define \(U=\mu^*(H_1\cup H_2)+\mu^*((H_1\cup H_2)^c)\) and
\(I=\mu^*(H_1\cap H_2)+\mu^*((H_1\cap H_2)^c)\) Following, \ref{lemma:defining
property of Hcal} \(U,I\geq 1\), which means that \(2\leq U+I\leq2\), whence
\(U=I=1\). It follows that \(H_1\cup H_2\in\mathcal{H}\), \(H_1\cap H_2\in\mathcal{H}\).
Furthermore, equalities in \eqref{equation:H inequalities} hold, for if inequalities
were strict, so would be the right inequality in \(2\leq U+I\leq2\). In the case
when \(H_1\) and \(H_2\) are disjoint, the first equality degenerates into
\(\mu^*(H_1\cup H_2)=\mu^*(H_1)+\mu^*(H_2))\). This shows both that \(\mathcal{H}\) is
a field and that \(\mu^{*}\) is additive on it.
	
Now consider the countable case. Let \(A_n\in\mathcal{H}\), \(A_n\uparrow A\)
(it is enough to consider this case since \(\mathcal{H}\) is a field). Note that
\(A^c\subseteq A_n^c\), so
	\[\mu^*(A_n)+\mu^*(A^c)\leq\mu^*(A_n)+\mu^*(A_n^c)=1.\] Taking limits, and
following \cref{lemma:extension to outer measure increasing limits},
\(\mu^*(A)+\mu^*(A^c)\leq 1\). Thus, \(A\in\mathcal{H}\). Moreover, \(\mu^*\) is
countably additive in \(\mathcal{H}\) by \cref{lemma:extension to outer measure
increasing limits} and \cref{proposition:sigma-additivity from below}.
\end{proof}
We can now state our first extension theorem.

\begin{thrm} A finite measure \(\mu\) on a field \(\mathcal{F}_0\) extends to a
measure on \(\sigma(\mathcal{F}_0)\).
\end{thrm}
\begin{proof} Scale the measure to a probability measure by considering
\(p=\mu/\mu(\Omega)\). Apply the construction made through \Cref{lemma:extension
to monotone class} to \Cref{theorem:extension to Hcal} to obtain a
\(\sigma\)-field \(\mathcal{H}\) containing \(\mathcal{F}_{0}\) and extend \(p\) to a measure
on \(\mathcal{H}\), \(\overline{p}\). Then, \(\mu(\Omega)\cdot \overline{p}\) is an
extension of \(\mu\) to \(\mathcal{H}\). Since \(\mathcal{H}\) is a \(\sigma\)-field
containing \(\mathcal{F}_0\) , we have \(\sigma(\mathcal{F}_0)\subseteq\mathcal{H}\), and we can restrict the
extension to \(\sigma(\mathcal{F}_0)\).
\end{proof}
We have developed all the tools we need to prove the existence part of our more
general extension theorem. However, a few extra results can be juiced out of the
construction, and some additional tools are required to prove uniqueness.
This is why we are going to introduce the concept
of \textit{completeness}:
\begin{defn}\label{definition:completion of a measure space} Let \((\Omega,\mathcal{F},\mu)\) be a measure space. A set
\(A\subseteq\Omega\) is said to be \textbf{null} whenever \(A\in\mathcal{F}\) and
\(\mu(A)=0\). The measure space is said to be \textbf{complete} if every subset
of a null set is measurable (and, therefore, null too).
	
Consider a measure space \( \left(\Omega,\mathcal{F},\mu\right)\), and let \(\mathcal{N}\) be the class of all subsets of null sets in
\((\Omega,\mathcal{F},\mu)\) (i.e. sets \(A\) such that \(A\subseteq B\) for some null set \(B\in\mathcal{F}\)).
	Define \(\mathcal{F}_\mu\) as the class of all sets of the form \(A\cup N\), where
\(A\in\mathcal{F}\) and \(N\in\mathcal{N}\), and extend \(\mu\) to \(\mathcal{F}_\mu\) as
\(\mu(A\cup N)=\mu(A)\). The measure space\footnote{Indeed, \(\mathcal{F}_\mu\) is a
\(\sigma\)-field: it is clearly closed under countable union, and closed under
complementation: if \(A\in\mathcal{F}_0\) and \(N\subseteq M\), \(M\in\mathcal{F}\), \(\mu(M)=0\),
then \((A\cup N)^c=(M^c\cap A^c)\cup (M\setminus (A\cup N))\), where
\(A^c\cap M^c\in\mathcal{F}\) and \(M\setminus (A\cup N)\subseteq M\). Also, \(\mu\) is well-defined on \(\mathcal{F}_\mu\): if
\(A_1\cup N_1=A_2\cup N_2\) with \(N_i\subseteq M_i\), \(\mu(M_i)=0\), then
\(A_1\subseteq A_1\cup N_1=A_2\cup N_2\subseteq A_2\cup M_2,\) whence
\[
		\mu(A_1)\leq\mu(A_2\cup M_2)\leq\mu(A_2)+\mu(M_2)=\mu(A_2).
\]
Similarly, \(\mu(A_2)\leq\mu(A_1)\). Checking that \(\mu\) satisfies the axioms
of a measure on \(\mathcal{F}_\mu\) is simple.} \((\Omega,\mathcal{F}_\mu,\mu)\) is said to be the
\textbf{completion} of \((\Omega,\mathcal{F},\mu)\).
\end{defn}
It is not complicated to show that if a measure is complete, then it is its own completion. Along these lines, an interesting thing to note about the completion of a measure space is
that it is minimal in the following sense:

\begin{remk}\label{remark:minimality of completion} Let
\((\Omega,\mathcal{F}_1,\mu_1)\) and \((\Omega,\mathcal{F}_2,\mu_2)\) be measure spaces over the
same set \(\Omega\). We say that \((\Omega,\mathcal{F}_2,\mu_2)\) extends
\((\Omega,\mathcal{F}_1,\mu_1)\) whenever \(\mathcal{F}_1\subseteq\mathcal{F}_2\) and
\(\mu_2\vert_{\mathcal{F}_1}\equiv\mu_1\). Then, every
complete extension of a measure space also extends its completion.
\end{remk}

With this in mind, we can show a curious relationship between the
\(\sigma\)-field given by \Cref{theorem:extension to Hcal} and \(\sigma(\mathcal{F}_0)\):

\begin{prop} In \Cref{theorem:extension to Hcal}, \((\Omega,\mathcal{H},\mu^*)\) is
the completion of \((\Omega,\sigma(\mathcal{F}_0),\mu^*)\).
\end{prop}
\begin{proof} Let \((\Omega,\mathcal{T},\mu^*)\) denote the completion of
\((\Omega,\sigma(\mathcal{F}_0),\mu^*)\). We shall show that \(\mathcal{H}\) extends
\(\mathcal{T}\) by showing it is complete (see \Cref{remark:minimality of
completion}), since it clearly extends \(\sigma(\mathcal{F}_0)\). Let \(A\in\mathcal{H}\) be a
null set, and let \(B\subseteq A\). By monotonicity, \(\mu^*(B)=0\), and
therefore \(\mu^*(B)+\mu^*(B^c)=\mu^*(B^c)\leq 1\). Therefore, \(B\in\mathcal{H}\).
	
	Reciprocally, let \(A\in\mathcal{H}\). Following the definition of \(\mu^*\), for
every \(n\in\mathbb{Z}^+\) there exists some \(G_n\in\mathcal{C}\) such that \(A\subseteq G_n\),
\(\mu^*(G_n)-1/n\leq\mu^*(A)\). Similarly, there exists some \(G_n'\in\mathcal{C}\) such
that \(A^c\subseteq G_n'\) and \(\mu^*(G_n')-1/n\leq\mu^*(A^c)\). Let
\(H_n=G_n'^c\). Then, since \(A\in\mathcal{H}\) and \(\mathcal{C}\subseteq\mathcal{H}\),
\(1-\mu^*(H_n)-1/n\leq1-\mu^*(A)\). Thus,
	\[\mu^*(G_n)-1/n\leq\mu^*(A)\leq\mu^*(H_n)+1/n.\] Let
\(H=\bigcup_nH_n\in\sigma(\mathcal{F}_0)\) and \(G=\bigcap_nG_n\in\sigma(\mathcal{F}_0)\). It is
clear that \(H\subseteq A\subseteq G\). Thus, we can write
\(A=H\cup (H\setminus A)\), with
\(H\setminus A\subseteq G\setminus H\in\sigma(\mathcal{F}_0)\), and since
\(G\setminus H\subseteq G_n\setminus H_n\),
	\[\mu^*(G\setminus H)\leq\mu^*(G_n\setminus H_n)=\mu^*(G_n)-\mu^*(H_n)\leq 2/n.\]
It follows that \(\mu^*(G\setminus H)=0\), and thus \(A\in\mathcal{T}\).
\end{proof}
We need one last tool to be able to prove our extension theorem:
\begin{thrm}[Monotone Class Theorem]\label{theorem:Monotone Class}
Let \(\mathcal{F}_0\) be a field of subsets of \(\Omega\). Then, \(\mathcal{C}(\mathcal{F}_0)=\sigma(\mathcal{F}_0)\). In particular,
if \(\mathcal{C}\) is a monotone class of subsets of \(\Omega\) and \(\mathcal{F}_0\subseteq\mathcal{C}\),
then \(\sigma(\mathcal{F}_0)\subseteq\mathcal{C}\).
\end{thrm}
\begin{proof} Let \(\mathcal{M}=\mathcal{C}(\mathcal{F}_0)\) and \(\mathcal{F}=\sigma(\mathcal{F}_0)\). Firstly, note that since \(\mathcal{F}\) is a
monotone class, then \(\mathcal{M}\subseteq\mathcal{F}\). Let \(A\in\mathcal{F}\), and define
\(\mathcal{M}_A=\{B\in\mathcal{M}\colon A\cap B, A\cap B^c \textit{ and } A^c\cap B\in\mathcal{M}\}\).
Then, \(\mathcal{M}_A\) is a monotone class itself. The rest of the proof is structured as follows:
	\begin{enumerate}
		\item For all \(A\in\mathcal{F}_{0}\), \(\mathcal{M}_{A}=\mathcal{M}\): we have
\(\mathcal{F}_{0}\subseteq \mathcal{M}_{A}\), and by minimality of \(\mathcal{M}\), \(\mathcal{M}\subseteq\mathcal{M}_{A}\).
By definition, \(\mathcal{M}_{A}\subseteq\mathcal{M}\). Hence, \(\mathcal{M}_A=\mathcal{M}\)
\item If \(B\) is a set in \(\mathcal{M}\) (not necessarily in \(\mathcal{F}_0\)), then \(\mathcal{M}_{B}=\mathcal{M}\): if \(A\in\mathcal{F}_0\), since \(\mathcal{M}_A=\mathcal{M}\), we have \(A\cap B, A\cap B^c \text{ and }A^c\cap B\in\mathcal{M}\). It follows that \(\mathcal{F}_0\subseteq\mathcal{M}_B\), and this implies that \(\mathcal{M}=\mathcal{C}(\mathcal{F}_0)\subseteq\mathcal{M}_B\) (see \cref{remark:generated structures}). \(\mathcal{M}_B\subseteq\mathcal{M}\) by definition of \(\mathcal{M}_B\).
		\item \(\mathcal{M}\) is a field because \(\emptyset\in\mathcal{F}_0\subseteq\mathcal{M}\) and it is closed under finite union and complementation because \(\mathcal{M}=\mathcal{M}_B\) for every \(B\in\mathcal{M}\). Since \(\mathcal{M}\) is both a field and a monotone class, it is a \(\sigma\)-field.
	\end{enumerate}
	Since \(\mathcal{M}\) is a \(\sigma\)-field containing \(\mathcal{F}_0\), we have \(\mathcal{F}=\sigma(\mathcal{F}_0)\subseteq\mathcal{M}\) (again, see \cref{remark:generated structures}).
\end{proof}
We are now ready to prove the main theorem of this section.

\begin{thrm}[Carathéodory Extension Theorem]\label{theorem:Caratheodory
Extension} Let \(\mathcal{F}_0\) be a field of subsets of \(\Omega\) and \(\mu\) a
nonnegative, countably additive set function over \(\mathcal{F}_0\). Assume that \(\mu\)
is \(\bm{\sigma}\)\textbf{-finite} over \(\mathcal{F}_0\), that is, that \(\Omega\) can
be written as \(\bigcup_n A_n\), where \(A_n\in\mathcal{F}_0\) and \(\mu(A_n)<+\infty\).
Then, \(\mu\) has a unique extension to the minimal \(\sigma\)-field over
\(\mathcal{F}_0\).
\end{thrm}
\begin{proof} Let \(\mathcal{F}=\sigma(\mathcal{F}_0)\). Suppose, without loss of generality, that
the sets \(A_n\) are disjoint (otherwise, take
\(A_n'=A_n\setminus\bigcup_{k=1}^{n-1}A_k\); then
\(\mu(A_n')\leq\mu(A_n)<+\infty\)). Define \(\mu_n\) in \(\mathcal{F}_0\) as
\(\mu_n(B)=\mu(A_n\cap B)\).  Then, \(\mu_n\) is a finite, nonnegative,
countably additive set function in \(\mathcal{F}_0\) and can therefore be extended to a
measure on \(\mathcal{F}\), which we will denote by  \(\mu_n^*\).  Now extend \(\mu\) to
\(\mathcal{F}\) by setting \(\mu^*=\sum_n\mu_n^*\). Then, \(\mu^*\) is a measure because
the order of summation of any double  series of positive terms can be switched
(see \cref{corollary:switching double series}).
	
	To see that \(\mu^*\) is unique, suppose that \(\lambda\) is also a measure
that extends \(\mu\) to \(\mathcal{F}\). Define  \(\lambda_n(B)=\lambda(B\cap A_n)\). Let
\(\mathcal{C}_n\) be the class of subsets of \(\Omega\) where \(\mu^*_n\) and
\(\lambda_n\) coincide.
	
	Clearly, \(\mathcal{F}_0\subseteq\mathcal{C}_n\). Moreover, \(\mathcal{C}_n\) is a monotone class: if
\(A_m\uparrow A\) or \(A_m\downarrow A\), since  both \(\mu^*_n\) and
\(\lambda_n\) are finite, then
\(\mu^*_n(A)=\lim_m\mu_{n}^*(A_m)=\lim_m\lambda_n(A_m)=\lambda_{n}(A)\), whence
\(A\in\mathcal{C}_n\). By the \hyperref[theorem:Monotone Class]{Monotone Class Theorem},
\(\lambda_n\equiv\mu^*_n\). Therefore,
	\[
			\lambda=\sum_n\lambda_n=\sum_n\mu^*_n=\mu^*
	,\]
finishing the proof.
\end{proof} 
\section{Integration}

We are now in position of defining the
flagship of measure theory: the Lebesgue integral\footnote{This is actually the
historical motivation behind measure theory: see \cite{lebesgue1901}, the
original paper by Henri Lebesgue introducing his theory of integration. Measure
theory was developed afterwards, to formalise, generalise and justify this new type of
integration.}. However, in order to be able to integrate functions, we first
need to establish which functions are to be integrated. The concept which will be used reminds that of continuity\footnote{This will be important later on, when we study the relation between topology and measure theory.}.

\begin{defn}
Let \(\left(\Omega_1,\mathcal{F}_1\right)\) and
\(\left(\Omega_2,\mathcal{F}_2\right)\) be measurable spaces. A function
\(f\colon\Omega_1\to\Omega_2\) is said to be \(\mathcal{F}_1,\mathcal{F}_2\)-\textbf{measurable} whenever
\(f^{-1}(B)\in\mathcal{F}_1\) for all \(B\in\mathcal{F}_2\). Sometimes, one or both \(\sigma\)-fields can be omitted (and, for instance, simply say that \(f\) is \emph{measurable}) if it is clear on which \(\sigma\)-fields we are working at.
	
When \(\Omega_2\) is a topological space, \(f\) is said to be Borel measurable (on \(\left(\Omega_1,\mathcal{F}_1\right)\)) if \(\mathcal{F}_2=\mathscr{B}\left(\Omega_2\right)\) and \(f\) is \(\mathcal{F}_1,\mathcal{F}_2\)-measurable.  %Similarly, \(f\) is said to be Lebesgue measurable if \(\mathcal{F}_2\) is the class of Lebesgue measurable sets.
\end{defn}
Borel measurable functions taking (extended) real values will be our candidate functions to
be integrated. Henceforth, if a function with codomain \(\mathbb{R}\) or \(\overline{\mathbb{R}}\) is said to be measurable, it will be understood that it is Borel measurable. Before defining the integral, we need to enunciate a series
of simple lemmas needed to prove deeper results:

\begin{lemm}\label{lemma:composition of measurable functions} The composition
of measurable functions is measurable.
\end{lemm}
\begin{proof} Let \(\left(\Omega_1,\mathcal{F}_1\right)\), \(\left(\Omega_2,\mathcal{F}_2\right)\) and
\(\left(\Omega_3,\mathcal{F}_3\right)\) be measure spaces and \(f\colon\Omega_1\to\Omega_2\),
\(g\colon\Omega_2\to\Omega_3\) measurable functions. Then, for every
\(M\in \Omega_3\),
\[
	\left(f\circ g\right)^{-1}(M)=g^{-1}\left(f^{-1}(M)\right).
\]
This set is an element of \(\mathcal{F}_1\), since \(f^{-1}(M)\in \mathcal{F}_2\) because \(f\) is measurable
and \(g\) is measurable.
\end{proof}

Consider the class of all measurable spaces. This class forms a category whose
morphisms are measurable functions. This lemma is stating the composability of these morphisms.
Identity morphisms are defined as identity functions and
associativity follows from associativity of function composition. Said category
is sometimes denoted as \textbf{Meas}.

\begin{lemm}\label{lemma:measurability by generator}
Let \(\left(\Omega_1,\mathcal{F}_1\right)\) and \(\left(\Omega_2,\mathcal{F}_2\right)\) be measure
spaces such that \(\mathcal{F}_2=\sigma(\mathcal{S})\) for some class \(\mathcal{S}\) of
subsets of ~\(\Omega_{2}\). Let \(h:\Omega_1\to\Omega_2\). Then, \(h\) is
measurable if, and only if, \(h^{-1}(C)\in\mathcal{F}_1\) for all \(C\in\mathcal{S}\).
\end{lemm}
\begin{proof} One implication is trivial. To see the other, define
	\[\mathcal{M}=\left\{A\in\mathcal{F}_2\colon h^{-1}(A)\in\mathcal{F}_1\right\}.\] By
hypothesis, \(\mathcal{S}\subseteq\mathcal{M}\). Since \(\mathcal{F}_1\) is a
\(\sigma\)-field and \(h^{-1}\) preserves arbitrary unions, intersections and
complements, \(\mathcal{M}\) is a \(\sigma\)-field. Thus,
\(\mathcal{F}_2=\sigma(\mathcal{C})\subseteq\mathcal{M}\), and therefore \(h\) is measurable.
\end{proof}
This result will show to be very useful, particularly in combination
with \cref{proposition:Borel sets on RB} if one wants to show that a
given function \(h\colon\Omega\to\overline{\mathbb{R}}\) is Borel measurable, it
suffices to show, for instance, that \(\left\{h\geq c\right\}\) (or \(\left\{h<c\right\}\)) is
measurable for every \(c\in\overline{\mathbb{R}}\).

\begin{lemm}\label{lemma:min and max are Borel measurable} Let \(\Omega,\mathcal{F}\) be
a measurable space and \(f,g\) Borel measurable functions. Then,
\(\max(f,g)\) defined by \(\max(f,g)(\omega)=\max(f(\omega),g(\omega))\)
is Borel measurable. Similarly, \(\min(f,g)\) is Borel measurable.
\end{lemm}
\begin{proof} By \cref{lemma:measurability by generator}, it suffices to see that
	\[\left\{\omega\colon \max(f,g)(\omega)\leq c\right\}=\left\{\omega\colon f(\omega)\leq c\right\}\cap\left\{\omega\colon g(\omega)\leq c\right\}\]
	
	is measurable for all \(c\in\overline{\mathbb{R}}\). Similarly,
\[
\left\{\omega\colon \min(f,g)(\omega)\geq c\right\}=\left\{\omega\colon f(\omega)\geq c\right\}\cap\left\{\omega\colon g(\omega)\geq c\right\}
\]
is so for all
\(c\in\overline{\mathbb{R}}\).
\end{proof}

Two last results that are very easy to prove are that for any set
\(A\subseteq\Omega\), \(A\) is measurable if, and only if, its indicator
function is Borel measurable; and that every constant function between measure
spaces is measurable. Following this last result and \Cref{lemma:min and max are
Borel measurable}, if \(h\) is a Borel measurable function, so are \(h^+\) and
\(h^-\).

\begin{prop}\label{proposition:limit of Borel is Borel} The pointwise
limit of Borel measurable functions is Borel measurable.
\end{prop}
\begin{proof} Let \(\left(\Omega,\mathcal{F}\right)\) be a measurable space and
\(\left\{h_n\right\}_{n\in\mathbb{Z}^+}\) be a sequence of Borel measurable
functions \(h_n\colon\Omega\to\overline{\mathbb{R}}\) converging pointwise to a limit \(h\).  By
\Cref{lemma:measurability by generator}, it suffices to show that
\(\left\{h>c\right\}=\left\{\omega\in\Omega\left|h(\omega)>c\right.\right\}\) is measurable
for all \(c\). Now, a simple analytical argument shows that if \(a_n\) is a converging sequence of extended real numbers, then for any \(c\in\overline{\mathbb{R}}\), \(\lim_na_n>c\) if, and only if, there exist some \(r,n_0\in\mathbb{Z}^+\) such that \(a_n>c+\frac{1}{r}\) for any \(n\geq n_0\). Therefore,
	\[
			\left\{h>c\right\}=\left\{\lim_nh_n>c\right\}=\bigcup_{r\in\mathbb{Z}^+}\bigcup_{n_0\in\mathbb{Z}^+}\bigcap_{n\geq n_0}\left\{h_n>c+\frac{1}{r}\right\}
	\]
	Since \(\left\{h_n>c+\frac{1}{r}\right\}\) is measurable for every \(n\) and every \(c\), the proof is completed.
\end{proof}


A special kind of Borel measurable functions will be of interest. Concretely,
those whose range is a finite set (that is, they take finitely many values).
These functions are interesting because their integral can be defined intuitively,
and, as we will see, they are closed under arithmetic operations and are able to
``generate'' all other measurable functions via limits. This will allow us to
both define the integral of Borel measurable functions and, later on, find
conditions to exchange the integral and limit signs.

\begin{defn}[Simple functions] Let \(\left(\Omega,\mathcal{F}\right)\) be a
measurable space and \(h\colon\Omega\to\overline{\mathbb{R}}\) a Borel measurable function. Then,
\(h\) is said to be a \textbf{simple function} whenever it takes finitely many
values. Equivalently, \(h\) is simple whenever we can find finitely many
measurable sets \(A_i\subseteq\Omega\) and values \(x_i\in\overline{\mathbb{R}}\) such that
	\[h=\sum_{i=1}^{n}x_iI_{A_i},\] where \(I_{A_i}\) is the indicator function
of \(A_i\) and the sum is defined, in the sense that the expression
\(\infty-\infty\) never occurs.
\end{defn}
\begin{remk} If we impose the sets \(A_i\) to form a partition of the set
\(\Omega\), then every simple function \(h\) with range
\(h(\Omega)=\left\{x_1,\dots,x_n\right\}\) can be written uniquely as
	\[h=\sum_{i=1}^{n}x_iI_{h^{-1}(x_i)}.\]
	
	We will call this expression the \textbf{standard form} of
\(h\)\footnote{This notation is not common in the literature, but it will show to
be useful in our text.}.
\end{remk}
As we will see, Borel measurable functions are closed under arithmetic
operations. However, in order to prove it we need the following lemma.  %It may be noteworthy that a particular case of this result is mentioned but not proved in \cite{ash1972real}.

\begin{lemm} Any pointwise operation of finitely many simple functions, if
defined, is simple. More precisely, if \(op\) is an arbitrary mapping
\(op\colon D\to\overline{\mathbb{R}}\), where \(D\subseteq\overline{\mathbb{R}}^n\), and \(s_1,\dots,s_n\) are
simple functions such that the function
\(h(\omega)=op(s_1(\omega),\dots,s_n(\omega))\) is well-defined, then \(h\) is
simple.\footnote{Not much attention is given to this in \cite{ash1972real}. Both the statement and the proof of this result are original.}
\end{lemm}
\begin{proof} Write, for any \(k\leq n\), \(s_k\) in standard form as
\(s_k=\sum_{m=1}^{N_k}x_{km}I_{A_{km}}\) (that is, the sets \(A_{km}\) form a
partition of \(\Omega\)). Set
\(\mathcal{I}=\left\{1,\dots,N_1\right\}\times\dots\times\left\{1,\dots,N_n\right\}\),
and for every \((m_1,\dots,m_n)\in\mathcal{I}\), define
\(C_{(m_1,\dots,m_n)}=\bigcap_{k=1}^{n}A_{km_k}\), and
\(\mathcal{I}'=\left\{\phi\in \mathcal{I}\colon C_{\phi}\neq\emptyset\right\}\).  Note
that, by definition, the family of sets \(C_{\phi}\) such that
\(\phi\in\mathcal{I}'\) forms a partition of \(\Omega\).
	
	Now, for any \(\phi=(m_1,\dots,m_n)\in\mathcal{I}'\), define
\(x_{\phi}=op(x_{1m_1},\dots,x_{nm_n})\). Such value \(x_\phi\) exists because
\(h\) is defined and for any value \(\omega\in C_\phi\) (at least one exists
because \(\phi\in\mathcal{I}'\), hence \(C_\phi\) is nonempty), then
\(h(\omega)=op(s_1(\omega),\dots,s_n(\omega))=op(x_{1m_1},\dots,x_{nm_n})\).
	
	Using the notation established earlier, we can write
	\[h=\sum_{\phi\in\mathcal{I}'}x_\phi I_{C_\phi}.\] 
	
	which is clearly a simple function, because
\(\mathcal{I}'\subseteq\mathcal{I}\) is finite.
\end{proof}
\begin{corl}\label{corollary:operation of simple functions is simple}
The sum, product and quotient of simple functions, if defined, is a simple function.
\end{corl}
Now we will introduce a theorem that is central for the construction of the
Lebesgue integral. It allows us to approximate every  measurable function via a
sequence of simple functions satisfying useful properties.

\begin{thrm} \label{theorem:approximation of measurable by simple}
	\begin{enumerate}
		\item  Every nonnegative measurable function is the pointwise limit of
an increasing sequence of nonnegative, real-valued simple functions. Moreover,
if the function is bounded, convergence is uniform.
		\item Every measurable function \(h\) is the pointwise limit of a
sequence of finite-valued simple functions \(s_n\) which satisfy
\(|s_n|\leq|h|\) for all \(n\in\mathbb{Z}^+\).
	\end{enumerate}
\end{thrm}
\begin{proof}
	\begin{enumerate}
		\item We want to approximate, pointwise, the function \(h\) by simple functions \(s_n\). We have almost no information about the domain \(\Omega\) except from the fact that \(h\) is measurable, so we need to work with the codomain \(\mathbb{R}\). The idea is to group the \emph{values} the function \(h\) takes into finitely many intervals for every fixed value of \(n\), and then recover the subsets of \(\Omega\) where \(h\) takes these values.

For that purpose, equally split the interval \([0,n)\) into \(N(n)\) many consecutive intervals 
\(V_k^n=\left[\frac{k-1}{N(n)},\frac {k}{N(n)}\right)\), \(k=1,\dots,nN(n)\). In
this interval, approximate \(h\) by its left endpoint \(\frac{k-1}{N(n)}\). That
is, set
		\[s_n(\omega)=\frac{k-1}{N(n)} \text{ ~~whenever } h(\omega)\in V_k^n,\]
and \(s_n(\omega)=n\) whenever \(h(\omega)\geq n\). This way, \(s_n\) is
nonnegative, finite-valued and \(|h-s_n|\leq \frac{1}{N(n)}\) if \(h(\omega)\)
is in the interval \([0,n]\).
		
		With this construction, each \(s_n\) is simple: we can write
		\[s_n=nI_{\left\{h\geq n\right\}}+\sum_{k=1}^{nN(n)}\frac{k-1}{N(n)}I_{\left\{h\in V^n_k\right\}},\]
which is clearly a simple function.
		
		If we want \(s_n\) to converge to \(h\), it is sufficient that \(N(n)\)
tend to \(+\infty\) when \(n\to+\infty\). This also ensures that convergence is
uniform when \(h\) is bounded. The main remaining desired condition is that the
sequence of functions is increasing. If we impose that it is, we obtain a series
of inequalities depending on where \(h(\omega)\) falls regarding intervals
\(V_k^n\) and \(V_{k'}^{n+1}\):
		
		\begin{itemize}
			\item If \(h(\omega)\geq n+1\), then
\(s_n(\omega)=n\leq n+1=s_{n+1}(\omega)\). This inequality holds for any
\(N(n)\).
			\item If \(n\leq h(\omega) < n+1\), then \(s_n(\omega)=n\) and there
exists some \(k'\) such that \(h(\omega)\in V^{n+1}_{k'}\). This implies that
\(n\leq h(\omega)<\frac{k'}{N(n+1)}\), from which one can deduce that
\(k'>nN(n+1)\), hence \(k'-1\geq nN(n+1)\). Thus,
			\[s_{n+1}(\omega)=\frac{k'-1}{N(n+1)}\geq\frac{nN(n+1)}{N(n+1)}=n=s_{n}(\omega).\]
This inequality holds for any value \(N(n)\) too.
			\item If \(0\leq h(\omega)<n\), then
\(h(\omega)\in V_k^n\cap V_{k'}^{n+1}\) for some \(k,k'\). This implies that
			\[\frac{k-1}{N(n)}\leq h(\omega)<\frac{k'}{N(n+1)},\] from which one
deduces the inequality \(k'>\frac{N(n+1)}{N(n)}\left(k-1\right)\). If we impose
\(\frac{N(n+1)}{N(n)}\) to be a positive integer, it follows that
\(k'-1\geq\frac{N(n+1)}{N(n)}\left(k-1\right).\) Therefore,
			\[s_{n+1}(\omega)=\frac{k'-1}{N(n+1)}\geq\frac{k-1}{N(n)}=s_n(\omega).\]
		\end{itemize}
		
		So far, the only conditions imposed to \(N(n)\) have been that
\(\frac{N(n+1)}{N(n)}\) is a positive integer and that \(\lim_nN(n)=+\infty\).
There are infinitely many ways to do this, but the easiest one is by setting
\(N(n)=2^n\).\footnote{The exposition of this proof is original. A much more \emph{straight-to-the-point} proof is found in \cite{ash1972real}.}
		\item Decompose \(h=h^+-h^-\). Approximate \(h^+\) and \(h^-\) by
increasing sequences of nonnegative, finite-valued, simple functions
\(s_n^+,s_n^-\). Then, setting \(s_n=s_n^+-s_n^-\) yields the desired sequence
of simple functions.
	\end{enumerate}
\end{proof}

This theorem, combined with
\cref{proposition:limit of Borel is Borel} gives us a nice characterization for
Borel measurable functions:  A function is Borel measurable if, and only if, it
is the pointwise limit of simple functions.  This will be a key result later on,
when we make a kind of reasoning very common in measure theory:  if we want to
prove some property of some set of measurable functions, we first restrict
ourselves  to indicator functions, then we extend the property to simple
functions, and finally
we extend it to measurable functions via limits. 

A first example of this kind of reasoning is the proof of following result, which we already showed for simple functions (this is \cref{corollary:operation of simple functions is simple}):

\begin{prop} The sum, product and division of measurable functions is
measurable, provided it is defined.
\end{prop}
\begin{proof} Let \(h_1\) and \(h_2\) be measurable functions, and approximate
them by simple functions \(s_n^1, s_n^2\) using \cref{theorem:approximation of
measurable by simple}. Then, wherever defined, results of arithmetic operations of measurable functions can be approximated in the following way:
	\begin{itemize}
		\item \(s_n^1+s_n^2\to h_1+h_2\)
		\item
\(s_n^1s_n^2I_{\left\{h_1\neq0\right\}}I_{\left\{h_2\neq0\right\}}\to h_1h_2\)
		\item
\(\displaystyle \frac{s_n^1}{s_n^2+(1/n)I_{\left\{s_n^2=0\right\}}}\to \frac{h_1}{h_2}\)
	\end{itemize}
\end{proof}
\begin{remk} A consequence of this theorem is that any extended real-valued
function is Borel measurable if, and only if, its negative and positive parts
are.
\end{remk}
We finally have developed all the machinery required to define the Lebesgue integral
and show some of its properties. Let \(\left(\Omega,\mathcal{F},\mu\right)\) be a measure
space that will be fixed throughout the discussion.

\begin{defn} Let \(s\) be a simple function with domain \(\Omega\). Write
	\[s=\sum_{i}^{}x_iI_{A_i}.\]
	
	We define the \textbf{Lebesgue integral} of \(s\) with respect to \(\mu\), and denote it by
\(\int_{\Omega}s ~d\mu\), \(\int_{\Omega}s(\omega)~d\mu\) or
\(\int_{\Omega}s(\omega)~\mu(d\omega)\), as
	\[\int_{\Omega}s~d\mu=\sum_{i}^{}x_i\mu(A_i).\]
\end{defn}
It is easy to check that this definition is well-posed in the sense that if
\(s\) admits a different expression in terms of sums and products of indicator
functions, then the sum on the right coincides.

In analysis, it is often very useful to be able to exchange symbols regarding limits and integrals or derivatives. In this case, we would
like to be able to exchange signs while under the conditions of
\Cref{theorem:approximation of measurable by simple}, that is, if
\(s_n\uparrow h\), where \(s_n\) are nonnegative, finite-valued, simple
functions, then \(\int_\Omega s_n~d\mu.\uparrow\int_\Omega h~d\mu\). This could
be a way to define the integral for nonnegative Borel functions, but we would
need to show that the limit does not depend on the sequence chosen. We can work
around this by using suprema:

\begin{defn} Let \(h\) be a real-valued, Borel measurable function with
domain \(\Omega\). If \(h\) is nonnegative, define
	\[\int_\Omega h~d\mu=\sup\left\{\int_\Omega s~d\mu\colon 0\leq s\leq h, s \text{
simple}\right\}\]
	
	For the general case, split \(h=h^+-h^-\) and set
	\[\int_{\Omega}h~d\mu=\int_{\Omega}h^+~d\mu-\int_{\Omega}h^-~d\mu \text{
whenever the expression is not of the form }+\infty-\infty.\] If it is of the
form \(+\infty-\infty\), we say the integral is not defined. Moreover, if
\(\int_{\Omega}h~d\mu\) is finite we say that \(h\) is \textbf{integrable}.
	
	Finally, if \(B\in\mathcal{F}\), set \(\int_{B}h~d\mu=\int_{\Omega}hI_B~d\mu\)
\end{defn}
Now that we have defined the integral for the general class of measurable
functions, we need to show that it satisfies all the good properties an integral should satisfy.

\begin{prop} Let \(g,h\) be extended real-valued, Borel measurable
functions. Then,
	\begin{enumerate}
		\item\label{proposition:integral multiplicativity}If
\(\int_{\Omega}h~d\mu\) exists, so does \(\int_{\Omega}ch~d\mu\) and
\(\int_{\Omega}ch~d\mu=c\int_{\Omega}h~d\mu\) for every \(c\in\overline{\mathbb{R}}\).
		\item\label{proposition:integral monotonicity} The integral sign is
monotonous. That is, if \(g\leq h\), then
\[
		\int_{\Omega}g~d\mu\leq \int_{\Omega}h~d\mu
\] in the sense that if
\(\int_{\Omega}g~d\mu\) exists and is greater that \(-\infty\), then
\(\int_{\Omega}h~d\mu\) exists; if \(\int_{\Omega} h~d\mu\) exists and is lesser
than \(+\infty\), then \(\int_{\Omega}g~d\mu\) exists; and whenever both
integrals exist the inequality holds.
		\item If \(\int_{\Omega}h~d\mu\) exists, then
\(\left|\int_{\Omega}h~d\mu\right|\leq\int_{\Omega}|h|~d\mu.\)
		\item \label{proposition:integral in subspace coincides}If \(h\) is
nonnegative and \(B\in\mathcal{F}\), then
\(\int_B h~d\mu=\sup\left\{\int_{B}s~d\mu\colon0\leq s\leq h, s \text{
simple}\right\}\)
		\item If \(\int_{\Omega}h~d\mu\) exists, then so does \(\int_{B}h~d\mu\)
for each \(B\in\mathcal{F}\). If \(\int_{\Omega}h~d\mu\) is finite, so is
\(\int_{B}h~d\mu\) for each \(B\in\mathcal{F}\).
	\end{enumerate}
\end{prop}
\begin{proof}
	\begin{enumerate}
		\item The result clearly holds when \(h\) is simple. If \(c=0\) it is
also clearly true. If \(h\) is nonnegative and \(c>0\),
		\[
		\begin{array}{>{\displaystyle}r>{\displaystyle}l} \left\{\int_{\Omega}s~d\mu\colon 0\leq s\leq ch, s \text{
simple}\right\}&=c\left\{\int_{\Omega}s/c~d\mu\colon 0\leq s\leq ch, s \text{
simple}\right\}\\ &=c\left\{\int_{\Omega}s/c~d\mu\colon 0\leq s/c\leq h, s/c \text{
simple}\right\}\\ &=c\left\{\int_{\Omega}s~d\mu\colon 0\leq s\leq h, s \text{
simple}\right\}
		\end{array},
		\] where in the second-to-last step we used the fact that \(s\) is
simple if, and only if, \(s/c\) is simple. Taking suprema,
\(\int_{\Omega}ch~d\mu=c\int_{\Omega}h~d\mu.\)
		
		Now, if \(h\) is arbitrary, then for \(c>0,\) it holds that
\((ch)^+=ch^+\) and \((ch)^-=ch^-\). Hence, by what we just proved,
		\[\int_{\Omega}ch~d\mu=\int_{\Omega}ch^-~d\mu+\int_{\Omega}ch^-~d\mu=c\int_{\Omega}h^+~d\mu+c\int_{\Omega}h^-~d\mu=c\int_{\Omega}h~d\mu.\]
		
		If \(c<0\), then \((ch)^+=-ch^-\) and \((ch)^-=-ch^+\). Thus, 
		\[\int_{\Omega}ch~d\mu=\int_{\Omega}-ch^-~d\mu-\int_{\Omega}-ch^+~d\mu=-c\int_{\Omega}h^-~d\mu+c\int_{\Omega}h^+~d\mu=c\int_{\Omega}h~d\mu.\]
		\item If \(g\) is nonnegative, then \(h\) is nonnegative too. The result
follows immediately from the definition of the integral:
		\[\left\{\int_{\Omega}s~d\mu\colon 0\leq s\leq g\right\}\subseteq\left\{\int_{\Omega}s~d\mu\colon 0\leq s\leq h\right\}.\]
Thus, \(\int_{\Omega}g~d\mu\leq\int_{\Omega}h~d\mu.\) For the general case, note
that \(g\leq h\) if, and only if, \(g^+\leq h^+\) and \(h^-\leq g^-\).
Therefore,
		\[\int_{\Omega}g^+~d\mu\leq\int_{\Omega}h^+~d\mu\text{ and
}\int_{\Omega}h^-~d\mu\leq\int_{\Omega}g^-~d\mu.\] From this, if
\(\int_{\Omega}g~d\mu>-\infty\), then
\(\int_{\Omega}h^-~d\mu\leq \int_{\Omega}g^-<\infty\), and so
\(\int_{\Omega}h~d\mu\) exists. The case where \(\int_{\Omega}h~d\mu<+\infty\)
is similar. If both integrals exist,
		\[\int_{\Omega}g~d\mu=\int_{\Omega}g^+~d\mu-\int_{\Omega}g^-~d\mu\leq\int_{\Omega}h^+~d\mu-\int_{\Omega}h^-~d\mu=\int_{\Omega}h~d\mu.\]
		\item This follows from the previous item and the fact that
\(-|h|\leq h\leq|h|\).
		\item Note that \(0\leq s\leq hI_B\) implies that \(s=sI_{B}\), and
\(\int_{\Omega}s~d\mu=\int_{B}s~d\mu\). Therefore,
		\[
		\begin{array}{>{\displaystyle}r>{\displaystyle}l} \left\{\int_{\Omega}s~d\mu\colon 0\leq s\leq hI_B , s \text{
simple}\right\}=&\left\{\int_{B}s~d\mu\colon 0\leq sI_B\leq hI_B , s \text{
simple}\right\}\\ =&\left\{\int_Bs~d\mu\colon 0\leq s\leq h , s \text{
simple}\right\}
		\end{array} .\] Hence, the result follows taking suprema.
		\item This follows from \ref{proposition:integral monotonicity} and the
facts that \((hI_B)^+=h^+I_B\) and \((hI_B)^-=h^-I_B\).
	\end{enumerate}
\end{proof}
\begin{remk}\label{remark:subspace of a measure space has the same integral}
\cref{proposition:integral in subspace coincides} is interesting because
it tells us that if we take some nonempty subset \(A\in\mathcal{F}\) and regard it as a
subspace of \(\Omega\) (that is, consider the measure space
\(\left(A,\mathcal{F}_{A},\mu|_{\mathcal{F}_{A}}\right)\), with \(\mathcal{F}_{A}\) interpreted as
\(\mathcal{F}_{A}=\left\{B\cap A\colon B\in\mathcal{F}\right\}\)), then the integral in this
measure space coincides with the previously defined integral
\(\int_{A}h~d\mu=\int_{\Omega}hI_A~d\mu\).
\end{remk}

Now we have all the tools needed to prove a series of very powerful theorems
concerning Lebesgue integration.
\begin{lemm}\label{lemma:integral of simple function defines a measure} Let
\(s\) be a nonnegative simple function defined on a measure space
\(\left(\Omega,\mathcal{F},\mu\right)\). Then, the function
\(\lambda(B)=\int_{B}s ~d\mu\) is a measure on \(\mathcal{F}\).
\end{lemm}
\begin{proof} If \(s\) is an indicator \(I_{A}\), then
\(\int_{B}s~d\mu=\mu(A\cap B)\), and in this case \(\lambda\) is clearly a
measure because \(\mu\) is.  If we write \(s\) in its standard form
\(s=\sum_{i}^{n}x_{i}I_{A_{i}}\), then
	\[ \int_{B}s~d\mu=\sum_{i}^{n}x_{i}\int_{B}I_{A_{i}}~d\mu,
	\] which clearly implies that \(\lambda\) is a measure in this case too.
\end{proof}
\begin{thrm}[Monotone Convergence Theorem]\label{theorem:Monotone
Convergence}
	
	Let \(h_1,h_2\dots\) be an increasing sequence of nonnegative Borel
measurable functions converging to a pointwise limit \(h\). Then,
\(\int_{\Omega}h_n~d\mu\uparrow\int_{\Omega}h~d\mu\).
\end{thrm}
\begin{proof}
	Since \(h_n\leq h\), by \cref{proposition:integral monotonicity}, the
sequence of integrals \(\int_{\Omega}h_n~d\mu\) is increasing and bounded above
by \(\int_{\Omega}h~d\mu\). Therefore, the limit \(\lim_n\int_{\Omega}h_n~d\mu\)
exists and satisfies \(\lim_n\int_{\Omega}h_n~d\mu\leq\int_{\Omega}h~d\mu.\) Let
\(k=\lim_n\int_{\Omega}h_n~d\mu\).
	
	Now let \(s\) be a simple function satisfying \(0\leq s\leq h\). Let
\(b\in(0,1)\), and define a sequence of sets
\(B_n=\left\{h_n\geq bs\right\}\)\footnote{This set is measurable: write \(s\) in standard form and express \(B_n\) as an intersection of measurable sets.}. It is clear that
\(B_n\uparrow\Omega\) and that \(\int_{B_n}h_n~d\mu\geq b\int_{B_n}s~d\mu\).
Therefore, by \cref{proposition:integral monotonicity},
	\[\int_{\Omega}h_n~d\mu\geq\int_{B_n}h_n~d\mu\geq b\int_{B_n}s~d\mu.\] Note
that, by \Cref{lemma:integral of simple function defines a measure}, the
function \(\lambda(B)=\int_{B}s~d\mu\) is a measure. By taking limits when
\(n\to+\infty\) and following \cref{proposition:limit of increasing sets}, we
have \(k\geq b\int_{\Omega}s~d\mu\). By taking limits when \(b\to1\), we have
\(k\geq\int_{\Omega}s~d\mu\). By taking suprema for \(s\),
\(k\geq\int_{\Omega}h~d\mu\).
\end{proof}
\begin{thrm}\label{theorem:integral defines a measure} Let \(h\) be an
extended real-valued, Borel measurable function such that
\(\int_{\Omega}h~d\mu\) exists. Then, the function \(\lambda(B)=\int_{B}h~d\mu\)
is countably additive. In particular, if \(h\geq0\), then \(\lambda\) is a
measure.
\end{thrm}
\begin{proof} First suppose \(h\) nonnegative. Use \Cref{theorem:approximation
of measurable by simple} to obtain a sequence of simple functions \(s_{m}\) with
\(s_{m}\uparrow h\). Let \(B_{1},B_{2},\dotsc\) be a sequence of disjoint
measurable sets, and let \(B=\bigcup_{n}B_{n}\). Note that
\(s_{m}I_{B}\uparrow hI_{B}\). By the \hyperref[theorem:Monotone
Convergence]{Monotone Convergence Theorem},
\(\int_{B}s_{m}~d\mu\uparrow_{m}\int_{B}h~d\mu\). Since, for every \(m\), the
function defined by \(\lambda_{m}(A)=\int_{A}s_{m}~d\mu\) is a measure by
\Cref{lemma:integral of simple function defines a measure}, we have
\(\int_{B}s_{m}~d\mu=\sum_{n}\int_{B_{n}}s_{m}~d\mu\).  Define the sequence
\(a_{nm}=\sum_{k=1}^{n}\int_{B_{k}}s_{m}~d\mu\). Then, by what we have just
seen,
	\[ \lambda(B)=\lim_{m}\lim_{n}a_{nm}
	\] Since \(a_{nm}\) is increasing with respect to both indices, we can apply
\Cref{lemma:switching limits of increasing requences of real numbers}, and thus
	\[ \lambda(B)=\lim_{n}\lim_{m}a_{nm}=\lim_{n}\sum_{k=1}^{n}\lim_{m}\int_{B_{k}}s_{m}~d\mu
	\] However, for every \(n\), \(s_{m}I_{B_{n}}\uparrow_mhI_{B_{n}}\) which
implies that \(\lim_{m}\int_{B_{n}}s_{m}~d\mu=\int_{B_{n}}h~d\mu\). Therefore,
\(\lambda(B)=\sum_{n}\int_{B_{n}}h~d\mu=\sum_{n}\lambda(B_{n})\), as we wanted
to see.
	
	For the general case, split \(h=h^+-h^-\). Apply the result proved so far to
\(h^+\) and \(h^-\) to obtain two measures, \(\lambda^+\) and \(\lambda^-\).
Since \(\int_{\Omega}h~d\mu\) exists, at least one of \(\int_{\Omega}h^+~d\mu\)
and \(\int_{\Omega}h^-~d\mu\) is finite, and therefore at least one of
\(\lambda^+\) and \(\lambda^-\) is finite, which ensures that \(\lambda\) is
well defined and \(\sigma\)-additive.
\end{proof}
With this last result,\footnote{In \cite{ash1972real}, \Cref{theorem:integral defines a measure} is proved without employing the previous two results used here. However, the proof is, admittedly, rather technical and difficult. The approach followed here is original and (hopefully) a more comprehensible one.} we can finally prove that the integral is additive:
\begin{prop} Let \(f\), \(g\) be Borel measurable functions, and assume
that \(f+g\) is well-defined. If \(\int_{\Omega}f~d\mu\) and
\(\int_{\Omega}g~d\mu\) exist and their sum is well-defined, then the integral
\(\int_{\Omega}f+g~d\mu\) exists and
	\[\int_{\Omega}f+g~d\mu=\int_{\Omega}f~d\mu+\int_{\Omega}g~d\mu.\]
\end{prop}
\begin{proof} If \(f\) and \(g\) are nonnegative simple functions, the result
follows easily from the definition of the integral. If \(f\) and \(g\) are
nonnegative, we can approximate them by increasing sequences of nonnegative
simple functions via \Cref{theorem:approximation of measurable by simple}:
\(s_n^1\uparrow f, s_{n}^{2}\uparrow g\). Thus, by the \hyperref[theorem:Monotone Convergence]{Monotone Convergence Theorem},
	\[\int_{\Omega}s_n^1+s_n^2~d\mu=\int_{\Omega}s_n^1~d\mu+\int_{\Omega}s_n^2~d\mu\uparrow\int_{\Omega}f+g~d\mu=\int_{\Omega}f~d\mu+\int_{\Omega}g~d\mu.\]
	
	The proof of the general case is a casewise proof (depending on the signs of
\(f, g\) and \(f+g\); and splitting \(\Omega\) into the subsets where each
combination of signs takes place) with little interesting ideas - except from
the one exposed - and is omitted.
\end{proof}
\begin{corl}
	\begin{enumerate}
		\item \label{corollary:exchange series and integral} If
\(h_1, h_2,\dots\) are nonnegative Borel measurable, then
		\[\sum_{n=1}^{+\infty}\int_{\Omega}h_n~d\mu=\int_{\Omega}\left(\sum_{n=1}^{+\infty}h_n\right)~d\mu\]
		\item \label{corollary:integrable iff absolute value is}If \(h\) is
Borel measurable, \(h\) is integrable if, and only if, \(|h|\) is.
		\item If \(g\) and \(h\) are Borel measurable, \(|g|\leq h\) and \(h\)
is integrable, then \(g\) is integrable.
	\end{enumerate}
\end{corl}
\begin{proof}
	\begin{enumerate}
		\item Direct from the additivity of the integral,
				the \hyperref[theorem:Monotone Convergence]{Monotone Convergence Theorem} and the
fact that \(\sum_{n=1}^{N}h_n\uparrow\sum_{n=1}^{+\infty}h_n\) when
\(N\to+\infty\).
		\item If \(|h|=h^++h^-\) is integrable, it follows from additivity that
so are \(h^+\) and \(h^-\). Thus, \(h\) is integrable. If \(h\) is integrable,
then so are by \(h^+\) and \(h^-\) by the definition of integral. Thus, \(h\) is
integrable by additivity.
		\item \(|g|\) is integrable because of monotonicity, and by
\ref{corollary:integrable iff absolute value is}, \(g\) is integrable.
	\end{enumerate}
\end{proof}
\begin{defn} A condition is said to hold \textit{almost everywhere} with
respect to a measure \(\mu\) (written \(\mu\)-a.e., a.e. [\(\mu\)] or simply
a.e., if there is no confusion respect the measure of integration) whenever
there exists some \(B\in\mathcal{F}\) where the property is satisfied and such that
\(\mu(B^c)=0\).
\end{defn}

An important result is that, from the integration point of view, two functions
that coincide almost everywhere are identical. This is captured in the following
proposition:

\begin{prop} Let \(f, g\) and \(h\) be Borel measurable functions.
	\begin{enumerate}
		\item \label{proposition:null a.e. has null integral}If \(f=0\)
\(\mu\)-a.e., then \(\int_{\Omega}f~d\mu=0\).
		\item \label{proposition:almost everywhere monotonicity of the
integral}If \(g\leq h\) \(\mu\)-a.e., then
\(\int_{\Omega}g~d\mu\leq \int_{\Omega}h~d\mu\) in the sense of
\cref{proposition:integral monotonicity}.
		\item \label{proposition:equality a.e. implies equality of integrals}If
\(g=h\) \(\mu\)-a.e., then \(\int_{\Omega}g~d\mu=\int_{\Omega}h~d\mu\), in the
sense that one exists if, and only if, the other one does, and they are equal.
	\end{enumerate}
	
\end{prop}
\begin{proof}
	\begin{enumerate}
		\item If \(s\) is simple and nonnegative, we can write \(s\) in standard form as
\[
		s=\sum_{i}{x_iI_{A_i}}
.\] If we define
\(N=\left\{s\neq 0\right\}\), then \(x_i\neq0\) implies
\(A_i\subseteq N\), whence \(\mu(A_i)\leq\mu(N)=0\).
		
		If \(f\geq0\) and \(s\) is a simple function such that \(0\leq s\leq f\), then
\(s=0\) a.e., and thus \(\int_{\Omega}s~d\mu=0\). It follows that
\(\int_{\Omega}f~d\mu=0\).
		
		For the general case, \(f=0\) a.e. implies \(|f|=0\) a.e. Since \(f^+\)
and \(f^-\) are both nonnegative and bounded above by \(|f|\), they are both
null a.e. Thus, \(\int_{\Omega}f^+~d\mu=\int_{\Omega}f^-~d\mu=0\).
\item Define \(B=\left\{g>h\right\}\)\footnote{It is not immediate that \(B\) is measurable. One way to see it is by writing \(B=\left\{h<+\infty\right\}\cap\left\{g>-\infty\right\}\cap\bigcap_{q\in \mathbb{Q}}\left(\left\{g>q\right\}\cap\left\{q>h\right\}\right)\).}. Let \(A=B^{c}\).
\(g=gI_{A}+gI_{B}\) and \(h=hI_{A}+hI_{B}\). Since \(gI_{B}\) and \(hI_{B}\), by
\ref{proposition:null a.e. has null integral} their integrals are \(0\). Then,
by \cref{proposition:integral monotonicity},
		\[ \int_{\Omega}g~d\mu=\int_{A}g~d\mu\leq\int_{A}h~d\mu=\int_{\Omega}h~d\mu
		\]
		
		
\item Define \(B=\left\{g\neq h\right\}=\left\{g>h\right\}\cup\left\{h>g\right\}\), and \(A=B^c\). Therefore,
		\[
				\int_{\Omega}g~d\mu=\int_{A}g~d\mu=\int_{A}h~d\mu=\int_{\Omega}h~d\mu
		.\]
	\end{enumerate}
\end{proof}
\begin{prop}
	\begin{enumerate}
		\item If \(h\) is integrable with respect to \(\mu\), then it is finite
\(\mu\)-a.e.
		\item If \(h\geq0\) and \(\int_{\Omega}h~d\mu=0\), then \(h=0\) a.e.
	\end{enumerate}
\end{prop}
\begin{proof}
		\begin{enumerate}
			\item Let \(B=\left\{h\in\mathbb{R}\right\}\) and \(A=B^c\). Then, if \(\mu(A)>0\), we would have
			\[\int_{\Omega}|h|~d\mu\geq\int_{A}|h|=\infty\cdot\mu(A)=\infty,\]
which is a contradiction.
			\item Define \(B=\left\{h>0\right\}\) and
\(B_n=\left\{h>\frac{1}{n}\right\}\). Note that
\(B_n\uparrow B\). Therefore, \(\mu(B_n)\uparrow\mu(B)\) and
\(\int_{B_n}h~d\mu\uparrow\int_{B}h~d\mu\) by \cref{theorem:integral defines a
measure} and \cref{proposition:limit of increasing sets}.	Note that,
since \(0\leq\int_{B_n}h~d\mu\leq\int_{B}h~d\mu=0\), we have
\(\int_{B_n}h~d\mu=0\). However, in \(B_n\), \(h\geq 1/n\), whence
\(\int_{B_n}h~d\mu\geq \mu(B_n)/n\). One deduces that \(\mu(B_n)=0\) for all
\(n\). Therefore, \(\mu(B)=0.\)
		\end{enumerate}
\end{proof}
As we said, differences between functions in null sets are irrelevant regarding integration. The \hyperref[theorem:Monotone Convergence]{Monotone Convergence Theorem} can be extended in order to account for this, and additionally, the nonnegativity hypothesis can be greatly relaxed.

\begin{thrm}[Extended Monotone Convergence Theorem]\label{theorem:Extended
Monotone Convergence} Let \(g_1,g_2\dots\) be a sequence of Borel measurable
functions.
	\begin{enumerate}
		\item Suppose that \(\int_{\Omega}g_1~d\mu>-\infty\) and
\(g_n\uparrow g\) a.e., that is, we have \(g=\lim_{n}g_{n}\) a.e. and, for every
\(n\), \(g_{n}\leq g_{n+1}\) a.e. Then, the integrals \(\int_{\Omega}g~d\mu\)
and \(\int_{\Omega}g_n~d\mu\) exist for all \(n\), and
		\[\int_{\Omega}g_n~d\mu\uparrow\int_{\Omega}g~d\mu.\]
		\item Suppose that \(\int_{\Omega}g_1~d\mu<+\infty\) and
\(g_n\downarrow g\) a.e., that is, we have \(g=\lim_{n}g_{n}\) a.e. and, for
every \(n\), \(g_{n}\geq g_{n+1}\) a.e. Then, the integrals
\(\int_{\Omega}g~d\mu\) and \(\int_{\Omega}g_n~d\mu\) exist for all \(n\), and
		\[\int_{\Omega}g_n~d\mu\downarrow\int_{\Omega}g~d\mu.\]
	\end{enumerate}
\end{thrm}
\begin{proof}
	\begin{enumerate}
			\item Let \(P_{n}=\left\{g_{n}>g_{n+1}\right\}\), and
\(P=\bigcup_{n}P_{n}\). Note that \(\mu(P)=0\). Let \(L\) be the set where
\(g=\lim_{n}g_{n}\), and \(G=P\cup L^c\). Note that \(\mu(G)=0\) too. Define \(\overline g=gI_{G^c}\) and a sequence of functions \(\overline g_{n}=g_{n}I_{G^c}\), so that
\(\overline g_{n}\uparrow \overline g\),
\(\int_{\Omega}\overline g~d\mu=\int_{\Omega}g~d\mu\) and
\(\int_{\Omega}\overline{g}_{n}~d\mu=\int_{\Omega}g_{n}~d\mu\) for every \(n\).
Since \(\int_{\Omega}\overline g_{1}~d\mu>-\infty\), it must be that
\(\overline g_{1}^-\) is integrable, and thus finite almost everywhere. Let
\(A\) be the set where \(\overline g_{1}^-\) is finite. Note that, since the
sequence \(\{\overline g_{n}\}_{n\in\mathbb{N}}\) is increasing, then
\(\{\overline g_{n}^-\}_{n\in\mathbb{N}}\) is decreasing. Therefore, every
\(\overline g_{n}^-\) is integrable and finite on \(A\). This also tells us that
\(\int_{\Omega}\overline g_{n}~d\mu\) exists for all \(n\). Define
\(h_{n}=I_{A}\cdot\overline g_{n}+I_{A}\overline g_{1}^-\), so that
\(\int_{\Omega}h_{n}~d\mu=\int_{\Omega} g_{n}~d\mu+\int_{\Omega}\overline g_{1}^-~d\mu\).
Then, the functions \(h_{n}\) are nonnegative and increase to
\(I_{A}\cdot\overline g+I_{A}\overline g_{1}^-\). By the \hyperref[theorem:Monotone
Convergence]{Monotone Convergence Theorem},
		\[ \int_{\Omega}g_{n}~d\mu+\int_{\Omega}\overline g_{1}~d\mu\uparrow \int_{\Omega} 	g~d\mu+\int_{\Omega}\overline g_{1}^-~d\mu
		\] The proof is completed by noting that
\(\int_{\Omega}\overline g_{1}~d\mu\) is finite, and can therefore be substracted
from both sides of the expression above.
		\item Apply the previous section to \(\{-g_{n}\}_{n\in\mathbb{N}}\) and
				\(-g\).\footnote{In this theorem, hypotheses were simplified with respect to \cite{ash1972real}, and the proof is (somewhat) original.}
	\end{enumerate}
\end{proof}
Note that, if \(\left\{g_n\right\}_{n\in\mathbb{Z}^+}\) is a sequence of Borel
measurable functions, then \(g=\sup_nf_n\) and \(h=\inf_nf_n\) defined pointwise
(that is, \(g(\omega)=\sup_nf_n(\omega)\), and similarly for \(h\)) are Borel
measurable too: for all \(c\in\overline{\mathbb{R}}\),
\[
\left\{g\leq c\right\}=\bigcap_n\left\{f_n\leq c\right\}\in\mathcal{F}
.\]
Therefore, if we define them pointwise as well, \(\liminf_nf_n\) and
\(\limsup_nf_n\) are measurable too. We can now present a very important theorem
regarding these functions:

\begin{thrm}[Fatou's Lemma]\label{theorem:Fatou's Lemma} Let
\(f_1,f_2,\dots,f\) be Borel measurable. Then,
	\begin{enumerate}
		\item If \(f_n\geq f\) for all \(n\), where
\(\int_{\Omega}f~d\mu>-\infty\), then the integral
\(\int_{\Omega}\liminf_nf_n~d\mu\) exists and
		\[\liminf_n\int_{\Omega}f_n~d\mu\geq\int_{\Omega}\liminf_nf_n~d\mu.\]
		\item If \(f_n\leq f\) for all \(n\), where
\(\int_{\Omega}f~d\mu<+\infty\), then the integral
\(\int_{\Omega}\limsup_nf_n~d\mu\) exists and
		\[\limsup_n\int_{\Omega}f_n~d\mu\leq\int_{\Omega}\limsup_nf_n~d\mu.\]
	\end{enumerate}
\end{thrm}

\begin{proof}
	\begin{enumerate}
		\item Define \(g_n=\inf_{k\geq n}f_k\) and
\(g=\sup_n g_n=\liminf_nf_n\). Then, \(g_n\uparrow g\) and \(g_1\geq f\), whence
\(\int_{\Omega}g_1~d\mu\) exists and is greater that \(-\infty\). By
\Cref{theorem:Extended Monotone Convergence},
\(\int_{\Omega}g_n~d\mu\uparrow\int_{\Omega}g~d\mu\). Moreover, since
\(g_n\leq f_k\) for all \(k\geq n\), we have
\(\int_{\Omega}g_n~d\mu\leq\int_{\Omega}f_k~d\mu\) for all \(k\geq n\), so
\(\int_{\Omega}g_n~d\mu\leq\inf_{k\geq n}\int_{\Omega}f_k~d\mu\). Therefore, we
have
		\[\int_{\Omega}\liminf_nf_n~d\mu=\int_{\Omega}g~d\mu=\sup_n\int_{\Omega}g_n~d\mu\geq\sup_n\inf_{k\geq n}\int_{\Omega}f_k~d\mu=\liminf_n\int_{\Omega}f_n~d\mu\]
		\item Note that \(-f_1,-f_2,\dots,-f\) satisfy all the hypotheses of the
last item. Thus,
		\[\liminf_n\int_{\Omega}-f_n~d\mu\geq\int_{\Omega}\liminf_n-f_n~d\mu.\]
The result is obtained by multiplying the inequality by \(-1\).
	\end{enumerate}
\end{proof}
The following result can be obtained as a simple corollary
of \hyperref[theorem:Fatou's Lemma]{Fatou's Lemma}. It is one of the most important
theorems in analysis regarding integration:

\begin{thrm}[Dominated Convergence Theorem]\label{theorem:Dominated
Convergence} If \(f_1,f_2,\dots,f,g\) are Borel measurable functions, \(g\) is
\(\mu\)-integrable, \(|f_n|\leq g\)  and \(f_n\to f\) \(\mu\)-a.e., then \(f\)
is \(\mu\)-integrable and
	\[\int_{\Omega}f~d\mu=\lim_n\int_{\Omega}f_n~d\mu.\]
\end{thrm}
\begin{proof} By taking limits on the inequality \(|f_n|\leq g\), one deduces
that \(|f|\leq g\) \(\mu\)-a.e., hence \(f\) is integrable. Furthermore,
\(-g\leq f_n\leq g\). Since the sequence of \(f_n\) converges to \(f\) almost
everywhere, we have \(\liminf_nf_n=\limsup_nf_n=f\) \(\mu\)-a.e. Thus,
\(\int_{\Omega}\liminf_nf_n~d\mu=\int_{\Omega}\limsup_nf_n~d\mu=\int_{\Omega}f~d\mu\).
By Fatou's lemma,
	\[\liminf_n\int_{\Omega}f_n~d\mu\geq\int_{\Omega}f~d\mu\geq\limsup_n\int_{\Omega}f_n~d\mu.\]
	
	It follows that the limit \(\lim_n\int_{\Omega}f_n~d\mu\) exists and is
equal to \(\int_{\Omega}f~d\mu\).
\end{proof}
\begin{thrm}\label{theorem:inequality of integrals implies inequality of
functions} If \(\mu\) is \(\sigma\)-finite on \(\mathcal{F}\), \(g\) and \(h\) are Borel
measurable, \(\int_{\Omega}g~d\mu\) and \(\int_{\Omega}h~d\mu\) exist, and
\(\int_{A}g~d\mu\leq\int_{A}h~d\mu\) for all \(A\in\mathcal{F}\), then \(g\leq h\)
\(\mu\)-a.e.
\end{thrm}
\begin{proof} Decompose \(\Omega\) into countably many subsets with finite
measure \(A_n\). Regard each \(A_n\) as a finite measure space. Then, it
suffices to show that \(g\leq h\) \(\mu\)-a.e. in \(A_n\) for all \(n\in\mathbb{Z}^+\).
This allows us to suppose, without loss of generality, that \(\mu\) is finite.
	
	Let \(F\) be the set where \(h\) is finite. Define the sets
\(B=\left\{g>h\right\}\cap F\)  and, for
each \(n\in\mathbb{Z}^+\),
\(B_n=\left\{g\geq h+\frac{1}{n}\right\}\cap\left\{\left|h\right|\leq n\right\}\cap F\),
so that \(B_n\uparrow B\). Now note that, on one side,
\(\int_{B_n}g~d\mu\leq\int_{B_n}h~d\mu\leq n\mu(B_n)<+\infty\) (because \(\int_{A}g~d\mu\leq\int_{A}h~d\mu\) for all \(A\), and \(h\leq n\) on \(B_n\)), and on the
other,
	\[\int_{B_{n}}g~d\mu\geq\int_{B_n}h~d\mu+\mu(B_n)/n\geq\int_{B_n}g~d\mu+\mu(B_n)/n.\]
This implies that \(\mu(B_n)=0\). Since \(\mu(B_n)\uparrow\mu(B)\), it follows
that \(\mu(B)=0\).  Now let us consider
\(F^c=\left\{h=-\infty\right\}\cup\left\{h=+\infty\right\}\).
Clearly, \(g\leq h\) on \(\left\{h=+\infty\right\}\).
Define \(C=\left\{h=-\infty\right\}\cap\left\{g>h\right\}\), and
\(C_{n}=C\cap\left\{g\geq -n\right\}\). Therefore,
\(C_n\uparrow C\), and
	\[-\infty\cdot\mu(C_{n})=\int_{C_n}h~d\mu\geq\int_{C_n}g~d\mu\geq -n\mu(C_n),\]
	hence \(\mu(C_n)=0\) (\(\mu(C_n)=+\infty\) is impossible because \(\mu\) is finite). Since \(\mu(C_n)\uparrow\mu(C)\), we have \(\mu(C)=0\).
\end{proof}
\begin{corl}\label{corollary:equality of integrals implies equality of
functions} If \(\mu\) is \(\sigma\)-finite on \(\mathcal{F}\), \(g\) and \(h\) are Borel
measurable, \(\int_{\Omega}g~d\mu\) and \(\int_{\Omega}h~d\mu\) exist, and
\(\int_{A}g~d\mu=\int_{A}h~d\mu\) for all \(A\in\mathcal{F}\), then \(g=h\) \(\mu\)-a.e.
\end{corl}

We end the section with two useful theorems. The fisrt is a very general form of a change of
variables formula:
\begin{thrm}[Image Measure Theorem]\label{theorem:Image Measure} Let
\(\left(\Omega_{1},\mathcal{F}_{1}\right)\), \(\left(\Omega_{2},\mathcal{F}_{2}\right)\) be
measurable spaces and \(\mu_{1}\) a measure on \(\mathcal{F}_{1}\). Let
\(T\colon\Omega_{1}\to\Omega_{2}\) be a measurable mapping.
	
	Define a measure \(\mu_{2}\) on \(\mathcal{F}_{2}\) by
\(\mu_{2}(A)=\mu_{1}(T^{-1}(A))\). The measure \(\mu_{2}\) is often called the
\textbf{push-forward measure} or \textbf{image measure} of \(\mu_{1}\) by \(T\)
and denoted by \(T_{*}(\mu_{1})\) or \(\mu_{1}\circ T^{-1}\).
	
	Then, for every Borel measurable function
\(f\colon\Omega_{2}\to\overline{\mathbb{R}}\) and every \(A\in\mathcal{F}_{2}\), one has
	\[ \int_{A}f~d\mu_{2}=\int_{T^{-1}(A)}\left(f\circ T\right)d\mu_{1},
	\] in the sense that if one integral exists, so does the other, and the two
are equal.
\end{thrm}
\begin{proof} First suppose that \(f\) is an indicator \(I_{B}\). Then,
\(f\circ T=I_{T^{-1}(B)}\), and thus the desired formula becomes
\[\int_{\Omega_{2}}I_{A\cap B}d\mu_{2}=\int_{\Omega_{1}}I_{T^{-1}(A)\cap T^{-1}(B)}d\mu_{1},\] which is equivalent to \(\mu_{2}(A\cap B)=\mu_{1}(T^{-1}(A\cap B))\), and
this is true by definition.
	
If \(f\) is a nonnegative simple function \(\sum_{i=1}^{n}x_{i}I_{B_{i}}\),
then
\[\int_{A}fd\mu_{2}=\sum_{i=1}^{n}x_{i}\int_{A}I_{B_{I}}d\mu_{2}=\sum_{i=1}^{n}x_{i}\int_{T^{-1}(A)}\left(I_{B_{i}}\circ T\right)d\mu_{1}=\int_{T^{-1}(A)}\left(f\circ T\right)d\mu_{1}.\]
If \(f\) is a nonnegative Borel measurable function, take
\(s_{1},s_{2},\dots\) nonnegative simple functions increasing to \(f\). Thus,
\(s_{1}\circ T, s_{2}\circ T,\dots\) are nonnegative simple functions increasing
to \(f\circ T\). The \hyperref[theorem:Monotone Convergence]{Monotone Convergence
Theorem} yields the result.

Finally, if \(f=f^+-f^-\), the result proved so far yields the desired
formula for \(f^+\) and \(f^-\), since \((f\circ T)^+=f^+\circ T\) and
\((f\circ T)^-=f^-\circ T\). If, say, \(\int_{A}f^+d\mu_{2}\) is finite, so is
\(\int_{T^{-1}(A)}\left(f^+\circ T\right)d\mu_{1}\) and additivity implies the
result for \(f\).
\end{proof}
The second result is quite simple, both to state and prove, but it is a very useful theorem in analysis.
\begin{thrm}[Borel-Cantelli Lemma]\label{theorem:Borel-Cantelli Lemma}
  Let \(\mu\) be a measure on a \(\sigma\)-field \(\mathcal{F}\). Let \(A_{1},A_{2},\dotsc\) be a sequence
  of measurable sets. Then
  \[\sum_n\mu(A_{n})<\infty \text{ ~~~ \emph{implies} ~~~ }\mu\left(\limsup_{n}A_{n}\right)=0.\]
\end{thrm}
\begin{proof}
  For every \(n\), we have \[\mu\left(\bigcup_{k\geq n}A_{k}\right)\leq\sum_{k\geq n}\mu(A_{k}).\]
  It follows, by \cref{proposition:limit of decreasing sets} that \(\mu(\limsup_{n}A_{n})\leq0\), which in turn yields the desired result.
\end{proof}

%!TeX root=Final.tex

\chapter{ADVANCED RESULTS IN MEASURE THEORY}\label{chapter:advanced results in measure theory}

In the previous part of the text, we developed the \emph{basics} of Measure Theory. Many possibilities open now: for instance, it is possible to prove the famous Radon-Nikodým Theorem with the theory developed so far. This is done in great detail in \Cref{chapter:relations between measures}.

In this text, however, we will focus on developing all the theory necessary to prove the \hyperref[theorem:Kolmogorov Extension]{Kolmogorov Extension Theorem}. This is what will be done in this chapter.
\section{Daniell Theory}\label{section:Daniell Theory}
The study of linear functionals\footnote{A \emph{functional} is a function from a given vector space of real-valued functions to \(\mathbb{R}\).} is a very fruitful topic in Analysis. Weak topologies are based on the concept of continuity of linear functionals in a given Banach space, and they provide a solid basis for many useful theorems regarding convergence in function spaces \cite{brezis}.

The integral, in particular, is itself a linear functional from the space of integrable functions on a given measure space to \(\mathbb{R}\). As we will see, under appropiate hypotheses it is possible to construct a measure so that a given linear functional can be expressed as the integral over that measure. The approach followed to do this will be that of Daniell Theory. But first, we need some basic concepts and tools:

\begin{defn}
Let \(\mathcal{D}\) be a class of subsets of some nonempty set \(\Omega\). We say that \(\mathcal{D}\) is a \textbf{Dynkin system} (or \textbf{D-system} for short) if
\begin{enumerate}
		\item \label{definition:D-system 1}\(\Omega\in\mathcal{D}\).
		\item \label{definition:D-system 2}\(\mathcal{D}\) is closed under set differences; that is, if \(A\subseteq B\) with \(A,B\in\mathcal{D}\), then \(A\setminus B\in\mathcal{D}\).
		\item \label{definition:D-system 3}\(\mathcal{D}\) is closed under increasing sequences; that is, if \(A_1,\dots,A_n,\dots\) is a sequence of sets in \(\mathcal{D}\) and \(A_n\uparrow A\), then \(A\in\mathcal{D}\).
\end{enumerate}
\end{defn}
By conditions \ref{definition:D-system 1} and \ref{definition:D-system 2}, \(\mathcal{D}\) is closed under complementation. By conditions \ref{definition:D-system 2} and \ref{definition:D-system 3},  \(\mathcal{D}\) is a monotone class. If \(\mathcal{D}\) is closed under finite unions (or intersections), then it is a \(\sigma\)-field.

The arbitrary intersection of D-systems is a D-system too. This guarantees the existence of generated D-systems. If \(\mathcal{S}\) is a class of subsets of some nonempty set \(\Omega\), we will denote its generated D-system - that is, the smallest D-system containing \(\mathcal{S}\) - by \(\mathcal{D}(\mathcal{S})\).

Our interest on D-systems is motivated by the following theorem. It is analogous to the \hyperref[theorem:Monotone Class]{Monotone Class Theorem}.
\begin{thrm}[Dynkin System Theorem]\label{theorem:D-system}
		Let \(\mathcal{S}\) be a class of subsets of ~\(\Omega\). If 
		 \(\mathcal{S}\) is closed under finite intersection, then \(\mathcal{D}(\mathcal{S})=\sigma(\mathcal{S})\). In particular, if \(\mathcal{D}\) is a Dynkin system and \(\mathcal{S}\subseteq\mathcal{D}\), then \(\sigma(\mathcal{S})\subseteq\mathcal{D}\).
\end{thrm}
\begin{proof}
Let \(\mathcal{D}_0=\mathcal{D}(\mathcal{S})\) and \(\mathcal{F}=\sigma(\mathcal{S})\). Define \(\mathcal{V}=\{A\in\mathcal{D}_0\left|A\cap B\in \mathcal{D}_0 \text{ for every }B\in\mathcal{S} \right.\}\). Now, \(\mathcal{S}\subseteq\mathcal{V}\) since \(\mathcal{S}\) is closed under intersection and a subset of \(\mathcal{D}_0\). Also, it is easy to check that \(\mathcal{V}\) is a D-system because \(\mathcal{D}_0\) is. Thus, \(\mathcal{V}=\mathcal{D}_0\). Now define \(\mathcal{V}'=\{A\in\mathcal{D}_0\left|A\cap B\in\mathcal{D}_0 \text{ for every }B\in\mathcal{D}_0\right.\}\). Again, \(\mathcal{V}'\) is a D-system and, clearly, \(\mathcal{S}\subseteq\mathcal{V}\subseteq\mathcal{V}'\); hence \(\mathcal{V}'=\mathcal{D}_0\).

It follows that \(\mathcal{D}_0\) is closed under finite intersection, hence a \(\sigma\)-field. Thus, \(\mathcal{F}\subseteq\mathcal{D}_0\). The other inclusion is immediate because every \(\sigma\)-field is a D-system. Finally, \(\mathcal{F}=\mathcal{D}_0\subseteq\mathcal{D}\).
\end{proof}
In the \hyperref[theorem:Monotone Class]{Monotone Class Theorem}, a weaker hypothesis is imposed to the generating set (in this theorem, \(\mathcal{S}\) need not be a field), but a stronger hypothesis is imposed
to the structure (a monotone class need not be a D-system).

\begin{corl}\label{corollary:measures agree on a class of sets closed by intersection}
		Let \(\mathcal{S}\) be a class of subsets of ~\(\Omega\) that is closed under finite intersection and such that \(\Omega\in\mathcal{S}\). If \(\mu_1\) and \(\mu_2\) are finite measures on \(\sigma(\mathcal{S})\) that agree on \(\mathcal{S}\), then \(\mu_1=\mu_2\) on \(\sigma(\mathcal{S})\).
\end{corl}
\begin{proof}
		Let \(\mathcal{D}\) be the class of sets of \(\sigma(\mathcal{S})\) where \(\mu_1\) and \(\mu_2\) agree. Then, \(\mathcal{D}\) is a D-system:
		\begin{itemize}
				\item Condition \ref{definition:D-system 1} follows from the fact that \(\Omega\in\mathcal{S}\).
				\item Condition \ref{definition:D-system 2} follows from additivity.
				\item Condition \ref{definition:D-system 3} follows from \Cref{proposition:monotonicity of additive set functions}.
		\end{itemize}
		Hence, \(\sigma(\mathcal{S})=\mathcal{D}(\mathcal{S})\subseteq\mathcal{D}\). The reciprocal inclusion is true by definition.
\end{proof}
\begin{corl}
		Let \(\mathcal{S}\) be a class of subsets of ~\(\Omega\) that is closed under finite intersection and such that \(\Omega\in\mathcal{S}\). Let \(H\) be a vector space of real-valued functions on \(\Omega\), such that \(I_A\in H\) for each \(A\in\mathcal{S}\). Suppose that for every increasing sequence of nonnegative functions \(f_1,f_2\dots\) with a bounded limit (that is, \(f_n\uparrow f\) and there exists some \(M\in\mathbb{R}^+\) such that \(\left|f\right|\leq M\)), the limit function \(f\) belongs to \(H\).

		Then, \(I_A\in H\) for every \(A\in\sigma(\mathcal{S})\).
\end{corl}
\begin{proof}
		Let \(\mathcal{D}_0=\mathcal{D}(\mathcal{S})\) and \(\mathcal{D}=\{A\in\mathcal{D}_0\left|I_A\in H\right.\}\). Then, \(\mathcal{D}\) is a D-system, because:
		\begin{itemize}
				\item \(\Omega\in\mathcal{D}\) since \(\Omega\in\mathcal{S}\).
				\item If \(A\subseteq B\), \(A,B\in\mathcal{D}\), then \(I_{B\setminus A}=I_B-I_A\in H\).
				\item If \(A_n\uparrow A\), \(A_n\in\mathcal{D}\), then \(I_{A_n}\uparrow I_A\in H\).
		\end{itemize}
		Thus, \(\mathcal{D}=\mathcal{D}_0=\sigma(\mathcal{S})\), completing the proof.
\end{proof}
\begin{defn}\label{definition:Daniell Theory}
		Let \(L\) be a vector space of real-valued functions on a set \(\Omega\). We will say that \(L\) is closed under the \textbf{lattice operations} if~ \(\max(f,g)\in L\) and \(\min(f,g)\in L\) for every two functions \(f,g\in L\).

		If \(E\colon L\to \mathbb{R} \) is a linear functional, we say that it is \textbf{positive} if \(f\geq 0\) implies \(E(f)\geq 0\). From this, it follows that \(E\) is \textbf{monotone}; that is, \(f\geq g\) implies \(E(f)\geq E(g)\). Additionally, we will say that \(E\) is a \textbf{Daniell integral} if \(f_n\uparrow f, f_n\geq 0\) implies \(E(f_n)\uparrow E(f)\) and \(f_n\downarrow 0\) implies \(E(f_n)\downarrow E(0)=0\)\footnote{In practice, if we want to see that a given linear functional is a Daniell integral, it suffices to show that \(E\) is positive and that \(f_n\downarrow 0\) implies \(E(f_n)\downarrow 0\).}.

		If \(H\) is any class of functions from \(\Omega\) to \(\overline{\mathbb{R}}\), \(H^{+}\) will denote the class of nonnegative functions in \(H\), \(\{f\in H\left|f\geq 0\right.\}\). The collection of functions \(f\colon \Omega\to \overline{\mathbb{R}} \) such that there exists a sequence \(f_n\) in \(L^{+}\) with \(f_n\uparrow f\) will be denoted by \(L'\).

		If \(H\) is as above, the \(\sigma\)-field \textbf{generated} by \(H\), denoted by \(\sigma(H)\), is defined as the smallest \(\sigma\)-field making all functions in \(H\) Borel measurable; namely, \(\sigma(H)=\sigma(\mathcal{A})\), where \(\mathcal{A}=\left\{f^{-1}(B)\left|f\in H, B\in\right.\mathscr{B}\left(\overline{\mathbb{R}}\right)\right\}\).
\end{defn}
During the rest of the section, \(L\) will be a vector space as above, and \(E\) will be a Daniell integral on \(L\).

A great part of this section will follow a structure very similar to that of \Cref{section:Extension of measures}. We begin by extending \(E\) to \(L'\):

\begin{lemm}\label{lemma:extension of E to L'}
		Let \(\{f_n\}\) and \(\{g_n\}\) be sequences in \(L\) increasing to respective limits \(f\) and \(g\), with \(f\leq g\). Then,
		\[
				\lim_n E(f_n)\leq\lim_n E(g_n)
		.\]
		Hence, \(E\) may be extended to \(L'\) as \(E(\lim_nh_n)=\lim_nE(h_n)\).
\end{lemm}
\begin{proof}
		First, note that both limits exist (they may be \(+\infty\)) because we have increasing sequences of real numbers.
		Now, \(\min(f_m,g_n)\uparrow_n\min(f_m,g)=f_m\). Thus, \(E(f_m)=\lim_n E(\min(f_m,g_n))\leq\lim_n E(g_n)\). Take limits in \(m\) to complete the proof.
\end{proof}
We now study this extension to \(L'\):
\begin{lemm}\label{lemma:properties of extension of E to L'}
		The extension of \(E\) to \(L'\) has the following properties:
		\begin{enumerate}
				\item\label{lemma:properties of extension of E to L' 1} \(0\leq E(f)\leq+\infty\) for all \(f\in L'\).
				\item\label{lemma:properties of extension of E to L' 2} If \(f,g\in L'\) and \(f\leq g\), then \(E(f)\leq E(g)\).
				\item\label{lemma:properties of extension of E to L' 3} If \(f\in L'\) and \(c\) is a nonnegative real number, then \(cf\in L'\) and \(E(cf)=cE(f)\).
				\item\label{lemma:properties of extension of E to L' 4} If \(f,g\in L'\), then \(f+g,\min(f,g)\) and \(\max(f,g)\) all are in \(L'\), and
				\[
						E(f+g)=E(f)+E(g)=E(\min(f,g))+E(\max(f,g))
				.\]
				\item\label{lemma:properties of extension of E to L' 5} If \(f_n\in L'\) and \(f_n\uparrow f\), then \(E(f_n)\uparrow E(f)\).
		\end{enumerate}
\end{lemm}
\begin{proof}
		Items (i)-(iii) are immediate, either by definition or by \Cref{lemma:extension of E to L'}.

		To see \cref{lemma:properties of extension of E to L' 4}, simply take sequences \(f_n\uparrow f\), \(g_n\uparrow g\) in \(L'\), so that \(\min(f_n,g_n)\uparrow \min(f,g)\in L'\), \(\max(f_n,g_n)\uparrow \max(f,g)\in L'\) and \(f_n+g_n\uparrow f+g\in L'\). Additionally, \(E(f+g)=\lim_nE(f_n+g_n)=\lim_nE(f_n)+E(g_n)=E(f)+E(g)\). The last equality follows from linearity and the fact that \(f+g=\max(f,g)+\min(f,g)\).

		To see \ref{lemma:properties of extension of E to L' 5}, for each \(n\in\mathbb{Z}^{+}\) consider a sequence \(f_{nm}\) in \(L^{+}\) such that \(f_{nm}\uparrow_m f_n\). Define \(g_m=\max(f_{1m},\dots,f_{mm})\in L^{+}\), so that 
		\begin{equation}\label{equation:proof of properties of extension to L' 5}
		f_{nm}\leq g_m\leq f_m
		\end{equation}
		for \(n\leq m\). Take \(m\to+\infty\) to see that \(f_n\leq\lim_mg_m\leq f\), and then \(n\to+\infty\) to obtain \(g_m\uparrow f\); hence \(E(g_m)\uparrow E(f)\).
		Now apply \(E\) in equation (\ref{equation:proof of properties of extension to L' 5}) to obtain \(E(f_{nm})\leq E(g_m)\leq E(f_m)\). Let \(m\to+\infty\) to obtain \(E(f_n)\leq E(f)\leq\lim_m E(f_m)\). Now let \(n\to+\infty\) to obtain the desired result.
\end{proof}
We now begin the construction of a \(\sigma\)-field and a measure derived from \(E\). Henceforth, we assume that all constant functions belong to \(L\). We can rescale \(E\) so that \(E(1)=1\) (hence, \(E(c)=c\) for all \(c\in\mathbb{R}\)).

\begin{lemm}\label{lemma:extension of E to monotone class}
		Let \(\mathcal{C}\) be the class of subsets \(G\subseteq\Omega\) such that \(I_G\in L'\) and define \(\mu(G)=E(I_G)\). Then, \(\mathcal{C}\) and \(\mu\) satisfy all four conditions of \Cref{lemma:extension to monotone class}; namely:
		\begin{enumerate}
				\item \label{lemma:extension of E to monotone class
						coincides}\(\emptyset,\Omega\in\mathcal{C}\), \(\mu(\emptyset)=0\), \(\mu(\Omega)=1\) and
						\(0\leq\mu(A)\leq1\) for all \(A\in\mathcal{C}\)
				\item \label{lemma:extension of E to monotone class additivity} If
						\(G_1,G_2\in\mathcal{C}\), then \(G_1\cup G_2,G_1\cap G_2\in\mathcal{C}\) and
						\(\mu(G_1\cup G_2)+\mu(G_1\cap G_2)=\mu(G_1)+\mu(G_2)\).
				\item \label{lemma:extension of E to monotone class is monotone} If
						\(G_1,G_2\in\mathcal{C}\) and \(G_1\subseteq G_2\), then \(\mu(G_1)\leq\mu(G_2)\).
				\item \label{lemma:extension of E to monotone class incerasing limits} If
						\(G_n\in\mathcal{C}\), and \(G_n\uparrow G\), then \(G\in\mathcal{C}\) and
						\(\mu(G_n)\to\mu(G)\).
		\end{enumerate}
Hence, by \Cref{lemma:extension to outer measure properties} and \Cref{theorem:extension to Hcal}, the mapping \(\mu^*(A)=\inf\{\mu(G)\left|G\in\mathcal{C}, A\subseteq G\right.\}\) is a probability measure on the \(\sigma\)-field \(\mathcal{H}=\{H\subseteq\Omega\left|\mu^*(H)+\mu^*(H^c)=1\right.\}\) such that \(\mu\equiv\mu^*\) on \(\mathcal{C}\).
\end{lemm}
\begin{proof}
		\begin{enumerate}
				\item Since \(L\) contains all constant functions, \(I_{\emptyset}=0\) and \(I_{\Omega}=1\) are in \(L\). Additionally, \(E(c)=c\); in particular, \(E(I_{\emptyset})=0\) and \(E(I_{\Omega})=1\). The rest follows from \(E\) being monotone.
				\item Direct consequence of \Cref{lemma:properties of extension of E to L' 4}, taking \(f=I_{G_1}\) and \(g=I_{G_2}\). Simply note that \(\min(I_{G_1},I_{G_2})=I_{G_1\cap G_2}\) and \(\max(I_{G_1},I_{G_2})=I_{G_1\cup G_2}\).
				\item Immediate by the monotonicity of \(E\).
				\item Consequence of \Cref{lemma:properties of extension of E to L' 5}, taking \(f_n=I_{G_n}\); note that \(f=I_{G}\).
		\end{enumerate}
\end{proof}
Just as in \Cref{section:Extension of measures}, it is true that \(\sigma(\mathcal{C})\subseteq\mathcal{H}\). However, the proof of this is a little harder this time, and we will need some previous results.

First, we investigate Borel measurability of functions on \(L'\) relative to the \(\sigma\)-field \(\sigma(\mathcal{C})\).
\begin{lemm}\label{lemma:functions in L' are measurable}
		If \(f\in L'\) and \(a\in\mathbb{R}\), then the set \(Z=\{f(\omega)>a\}\) belongs to the class \(\mathcal{C}\). Therefore, \(f\) is Borel measurable with respect to \(\sigma(\mathcal{C})\).
\end{lemm}
\begin{proof}
		Let \(f_n\) be a sequence in \(L^{+}\) such that \(f_n\uparrow f\). For any \(a\in\mathbb{R}\), \((f_n-a)^{+}=\max(f_n-a,0)\in L^{+}\), so that \((f-a)^{+}=\lim_n(f_n-a)^{+}\in L'\). Then, by \Cref{lemma:properties of extension of E to L' 3}, for every \(k\in\mathbb{Z}^{+}\), \(k(f-a)^{+}\in L'\), and
		\[
				\min(1,k(f-a)^{+})\uparrow_k I_{Z}
		.\]
		By \Cref{lemma:properties of extension of E to L' 5}, \(I_Z\in L'\), hence \(Z\in\mathcal{C}\).
\end{proof}

\begin{lemm}\label{lemma:equality of sigma-fields in Daniell theory}
		The \(\sigma\)-fields \(\sigma(L), \sigma(L')\) and \(\sigma(\mathcal{C})\) are identical.
\end{lemm}
\begin{proof}
		By \Cref{lemma:functions in L' are measurable}, every function in \(L'\) is measurable. Therefore, \(\sigma(L')\subseteq\sigma(\mathcal{C})\). The other inclusion is immediate by the definition of \(\mathcal{C}\): if \(G\in\mathcal{C}\), then \(I_G\in L'\); hence \(G=\left\{I_G=1\right\}\in\sigma(L')\). Thus, \(\mathcal{C}\subseteq\sigma(L')\), from where \(\sigma(\mathcal{C})\subseteq\sigma(L')\).

		Now let \(f\in L'\) and consider a sequence in  \(L\) with \(f_n\uparrow f\). Since every \(f_n\) is \(\sigma(L)\)-Borel measurable, it follows that their pointwise limit \(f\) is \(\sigma(L)\)-Borel measurable. Thus, \(\sigma(L')\subseteq\sigma(L)\), because \(\sigma(L')\) is the smallest \(\sigma\)-field making every \(f\in L'\) Borel measurable.

		If \(f\in L\), we can split \(f=f^{+}-f^{-}\). Note that, since \(L\) is closed under the lattice operations and contains all constant functions, then both \(f^{+}\) and \(f^{-}\) are in \(L\). Now \(f^{+},f^{-}\in L^{+}\subseteq L'\), and thus \(f\) is \(\sigma(L')\)-Borel measurable. It follows that \(\sigma(L)\subseteq\sigma(L')\).
\end{proof}
\begin{lemm}\label{lemma:outer measure on L'}
		For any \(A\subseteq\Omega\), \(\mu^*(A)=\inf\{E(f)\left|f\in L', f\geq I_A\right.\}\).
\end{lemm}
\begin{proof}
		By definition of \(\mu\), \(\mu^*(A)=\inf\{E(I_G)\left|G\in\mathcal{C}, A\subseteq G\right.\}\). It is then clear that \(\mu^*(A)=\inf\{E(f)\left|f=I_G, G\in\mathcal{C}, f\geq I_A\right.\}\geq\inf\{E(f)\left|f\in L', f\geq I_A\right.\}\). To see the other inequality, let \(f\) be a function in \(L'\) with \(f\geq I_A\), and a real number in the interval \(a\in(0,1)\). Let \(Z=\{f>a\}\). Note that, since \(f\) is measurable, we have \(Z\in\sigma(\mathcal{C})\). Additionally, \(f\geq I_A\) and since \(I_A\) is \(1\) on \(A\), it follows that \(A\subseteq Z\). Thus \(\mu^*(A)\leq \mu(Z)=E(I_Z)\). Now note that \(f\geq aI_Z\), and therefore \(E(f)\geq aE(I_Z)\). Finally, we have \(\mu^*(A)\leq \frac{E(f)}{a}\). The result is deduced by letting \(a\to 1^{-}\).
\end{proof}
\begin{lemm}\label{lemma:Hcal contains Ccal}
		If \(\mathcal{H}=\{H\subseteq\Omega\left|\mu^*(H)+\mu^*(H^c)=1\right.\}\), then \(\mathcal{G}\subseteq\mathcal{H}\). Therefore, \(\sigma(\mathcal{C})\subseteq\mathcal{H}\).
\end{lemm}
\begin{proof}
		Let \(G\in\mathcal{C}\). Since \(I_G\in L'\) by definition, we can find a sequence of functions in \(L^{+}\) such that \(f_n\uparrow I_G\). On one side, this implies that \(E(f_n)\uparrow E(I_G)=\mu(G)=\mu^*(G)\). On the other, it imples that \(1-f_n\downarrow 1-I_{G}=I_{G^c}\geq 0\), and then \(1-f_n\in L^{+}\subseteq L'\). Hence, by \Cref{lemma:outer measure on L'},
		\[
				\mu^*(G^c)=\inf\{E(f)\left|f\in L', f\leq I_{G^c}\right.\}\geq\inf_nE(1-f_n)=1-\lim_nE(f_n)=1-E(I_G)
		.\]
		Therefore, \(\mu^*(G)+\mu^*(G^c)\leq 1\). Since the inequality \(\mu^*(G)+\mu^*(G^c)\geq 1\) always holds by \Cref{lemma:defining property of Hcal}, we have \(G\in\mathcal{H}\).
\end{proof}
Finally, we are in position to obtain the main result of the section, and a useful corollary. For clarity, the hypotheses accumulated so far are gathered in the statement of the theorem.
\begin{thrm}[Daniell Representation Theorem]\label{theorem:Daniell Representation}
		Let \(L\) be a vector space of real-valued functions on the set \(\Omega\) that contains all constant functions and is closed under lattice operations. Let \(E\) be a Daniell integral on \(L\) such that \(E(1)=1\).

		Then, there is a unique probability measure \(P\) on \(\sigma(L)\) such that each \(f\in L\) is \(P\)-integrable and
		\[
				E(f)=\int_{\Omega}f~dP
		.\]
\end{thrm}
\begin{proof}
		Let \(P\) be the restriction of \(\mu^*\) to \(\sigma(L)\). By \Cref{lemma:extension of E to monotone class} and \Cref{lemma:Hcal contains Ccal}, \(P\) is a probability measure on \(\sigma(\mathcal{C})\), which is equal to \(\sigma(L)\) by \Cref{lemma:equality of sigma-fields in Daniell theory}.
		
		As is usual when working with measures, we will show the result for indicators first and then work upwards. However, a specific detail needs to be addressed: we cannot consider any indicator function \(I_G\), with \(G\in\sigma(\mathcal{C})\), since if \(G\in\sigma(\mathcal{C})\setminus \mathcal{C}\), then \(I_G\not\in L'\), and then \(E(I_G)\) need not be defined. However, the result clearly holds for indicators of sets \(B\in\mathcal{C}\):

		\[
			E(I_B)=\mu^*(B)=P(B)=\int_{\Omega}I_B~dP
		.\]
		
		By additivity, it also holds for (real) linear combinations of such indicators (a subclass of simple functions). However, this class of functions is enough to approximate - via limits - all functions in \(L^+\): take some \(f\in L^+\), and consider the sequence of functions

		\[
			s_n=\sum_{i=1}^{n2^n}\frac{i-1}{2^n}I_{\left\{(i-1)/2^n<f\leq i/2^n\right\}}+nI_{\{f>n\}}
		.\]
		
		Note that since \(I_{\left\{(i-1)/2^n<f\leq i/2^n\right\}}=I_{\{f>(i-1)/2^n\}}-I_{\{f>i/2^n\}}\) and \(\{f>a\}\in \mathcal{C}\) (\cref{lemma:functions in L' are measurable}), we can consider \(E(s_n)\) and \(E(s_n)=\int_{\Omega}s_n~dP\). Additionally, since \(s_n \uparrow f\), following the \hyperref[theorem:Monotone Convergence]{Monotone Convergence Theorem} and the fact that \(E\) is a Daniell integral, we have \(E(f)=\lim_nE(s_n)=\lim_n\int_{\Omega}s_n~dP=\int_{\Omega}f~dP\).

		If \(f\in L\), split \(f=f^{+}-f^{-}\). Since \(L\) is closed under the lattice operations and contains all constant functions (the function \(0\), in particular), we have \(f^{+},f^{-}\in L\). Then, by additivity,
		\[
				E(f)=E(f^{+})-E(f^{-})=\int_{\Omega}f^{+}~dP-\int_{\Omega}f^{-}~dP=\int_{\Omega}f~dP
		.\]
		(Since \(E\) is real-valued, both integrals are finite).

		To see uniqueness, consider another such probability measure \(P'\) and note that the class of sets of \(\sigma(\mathcal{C})\) where they coincide is a D-system and contains \(\mathcal{C}\), because \(\int_{\Omega}f~dP=\int_{\Omega}f~dP'\) for all \(f\in L\); hence on \(L'\). By the definition of \(\mathcal{C}\), \(I_G\in L'\) for all \(G\in\mathcal{C}\); thus, \(P(G)=\int_{\Omega}I_G~dP=\int_{\Omega}I_G~dP'=P'(G)\). Finally, by \Cref{corollary:measures agree on a class of sets closed by intersection}, \(P\) and \(P'\) agree on all of \(\sigma(\mathcal{C})\).
\end{proof}
Finally, we give an approximation theorem.
\begin{thrm}\label{theorem:approximation theorem in Daniell theory}
		Under the hypotheses and notation of \Cref{theorem:Daniell Representation}, assume in addition that \(L\) is closed under limits of uniformly convergent sequences of functions. Let
		\[
		\mathcal{C}'=\left\{G\subseteq\Omega\left|G=\{f(\omega)>0\} \text{ for some }f\in L^{+}\right.\right\}
		.\]
		Then,
		\begin{enumerate}
				\item\label{theorem:approximation theorem in Daniell theory 1} \(\mathcal{C}'=\mathcal{C}\) 
				\item\label{theorem:approximation theorem in Daniell theory 2} If \(A\in\sigma(L)\), then \(P(A)=\inf\{P(G)\left|G\in\mathcal{C}', A\subseteq G\right.\}\).
				\item\label{theorem:approximation theorem in Daniell theory 3} If \(G\in\mathcal{C}\), then \(P(G)=\sup\{E(f)\left|f\in L^{+},f\leq I_G\right.\}\)
		\end{enumerate}
\end{thrm}
\begin{proof}
		\begin{enumerate}
				\item First note that \(\mathcal{C}'\subseteq\mathcal{C}\) by \Cref{lemma:functions in L' are measurable}. Conversely, take \(G\in\mathcal{C}\) and consider a sequence in \(L^{+}\) with \(f_n\uparrow I_G\). 
				Define \(f=\sum_{n} 2^{-n}f_n\). Since \(0\leq f_n\leq 1\), the series is uniformly convergent, hence \(f\in L^{+}\). Now, \(f(\omega)=0\) if, and only if, \(f_n(\omega)=0\) for all \(n\). Thus,
				\[
					\{f(\omega)>0\} = \bigcup_{n} \{f_n(\omega)>0\}=\{I_G(\omega)>0\} = G	
				.\]
				Consequently, \(G\in \mathcal{C}\). 
				\item Immediate from \ref{theorem:approximation theorem in Daniell theory 1} and the fact that \(P=\mu^* \) on \(\sigma(L)\) (using the definition of \(\mu^*\)).
				\item If \(f\in L^{+}\) and \(f\leq I_G\), then \(E(f) \leq E(I_G)=P(G)\). Conversely, let \(G\in \mathcal{C}\) and consider a sequence in \(L^{+}\) with \(f_n\uparrow I_G\). Then \(f_n\leq f\) and \(P(G)=E(I_G)=\lim_n E(f_n) = \sup_n E(f_n)\), hence \(P(G) \leq \sup \{E(f)\left|f\in L^{+}, f \leq I_G\right.\}\). 

		\end{enumerate}
\end{proof}
\section{Measure and Topology}\label{section:Measure and Topology}

In this section, we will study how the Measure Theory developed so far relates to Topology. The main goal is to obtain a result which allows us to approximate probabilities of Borel sets by probabilities of compact sets. This will be a key tool in our proof of the \hyperref[theorem:Kolmogorov Extension]{Kolmogorov Extension Theorem}, which is the main goal of this work.
\begin{defn}
		Let \(\Omega\) be a \textbf{normal} topological space, that is, \(\Omega\) is Hausdorff and for every two disjoint closed subsets \(A, B\subseteq\Omega\), there exist two disjoint open sets \(U, V\subseteq\Omega\) such that \(A\subseteq U\) and \(B\subseteq V\).

		We will denote the class of all continuous functions from \(\Omega\) to \(\mathbb{R}\) (with the standard topology) as \(C(\Omega)\), and the class of all such functions that are, additionally, bounded, as \(C_b(\Omega)\).
\end{defn}

The basic property we will need about normal topological spaces is \emph{Urysohn's Lemma}. It has an admittedly technical proof, but it uses only the definition and elementary topology notions.

\begin{thrm}[Urysohn's Lemma]\label{Urysohn}
		If \(A\) and \(B\) are disjoint closed subsets of a normal topological space \(\Omega\), there exists a continuous function \(f\colon \Omega\to \left[0,1\right]\) such that \(f=0\) on \(A\) and \(f=1\) on \(B\).
\end{thrm}
\begin{proof}
		In this proof, \(D'\) and \(D\) will denote the sets of \emph{Dyadic rationals} in \((0,1)\) and \([0,1]\), respectively; a Dyadic rational is a rational number of the form \(\frac{k}{2^n}\), with \(n,k\in\mathbb{Z}\). 

		An immediate characterisation of normality is the following: if \(U\) is an open set, \(V\) is closed and \(V\subseteq U\), then there exist an open set \(U'\) and a closed set \(V'\) such that \(V\subseteq U'\subseteq V'\subseteq U\).

		We will use this characterisation to show that there exists a family of open sets \(U(r)\) and closed sets \(V(r)\), where \(r\in D'\), such that
		\begin{itemize}
				\item \(A\subseteq U(r)\subseteq V(r)\subseteq B^c\) for any \(r\in D'\),
				\item If \(r,s\in D'\) with \(r<s\), then \(V(r)\subseteq U(s)\).
		\end{itemize}
		Extend this notation from \(D'\) to \(D\) as \(U(1)=B^c\), \(V(0)=A\). Note that any \(r\in D\) can be written as \(\frac{k}{2^n}\), where \(n\) is a positive integer and \(k=0,\dots,2^n\). This allows us to proceed our construction by induction on \(n\):

		For \(n=1\), note that \(B^c\) is open and \(A\subseteq B^c\). Hence, applying the characterisation of normality, there exist an open set \(U\left(\frac{1}{2}\right)\) and a closed set \(V\left(\frac{1}{2}\right)\) such that \(A\subseteq U\left(\frac{1}{2}\right)\subseteq V\left(\frac{1}{2}\right)\subseteq B^c\).

		For the inductive step, consider some  integer \(0\leq k\leq 2^{n+1}\). If \(k\) is even, we can write \(\frac{k}{2^{n+1}}=\frac{k'}{2^n}\) (where \(k=2k'\) ), and in this case the construction is already made. If \(k\) is odd, write it as \(k=2k'+1\). Note that \(0\leq k'< 2^n\). Then, by the construction made so far yields us the sets \(V\left(\frac{k'}{2^n}\right)\) and \(U\left(\frac{k'+1}{2^n}\right)\), which satisfy \(V\left(\frac{k'}{2^n}\right)\subseteq U\left(\frac{k'+1}{2^n}\right)\). Apply the characerisation of normality to these sets (the former is closed and the latter is open) to obtain two intermediate sets \(U'\) (open) and \(V'\) (closed). Define \(U\left(\frac{k}{2^{n+1}}\right)=U'\) and \(V\left(\frac{k}{2^{n+1}}\right)=V'\).

		Having constructed the last class of sets, define 
		\[
				f(\omega)=\left\{
				\begin{array}{ll}
						1, & \text{ if }\omega\not\in U(1),\\
				\inf\left\{r\in D'\left|\omega\in U(r)\right.\right\}, & \text{ otherwise}.
				\end{array}
				\right.
		\]

		It is clear that \(f=0\) on \(A\), \(f=1\) on \(B\) and that \(0\leq f\leq 1\). We need only to show that \(f\) is continuous. Note that, for any given \(r\in D'\), \(\omega\in U(r)\) implies \(f(\omega)\leq r\) (this is immediate); and \(\omega\in V(r)^c\) implies \(f(\omega)\geq r\). This last statement can be shown by contradiction: suppose that \(f(\omega)<r\). It is impossible that \(f(\omega)=1\), because \(r\leq 1\). Therefore, \(f(\omega)=\inf\left\{s\in D'\left|\omega\in U(s)\right.\right\}\), hence there exists some \(s\in D'\) such that \(s<r\) and \(\omega\in U(s)\). But \(U(s)\subseteq V(s)\subseteq U(r)\subseteq V(r)\), so that \(\omega\in V(r)\), a contradiction. To see continuitiy, take some \(\omega\in\Omega\) and \(\varepsilon>0\).

		If \(f(\omega)=0\), take some \(r\in D'\) such that \(r<\varepsilon\) and \(\omega\in U(r)\). Then, \(U(r)\) is an open neighbourhood of \(\omega\) such that \(f\left(U(r)\right)\subseteq[0,\varepsilon)\). If \(f(\omega)=1\), choose some \(r\in D\) such that \(1-\varepsilon<r\). Then, \(V(r)^c\) is an open neighbourhood of \(\omega\) such that \(f\left(V(r)^c\right)\subseteq (1-\varepsilon,1]\).	If \(0<f(\omega)<1\), choose some \(r,s\in D'\) so that \(f(\omega)-\varepsilon<r<f(\omega)<s<f(\omega)+\varepsilon\). Then, \(U(s)\setminus V(r)\) is an open neighbourhood of \(\omega\) such that \(f\left(U(s)\setminus V(r)\right)\subseteq\left(f(\omega)-\varepsilon,f(\omega)+\varepsilon\right)\)\footnote{A proof for this result was not included in \cite{ash1972real}. The version presented here is somewhat original: various ideas were taken from online forums.}.
\end{proof}
\begin{defn}
		The class of \emph{Baire sets} of ~\(\Omega\), denoted by \(\mathcal{A}(\Omega)\), is the smallest \(\sigma\)-field on \(\Omega\) making all continuous real-valued functions Borel measurable. Namely,
\[
\mathcal{A}(\Omega)=\sigma(C(\Omega))=\sigma\left(\left\{f^{-1}(B)\left|f\in C(\Omega) \text{ and }B \text{ is open in }\mathbb{R}\right.\right\}\right)
\]

A \(\bm{\sigma}\)\textbf{-closed} set is a countable union of closed sets (which need not be closed nor open) and a \(\bm{\sigma}\)\textbf{-open} set is a countable intersecion of open sets (which need not be closed nor open either).
\end{defn}

\begin{remk}\label{remark:Baire sets generator}
		It is immediate from the definition that \(\mathcal{A}(\Omega)\subseteq\mathscr{B}\left(\Omega\right)\): every \(f^{-1}(B)\), with \(f\) continuous and \(B\) open, is open; hence Borel. Then, apply \Cref{remark:generated structures}.

		We can obtain a smaller class of generators of the Baire sets by considering only bounded functions (we are claiming that \(\sigma\left(C_b(\Omega)\right)=\sigma\left(C(\Omega)\right)=\mathcal{A}(\Omega)\)): suppose that all bounded, continuous functions are measurable. If \(f\in C(\Omega)\), then for each \(n\in\mathbb{N}\), the function \(\max(f,n)\) is continuous: it is the composition of \(f\) and the continuous mapping \(x\mapsto\max(x,n)\). We can apply the result for \(n=0\) to obtain that \(f^{+}=\max(f,0)\) is continuous. Then, \(\max(f^{+},n)\in C_b(\Omega)\) is also continuous, hence measurable. Since \(\max(f^{+},n)\uparrow f^{+}\), it follows that \(f^{+}\) is measurable. A similar argument used with \(f^{-}=-\max(-f,0)\) yields that \(f^{-}\) is measurable. Finally, \(f=f^{+}-f^{-}\) is measurable.
\end{remk}
\begin{lemm}\label{lemma:Baire sets characterisation}
		Let \(\Omega\) be a normal topological space. Then \(\mathcal{A}(\Omega)\) is the smallest \(\sigma\)-field containing the \(\sigma\)-open sets that are also closed (or equally well, the \(\sigma\)-closed sets that are also open).
\end{lemm}
\begin{proof}
		Let \(\mathcal{H}\) be the \(\sigma\)-field generated by all open, \(\sigma\)-closed sets.
		Let \(f\in C(\Omega)\). Then,
		\[
				\{f>a\}=\bigcup_{n}\left\{f\geq a+\frac{1}{n}\right\}
		,\]
		hence \(\{f>a\}\) is an open, \(\sigma\)-closed set. It follows that \(f\) is \(\mathcal{H}\)-measurable. Since \(\mathcal{A}(\Omega)\) is the smallest \(\sigma\)-field making all such functions measurable (by \cref{remark:Baire sets generator}), it follows that \(\mathcal{A}(\Omega)\subseteq\mathcal{H}\). To see the other inclusion, let \(H=\bigcup_{n}F_n\) be an open, \(\sigma\)-closed set (\(F_n\) is a closed set for each \(n\)). Note that, for each \(n\), \(H^c\) and \(F_n\) are disjoint closed sets. Use \hyperref[Urysohn]{Urysohn's Lemma} to obtain a function \(f_n\) such that \(f_n=0\) on \(H^c\) and \(f_n=1\) on \(F_n\). Then, define the function \(f=\sum_{n} 2^{-n}f_n\). This series is uniformly convergent (the Weierstrass \(M\) test can be used to see this) and bounded by \(1\) since \(0\leq f_n\leq 1\), hence \(f\in C_b(\Omega)\) and \(f\geq 0\). Additionally, 
		\[
				\left\{f>0\right\}=\bigcup_{n}\left\{f_n>0\right\}=H
		.\]
		Therefore, \(H\in\mathcal{A}(\Omega)\), so that \(\mathcal{H}\subseteq\mathcal{A}(\Omega)\).
\end{proof}

We can examine the last argument to obtain another result:

\begin{corl}\label{corollary:Baire sets and functions}
		If \(\Omega\) is a normal topological space, the open, \(\sigma\)-closed sets are precisely the sets \(\left\{f>0\right\}\) with \(f\in C_b(\Omega)\), \(f\geq 0\).
\end{corl}
\begin{proof}
		In the first part of the proof, consider the special case \(f\in C_b(\Omega)\) and \(a=0\). Then, \(\left\{f>0\right\}\) is shown there to be an open, \(\sigma\)-closed set.

		In the reciprocal part the proof, we showed that if \(H\) is an open, \(\sigma\)-closed set, then \(H\) can be written as \(\left\{f>0\right\}\), where \(f\in C_b(\Omega)\) and \(f\geq 0\).
\end{proof}
\begin{lemm}
		Let \(A\) be an open, \(\sigma\)-closed set in the normal space \(\Omega\). Then, \(I_A\) is the limit of an increasing sequence of continuous functions.
\end{lemm}
\begin{proof}
		Let \(f=I_A\). Use \Cref{corollary:Baire sets and functions} to write \(A=\left\{f>0\right\}\), and define \(A_n=\left\{f\geq\frac{1}{n}\right\}\), so that \(A_n\uparrow A\). Use \hyperref[Urysohn]{Urysohn's Lemma} to obtain a sequence of functions \(0\leq f_n\leq 1\) such that \(f_n=0\) on \(A^c\) and \(f_n=1\) on \(A_n\). Define \(g_n=\max(f_1,\dots,f_n)\), so that \(0\leq g_n\leq 1\). The functions \(g_n\) are clearly continuous and form an increasing sequence.

		Furthermore, they satisfy \(I_{A_n}\leq g_n\leq I_A\): outside of \(A\), all \(f_n\) are \(0\), and thus so is \(g_n\). Inside of \(A_n\), \(f_n=1\), hence  \(g_n=1\). Taking limits, since  \(I_{A_n}\uparrow I_A\), it follows that \(g_n\uparrow I_A\) too.
\end{proof}

At this point, we are close to obtaining our approximation theorem, but first we need to be able to approximate by closed sets. The Daniell Theory provides useful tools for this purpose.

\begin{thrm}\label{theorem:basic approximation}
		Let \(P\) be any probability measure on \(\mathcal{A}(\Omega)\), where \(\Omega\) is a normal topological space. If \(A\in \mathcal{A}(\Omega)\), we have
		\begin{enumerate}
		\item\label{theorem:basic approximation 1}
 \(P(A)=\inf\left\{P(V)\left|A\subseteq V \text{ and } V \text{ is an open, }\sigma\text{-closed set}\right.\right\}\).
		\item\label{theorem:basic approximation 2}
 \(P(A)=\sup\left\{P(C)\left|C\subseteq A \text{ and } C \text{ is a closed, }\sigma\text{-open set}\right.\right\}\).
		\end{enumerate}
\end{thrm}
\begin{proof}
		Define \(L=C_b(\Omega)\), so that \(\mathcal{A}(\Omega)=\sigma(L)\), and \(E(f)=\int_{\Omega}f~dP\). It is clear that \(L\) is closed under the lattice operations and that \(E(f)=\int_{\Omega}f~dP\) exists and is finite because \(f\) is bounded and \(P\) is a finite measure. By the \hyperref[theorem:Extended Monotone Convergence]{Monotone Convergence Theorem}, \(E\) is a Daniell integral: if \(f_n\uparrow f\), \(f_n\geq 0\), then \(E(f_n)\uparrow E(f)\); and if \(f_n\downarrow 0\), then \(E(f_n)\downarrow E(0)=0\).

		Therefore, we can use \cref{theorem:approximation theorem in Daniell theory 2} to see that
		\[
				P(A)=\inf\left\{P(G)\left|A\subseteq G, G\in\mathcal{C}'\right.\right\}
		,\]
where \(\mathcal{C}'=\left\{G\subseteq\Omega\left|G=\left\{f>0\right\}, f\in L^{+}\right.\right\}\). By \Cref{corollary:Baire sets and functions}, \(\mathcal{C}'\) is exactly the class of all open, \(\sigma\)-closed sets. 

The proof for \ref{theorem:basic approximation 2} is obtaned by applying the result seen so far to \(A^c\) and taking into account that \(P(B^c)=1-P(B)\) for every \(B\in\mathcal{A}(\Omega)\).
\end{proof}

In metric spaces (which are always normal), we can obtain stronger results:

\begin{prop}\label{proposition:Baire and Borel sets coincide in metric spaces}
If ~\(\Omega\) is a metric space, then every closed set is a \(\sigma\)-open set. Therefore, \(\mathcal{A}(\Omega)=\mathscr{B}\left(\Omega\right)\).
\end{prop}
\begin{proof}
		Consider a closed subset \(F\subseteq\Omega\). Define the \emph{distance to F} function  \(\rho(\omega)=\text{dist}(\omega,F)=\inf_{f\in F}d(\omega,f)\). Then, \(\rho\) is continuous:

		Let \(f\) be an arbitrary point in \(F\). Let \(\omega_1,\omega_2\in\Omega\). Call \(\delta=d(\omega_1,\omega_2)\). Then,
		\[
				\rho(\omega_2)\leq d(\omega_2,f)\leq d(\omega_1,f)+\delta
		.\]
		Since \(f\) is arbitrary, then \(\rho(\omega_2)\leq\rho(\omega_1)+\delta\). Switching papers for \(\omega_1\) and \(\omega_2\), one obtains that \(\rho(\omega_1)\leq\rho(\omega_2)+\delta\). Therefore, if we take \(\delta=\varepsilon\), we conclude that \(\rho\) is continuous as mapping between metric spaces.

		From this, it follows that \(F\) is \(\sigma\)-open set:
		\[
				F=\left\{\omega\in\Omega\left|\rho(\omega)=0\right.\right\}=\bigcap_{n}\left\{\omega\in\Omega\left|\rho(\omega)<\frac{1}{n}\right.\right\}
		.\]
(in the first equality, we used that \(F\) is closed). The inclusion  \(\mathscr{B}\left(\Omega\right)\subseteq\mathcal{A}(\Omega)\) now follows from \Cref{lemma:Baire sets characterisation}. The reciprocal inclusion was already established in \Cref{remark:Baire sets generator}.
\end{proof}

From this last theorem we can, of course, deduce that in a metric space every open set is a \(\sigma\)-closed set. Combining these facts with \Cref{theorem:basic approximation}, we obtain a simpler statement:

\begin{corl}\label{corollary:basic approximation, metric}
		Let \(P\) be any probability measure on \(\mathcal{A}(\Omega)\), where \(\Omega\) is a metric space. If \(A\in \mathcal{A}(\Omega)\), we have
		\begin{enumerate}
		\item\label{corollary:basic approximation, metric 1}
 \(P(A)=\inf\left\{P(V)\left|A\subseteq V \text{ and } V \text{ is an open set}\right.\right\}\).
		\item\label{corollary:basic approximation, metric 2}
 \(P(A)=\sup\left\{P(C)\left|C\subseteq A \text{ and } C \text{ is a closed set}\right.\right\}\).
		\end{enumerate}
\end{corl}

Finally, we can state the desired result:
\begin{thrm}[Approximation by compact sets]\label{theorem:approximation by compact sets}
		Let \(\Omega\) be a complete, separable metric space. If \(P\) is a probability measure on \(\mathscr{B}\left(\Omega\right)\), then, for each \(A\in\mathscr{B}\left(\Omega\right)\),
\[
P(A)=\sup\left\{P(K)\left|K\text{ is a compact subset of }A\right.\right\}
.\]
\end{thrm}
\begin{proof}
		We will first show that, for every \(\varepsilon>0\), there exists some compact set \(K_{\varepsilon}\) such that \(P(K_\varepsilon)>1-\frac{\varepsilon}{2}\).

		Since \(\Omega\) is separable, there exists a sequence of points
		\(\omega_m\) that is dense in \(\Omega\). Consider some arbitrary
		radius \(r>0\), the open balls \(B_m(r)=B(\omega_m,r)\)
		and their closures \(\overline{B}(\omega_m,r)\). Then,
		\(\Omega=\bigcup_{m}\overline{B}(\omega_m,r)\) for any given radius
		\(r\). Write
		\(U_{nm}=\bigcup_{k=1}^m\overline{B}\left(\omega_k,\frac{1}{n}\right)\), so
		that, for every \(n\in\mathbb{Z}^{+}\), we have \(U_{nm}\uparrow_m
		\Omega\). It follows that there exists some \(m(n)\in\mathbb{Z}^{+}\)
		such that \(P\left(U_{nm(n)}\right)\geq
		P\left(\Omega\right)-\varepsilon 2^{-n-1}=1-\varepsilon 2^{-n-1}\). Now
		define \(K_{\varepsilon}=\bigcap_{n}U_{nm(n)}\).

		Firstly, note that
		\[
				P(K_{\varepsilon}^c)=P\left(\bigcup_{n}U_{nm(n)}^c\right)\leq\sum_{n} P\left(U_{nm(n)}^c\right)=\sum_{n} 1-P\left(U_{nm(n)}\right)\leq\sum_{n} \varepsilon 2^{-n-1}=\frac{\varepsilon}{2}
		.\]

		It remains to show that \(K_{\varepsilon}\) is compact. In a metric space, compactness and sequential compactness are equivalent. Therefore, it suffices to show that every sequence in \(K_{\varepsilon}\) has a subsequence converging to a point in \(K_{\varepsilon}\). Since \(K_{\varepsilon}\) is clearly closed (it is the intersection of closed sets), if we show that any sequence in \(K_{\varepsilon}\) has a converging subsequence, its limit will automatically be in \(K_{\varepsilon}\). One last simplification can be made: since \(\Omega\) is complete, it suffices to show that every sequence in \(K_{\varepsilon}\) has a Cauchy subsequence.

		Let \(\{x_p\}_{p\in\mathbb{Z}^{+}}\) be a sequence of points in \(K_{\varepsilon}\). First, note that, since \(x_p\in U_{1m(1)}=\bigcup_{k=1}^{m(1)}\overline{B}\left(\omega_k,1\right)\), there exists some \(k_1\leq m(1)\) such that infinitely many \(x_p\) are in \(\overline{B}(\omega_{k_1},1)\). Let \(T_1\) be the (infinite) set of all such indices. Similarly, \(x_p\in\bigcup_{k=1}^{m(2)}\overline{B}\left(\omega_k,\frac{1}{2}\right)\) for all \(p\in T_1\); hence there exists some \(k_2\leq m(2)\) such that infinitely many values \(x_p\), with \(p\in T_1\), are in \(\overline{B}\left(\omega_{k_2},\frac{1}{2}\right)\). Let \(T_2\) be the (infinite) set of all such indices. Continue inductively to obtain integers \(k_1,k_2,\dots\) and infinite sets of indices \(T_1\supseteq T_2\supseteq\dots\) so that, for every \(i\),
		\[
				x_p\in\bigcap_{j=1}^{i}\overline{B}\left(\omega_{k_j},\frac{1}{j}\right) ~\text{ for all }p\in T_i
		\]
		Choose one \(p_i\in T_i\) in a manner such that \(p_1<p_2<\dots\) (this is always possible because the sets \(T_i\) are infinite), and consider the subsequence \(x_{p_{1}}, x_{p_{2}},\dots\). Then, if \(l>j\), both \(x_{p_j}\) and \(x_{p_l}\) are in \(\overline{B}\left(\omega_{k_j},\frac{1}{j}\right)\); thus,
		\[
				d(x_{p_j},x_{p_l})\leq \frac{2}{j}\to 0 \text{ when }j\to+\infty
		.\]
		We have now obtained the desired compact set  \(K_{\varepsilon}\). To see the result we wanted, note that any compact subset \(K\) of \(A\) satisfies \(P(K)\leq P(A)\). It follows that
		\[
		P(A)\geq\sup\left\{P(K)\left|K\text{ is a compact subset of }A\right.\right\}
		.\]
		To see the equality, note that by \ref{corollary:basic approximation, metric 2}, the result holds when ``compact'' is replaced with ``closed''. Therefore, for every \(\varepsilon>0\), there exists some closed set \(C\subseteq A\) such that \(P(A)- P(C)\leq\frac{\varepsilon}{2}\). Take \(K=C\cap K_{\varepsilon}\).

		Note that \(P(C)-P(K)=P\left(C\setminus(C\cap K_{\varepsilon})\right)=P(C\setminus K_{\varepsilon}))\leq P(K_{\varepsilon}^c)<\frac{\varepsilon}{2}\). Therefore, it follows that \(K\) is a compact subset of \(A\) and
		\[
				P(A)-P(K)=P(A)-P(C)+P(C)-P(K)\leq \frac{\varepsilon}{2}+\frac{\varepsilon}{2}=\varepsilon
		,\]
		finishing the proof.
\end{proof}
To end the section, we offer a short lemma regarding separable metric spaces.
\begin{lemm}\label{lemma:separability and second countability}
		In metric spaces, separability and second countability are equivalent: that is, if ~\(\Omega\) is a metric space, then there exists a countable, dense subset \(S\subseteq X\) if, and only if, there exists a countable basis for \(X\)\footnote{This result is original.}.
\end{lemm}
\begin{proof}
		First suppose that \(X\) is separable, and write \(S=\left\{x_n\right\}_{n\in\mathbb{Z}^{+}}\). Define, for each \(n,m\in\mathbb{Z}^{+}\), the open set
		\[
				B_{nm}=B\left(x_n,\frac{1}{m}\right)
		.\]
		We will show that the set \(\left\{B_{nm}\right\}_{n,m\in\mathbb{Z}^{+}}\) forms a basis for the topology in \(X\). Let \(U\) be an open set, and define
		\[
				U'=\bigcup_{a,b}B_{ab}
		,\]
		where \(a\) and \(b\) range over the pairs of positive integers \((n,m)\) such that \(B_{nm}\subseteq U\). It is then clear that \(U'\subseteq U\). To see the other inclusion, take some \(z\in U\). Then, there exists some \(\varepsilon>0\) such that \(B(z,\varepsilon)\subseteq U\). By the Archimedian Property, there exists some \(m\in\mathbb{Z}^+\) such that \(\frac{1}{m}<\frac{\varepsilon}{2}\). Since \(S\) is dense in \(X\), there exists some \(x_n\) such that \(d(x_n,z)<\frac{1}{m}\). Finally, \(z\in B_{nm}\subseteq U\), and thus \(z\in U'\).

		For the reciprocal implication, consider some countable basis \(\left\{B_n\right\}_{n\in\mathbb{Z}^+}\) and choose some \(x_n\in B_n\) for each \(n\). Then, the set \(S=\left\{x_n\right\}_{n\in\mathbb{Z}^+}\) is trivially dense in \(X\).
\end{proof}

\include{Product_spaces}
%!TeX root=Final.tex

\chapter{AN APPLICATION TO REAL ANALYSIS AND PROBABILITY THEORY}\label{chapter:an application to real analysis and probability}

Throughout this text, we have developed some powerful tools in Measure Theory. Some of them will allow us to gain some insight into Real Analysis,
and others - specially those developed in the last chapter, regarding product spaces - will allow us to answer some questions in Probability Theory with respect to the existence of certain objects. 

\section{Some applications to Real Analysis}\label{section:Real Analysis}
So far, we have developed all of Measure Theory \emph{abstractly}, that is, without mentioning any concrete measures. In Analysis, however, we are mostly interested in \emph{classical} notions, such as areas and volumes.
The Riemann integral is, precisely, a formalisation and generalisation of the notion of area under a curve. Luckily, there is a concrete measure on \(\mathbb{R}^{n}\) that achieves this: the \textbf{Lebesgue measure}.

We will begin by working on \(\mathbb{R}\). Here, the most basic metric notion is that of the \emph{longitude} of a segment. If we want to construct a measure that represents this and is also able to give values to wide enough class of sets, we could define it on \(\mathscr{B}\left(\mathbb{R}\right)\) and it should assign its length to each interval. 

The approach followed here will allow us to construct a broad class of measures on \(\mathscr{B}\left(\mathbb{R}\right)\), which will be useful when working in Probability Theory. Mainly, we will study two concepts:

\begin{defn}\label{definition:distribution functions and Lebesgue-Stieltjes measures} A \textbf{Lebesgue-Stieltjes} measure on \(\mathbb{R}\) is a measure
\(\mu\) over the usual \(\sigma\)-field \(\mathscr{B}(\mathbb{R})\) such that
\(\mu(I)<\infty\) for every bounded interval \(I\). A \textbf{distribution
function} on \(\mathbb{R}\) is a mapping \(F\colon\mathbb{R}\to\mathbb{R}\) that is \textbf{increasing}
(\(a\leq b\) implies \(F(a)\leq F(b)\)) and \textbf{right-continuous} (that is,
\(\lim_{x\to x_0^{+}}F(x)=F(x_0)\)\footnote{Equivalently, \(\lim_nF(x_n)=F(x_0)\) for every sequence \(x_n\downarrow x_0\).}).
\end{defn}

There exists a close relation between the concepts defined above.
Namely, the formula \(\mu((a,b])=F(b)-F(a)\) yields a one-to-one correspondence
between Lebesgue-Stieltjes measures and distribution functions, up to an
additive constant. 

If provided with the measure \(\mu\), checking that \(F\) is a distribution function is very simple using the properties
of measures developed through \Cref{section:introductory results and definitions}.

\begin{prop} Let \(\mu\) be a Lebesgue-Stieltjes measure on \(\mathbb{R}\). Define \(F(0)\) as any real number. Then, the function \(F\colon\mathbb{R}\to\mathbb{R}\) defined as
	\[F(x)= \left\{
	\begin{array}{rl} F(0)+\mu(0,x],& \text{ if
} x\geq0\\ F(0)-\mu(x,0],& \text{ if } x<0
	\end{array}.  \right.
	\] is a distribution function.
\end{prop}

Reciprocally, if provided with the distribution function \(F\), constructing the associated Lebesgue-Stieltjes measure is slightly harder. A sketch of the proof is given below:

\begin{thrm}\label{theorem:Lebesgue-Stieltjes}
		Let \(F\colon\mathbb{R}\to\mathbb{R}\) be a distribution function. Then, there exists a unique measure on \(\mathscr{B}\left(\mathbb{R}\right)\) that satisfies the formula \(\mu\left( (a,b]\right)=F(b)-F(a)\), \(a<b\in\mathbb{R}\). Moreover, this is a Lebesgue-Stieltjes measure.
\end{thrm}
\begin{proof}
The idea is to find a suitable \(\sigma\)-field to use the \hyperref[theorem:Caratheodory Extension]{Carathéodory Extension Theorem}. In \(\overline{\mathbb{R}}\), the class of disjount unions of right-semiclosed intervals, \(\mathcal{F}_0(\overline{\mathbb{R}})\), forms a field. Also, since \(\overline{\mathbb{R}}\) is compact (one way to see this is by noting that it is homeomorphic to \(\left[-\frac{\pi}{2},\frac{\pi}{2}\right]\) via \(\arctan(x)\)), it will be more convenient to work in \(\overline{\mathbb{R}}\) than in \(\mathbb{R}\). Extend \(F\) to \(\overline{\mathbb{R}}\) as \(F(-\infty)=\lim_{x\to-\infty}\)and \(F(+\infty)=\lim_{x\to+\infty}F(x)\) (both limits exist by monotonicity).

Now \(\mu\) is defined on right-semiclosed intervals as \(\mu\left( (a.b]\right)=F(b)-F(a)\) (where \(a\leq b\in\overline{\mathbb{R}}\)) and \(\mu([-\infty,a])=F(a)-F(-\infty)\). Then \(\mu\) is extended to \(\mathcal{F}_0(\overline{\mathbb{R}})\) by additivity. It is now shown that \(\mu\) is countably additive by using the compactness of \(\overline{\mathbb{R}}\) and \Cref{proposition:sigma-additivity from below} (this is Lemma 1.4.3 of \cite{ash1972real}):

Let \(A_1,A_2,\dots\) be a sequence of sets in \(\mathcal{F}_0(\overline{\mathbb{R}})\) decreasing to \(\emptyset\). Note that each set \(A_n\) is the union of a finite number of disjoint right-semiclosed intervals, and \(\mu\left(a',b\right]\to\mu\left(a,b\right]\) when \(a'\to a^+\) for each \(a\leq b\in\overline{\mathbb{R}}\). Then, for every \(\varepsilon>0\), it is possible to find a set \(B_n\) such that \(\overline{B}_n\subseteq A_n\) and \(\mu\left(A_n\setminus\overline{B}_n\right)<\varepsilon 2^{-n}
\). Note that
\[
		\bigcap_{n}\overline{B}_n\subseteq\bigcap_{n}A_n=\emptyset
,\]
hence the sets \(\overline{B}_n^c\) form an open covering of the compact \(\overline{\mathbb{R}}\). Therefore, there exists a finite collection \(\overline{B}_{n_{1}}, \dots , \overline{B}_{n_{k}}\) such that \(\bigcap_{i=1}^k\overline{B}_{n_i}=\emptyset\). Take \(n_0=\max(n_1,\dots,n_k)\). Therefore, for each \(n\geq n_0\), we have \(A_n\subseteq A_{n_i}\) for every \(i=1,\dots,k\). Thus,
\[
		\mu(A_n)=\mu\left(A_n\setminus\left(\bigcup_{i=1}^k\overline{B}_{n_i}\right)\right)+\mu\left(\bigcap_{i=1}^k\overline{B}_{n_i}\right)\leq\mu\left(\bigcup_{i=1}^k\left(A_n\setminus\overline{B}_{n_i}\right)\right)+0\leq\mu\left(\bigcup_{i=1}^k\left(A_{n_i}\setminus\overline{B}_{n_i}\right)\right)<\varepsilon
,\]
where in the last step we used that \(\mu\left(\bigcup_{n}C_n\right)\leq\sum_{n} \mu(C_n)\) for every sequence of measurable sets \(C_n\) whose union is measurable too. 

If \(\mu\) were \(\sigma\)-finite on \(\mathcal{F}_0(\overline{\mathbb{R}})\), we could extend it to a \(\sigma\)-field, but this need not be the case in general. This may be solved by restricting \(\mu\) to \(\mathcal{F}_0(\mathbb{R})\), the field of disjoint unions of right-semiclosed intervals (counting \((a,+\infty)\) as right-semiclosed): here, we may consider the sets \((-n,n]\), which cover \(\mathbb{R}\) and on which \(\mu\) is clearly finite.

We have defined \(\mu\) on the desired field and checked the necessary hypotheses, and hence it extends to a unique measure on \(\sigma(\mathcal{F}(\mathbb{R}))=\mathscr{B}\left(\mathbb{R}\right)\). This extension is clearly a Lebesgue-Stieltjes measure because \(\mu\left( (a,b]\right)=F(b)-F(a)\) and \(F\) takes finite values.
\end{proof}


With this theorem, it is easy to construct the Lebesgue measure on \(\mathscr{B}\left(\mathbb{R}\right)\): simply take \(F(x)=x\); then, the corresponding Lebesgue-Stieltjes measure, \(m\), is the unique measure on \(\mathscr{B}\left(\mathbb{R}\right)\) assigning its length to each interval.

We can extend this measure to \(\mathbb{R}^{n}\) with the theory developed so far: note that, by \Cref{lemma:Borel sets in separable spaces}, \(\mathscr{B}\left(\mathbb{R}^{n}\right)=\left(\mathscr{B}\left(\mathbb{R}\right)\right)^n\). Now take each \(\mu_k=m\) on \Cref{corollary:classical Product Measure} to obtain the unique measure on \(\mathscr{B}\left(\mathbb{R}^{n}\right)\) assigning its volume to each rectangle. Finally, the Lebesgue measure on \(\mathbb{R}^{n}\) is defined as the completion of this measure, in the sense of \Cref{definition:completion of a measure space}.

This measure allows us to revisit Real Analysis from a measure-theoretic standpoint. So much so, that (proper) Riemann integration can be regarded as a particular case or Lebesgue integration, in the sense of the following theorem:

\begin{thrm}\label{theorem:Riemann and Lebesgue} Let \(f\) be a bounded, real-valued function defined on a closed rectangle \(R\subseteq\mathbb{R}^n\). Then,
	\begin{enumerate}
		\item \(f\) is Riemann integrable on \(R\) if, and only if, it is
continuous almost everywhere on \(R\) with respect to the Lebesgue measure on \(\mathbb{R}^{n}\).
		\item If \(f\) is Riemann integrable on \(R\), then it is integrable on \(R\)
with respect to the Lebesgue measure on \(\mathbb{R}^{n}\) and the two integrals are
equal.
	\end{enumerate}
\end{thrm}

This result will be of little interest in this work, and its proof is not
included due to space limitations. The interested reader can consult Section 1.7 of \cite{ash1972real}.

Starting here, the remaining part of the text has no longer been based on \cite{ash1972real} and is mostly original instead.

More applications to Real Analysis are possible with this approach: Riemann-theoretic Fubini's Theorem is almost immediate from \Cref{corollary:classical Fubini's}. Another classical theorem regarding integration is proved in \Cref{chapter:A Brief Application to Analysis}.


\section{Existence of random variables and random samples}

In Probability Theory, a \textbf{random variable} is defined as a (Borel) \(\mathcal{F}\)-measurable function \(X\colon\Omega\to\mathbb{R}\), where \(\left(\Omega,\mathcal{F},P\right)\) is some probability space.
The \textbf{distribution function} of \(X\) is then defined as the function \(F_X\colon \mathbb{R}\to [0,1] \) given by \(F_X(x)=P(\{X\leq x\})\).

These two concepts are central in this theory, and the latter is always associated with the former. Usually, however, we are not interested in the random variable itself, but rather in its distribution function; in most introductory probability and statistics courses, random variables are introduced \emph{via their distributions}, and not the other way around.
Consider, for example, a normal distribution: we say that a given random variable \(X\) follows a \textbf{normal distribution} with mean \(\mu\) and variance \(\sigma^2\) if
\[
		F_X(x)=\frac{1}{\sigma\sqrt{2\pi}}\int_{-\infty}^x e^{-\frac{1}{2}\left(\frac{\mu-t}{\sigma}\right)^2}~dt
.\]

These random variables are widely used in mathematics, but it is not immediate that at least one of them exists: how do we know that there exist a probability space \(\left(\Omega,\mathcal{F},P\right)\) and random variable \(X\) on it such that \(F_X\) has the form specified above?

With the theory developed in this text, are able to construct any such function. Moreover, we will be able to guarantee that the probability space satisfies \(\Omega=\mathbb{R}\) and \(\mathcal{F}=\mathscr{B}\left(\mathbb{R}\right)\).
\begin{defn}
		Consider a mapping \(F\colon \mathbb{R}\to \mathbb{R}\). We will say that \(F\) is a \textbf{probabilistic distribution function} if \(F\) is a distribution function in the sense of \Cref{definition:distribution functions and Lebesgue-Stieltjes measures} that ranges between \(0\) and \(1\); namely, if
		\begin{enumerate}
				\item \(F\) is an increasing function.
				\item \(F\) is right-continuous; that is, if \(x_n\downarrow x\), then \(F(x_n)\to F(x)\) for every \(x\in\mathbb{R}\).
				\item \(\lim_{x\to-\infty}F(x)=0\) and \(\lim_{x\to+\infty}F(x)=1\).
		\end{enumerate}
\end{defn}
\begin{prop}
		Let \(F\colon \mathbb{R}\to \mathbb{R} \) be a probabilistic distribution function. Then, there exist a probability measure \(P\) on \(\mathscr{B}\left(\mathbb{R}\right)\) and a random variable \(X\colon \mathbb{R}\to \mathbb{R} \) such that \(F_X=F\).
\end{prop}
\begin{proof}
		By \Cref{theorem:Lebesgue-Stieltjes}, there exists a measure \(P\) on \(\mathscr{B}\left(\mathbb{R}\right)\) such that \(F(x)=P\left( (-\infty,x]\right)\) for all \(x\in\mathbb{R}\). To see that \(P\) is, in fact, a probability measure, take any sequence \(x_n\uparrow +\infty\) (\(x_n=n\) will do). Then,
\[
		P(\mathbb{R})=\lim_nP\left( (-\infty,x_n]\right)=\lim_nF(x_n)=1
.\]
		Now simply define \(X=id\). It is clear that \(X\) is measurable, and trivially
		\[
				F_X(x)=P\left(\left\{X\leq x\right\}\right)=P\left( (-\infty,x]\right)=F(x)
		.\]
\end{proof}

A similar question arises when working with random samples, but the solution now is more complicated. Before proceeding, we remind the reader of some basic concepts related to this topic:

Let \(\left\{X_t\right\}_{t\in T}\) be a family of random variables defined on the same probability space \(\left(\Omega,\mathcal{F},P\right)\). We say that this family is \textbf{independent} if, for any finite set of those random variables, \(X_{t_1},\dots,X_{t_n}\) and any measurable sets \(B_{t_1},\dots,B_{t_n}\), \(B_{t_k}\in\mathscr{B}\left(\mathbb{R}\right)\), we have
\[
P\left(\left\{X_{t_1}\in B_{t_1}\right\}\cap\dotsc\cap\left\{X_{t_n}\in B_{t_n}\right\}\right)=P\left(\left\{X_{t_1}\in B_{t_1}\right\}\right)\cdot\dotsc\cdot P\left(\left\{X_{t_n}\in B_{t_n}\right\}\right)
,\]
where \(\left\{X_{t_i}\in B_{t_i}\right\}\) denotes the (measurable) set \(\left\{\omega\in\Omega\left|X_{t_i}(\omega)\in B_{t_i}\right.\right\}=X_{t_i}^{-1}(B_{t_i})\). We say that the family is \textbf{identically distributed} if \(F_{X_t}=F_{X_l}\) for any two \(t,l\in T\). 

Following this notation, a \textbf{random sample} with a given distribution \(F\) is a countable, independent and identically distributed (with distribution \(F\)) family of random variables defined on the same probability space.

The concept of random sample is widely used in statistics, being at the core of many theorems and concepts. But, again, it is not trivial that there exists a random sample with a given distribution (for instance, a Bernoulli). The \hyperref[theorem:Kolmogorov Extension]{Kolmogorov Extension Theorem} allows us to construct this, and even generalise it to the case where we have an arbitrary amount of random variables and each random variable has a different distribution.

\begin{thrm}\label{theorem:existence of stochastic}
		Let \(T\) be an index set. Suppose that, for every given \(t\in T\), we are given a probabilistic distribution function \(F_t\). Then, there exists a probability space \(\Omega\) and an independent family of random variables \(\left\{X_t\right\}_{t\in T}\) (each defined on \(\Omega\)) such that 
		\(F_{X_t}=F_t\).
\end{thrm}
\begin{proof}
		For every \(t\in T\), let \(P_t\) be the Lebesgue-Stieltjes measure associated to \(F_t\). It is clear that \(P_t\) is a probability measure on \(\mathbb{R}\).

		Identify \(\Omega_t=\mathbb{R}\) and \(\mathcal{F}_t=\mathscr{B}\left(\mathbb{R}\right)\) for every \(t\) - so that each \(\Omega_t\) is a separable, complete metric space. Also, for each finite \(v\subseteq T\), write \(v=\left\{t_1,\dots,t_n\right\}\), with \(t_1<\dots<t_n\). Then, \(\Omega_v=\mathbb{R}^{n}\) and \(\mathcal{F}_v=\left(\mathscr{B}\left(\mathbb{R}\right)\right)^{n}\) for each finite \(v\subseteq T\). Use \Cref{corollary:classical Product Measure} to obtain a probability measure \(P_v\) on \(\mathcal{F}_v\) such that
		\[
				P_v\left(A_{t_{1}}\times \dots \times A_{t_{n}}\right)=P_{t_1}(A_{t_1})\cdot\dotsc\cdot P_{t_n}(A_{t_n})
		,\]
		for each finite family of sets \(A_{t_k}\in\mathcal{F}_{t_k}\).

		It is clear that the family of probability measures defined in this way is consistent: if \(v\subseteq w\), we can suppose, for simplicity, that \(w=v\cup\left\{t_{n+1}\right\}\), with \(t_{n+1}>t_n\) (the ``complete'' result is proved very similarly, but with more cumbersome notation). Define the probability measure \(P'\) on \(\mathcal{F}_{v}\) as \(P'(B)=P_{w}(B^w)\). Note that \(B^w=B\times\Omega_{t_{n+1say,}}\). Then, for every measurable rectangle \(B=A_{t_{1}}\times \dots \times A_{t_{n}}\), we have
		\[
				P'(B)=P_w(A_{t_{1}}\times \dots \times A_{t_{n}}\times\Omega_{t_{n+1}})=P_{t_1}(A_{t_{1}})\cdot\dotsc\cdot P_{t_n}(A_{t_{n}})\cdot P_{t_{n+1}}(\Omega_{t_{n+1}})=P_v(B)
		.\]
		Since \(P'\) and \(P_v\) agree on measurable rectangles, by the uniqueness part of \Cref{corollary:classical Product Measure} we have \(P'=P_v\). Therefore, \(P_v(B)=P_w(B^w)\) for every \(B\in\mathcal{F}_v\).

		Now use the \hyperref[theorem:Kolmogorov Extension]{Kolmogorov Extension Theorem} to construct the hoped-for probability space \(\left(\Omega,\mathcal{F},P\right)\), where \(\Omega=\prod_{t}\Omega_t\) and \(\mathcal{F}=\prod_{t}\mathcal{F}_t\).

		For each fixed \(t\in T\), define \(X_t(\omega)=\omega_{\left\{t\right\}}\). It is clear that \(X_t\) is a measurable function, since \(X_t^{-1}(B)=B_{\left\{t\right\}}\) is a measurable cylinder for each \(B\in\mathscr{B}\left(\mathbb{R}\right)=\mathcal{F}_t\). Moreover, \(F_{X_t}=F_t\), because
		\[
				P\left(\left\{X_t\leq x\right\}\right)=P\left( (-\infty,x]_{\left\{t\right\}}\right)=P_{\left\{t\right\}}\left( (-\infty,x]\right)=P_{t}\left( (-\infty,x]\right)=F_t(x)
		.\]
		Finally, the family of random variables \(\left\{X_t\right\}_{t\in T}\) is independent: consider finitely many random variables \(X_{t_1},\dots,X_{t_n}\), and suppose they are ordered (\(t_1<\dots<t_n\)). Write \(v=\left\{t_1,\dots,t_n\right\}\). Let \(B_{1},\dots,B_{n}\in\mathscr{B}\left(\mathbb{R}\right)\). Now note that, for each \(k=1,\dots,n\), we can regard \(B_k\in\mathscr{B}\left(\mathbb{R}\right)=\mathcal{F}_{t_k}=\mathcal{F}_{\left\{t_k\right\}}\). Therefore, we can consider its retraction to \(\Omega_v\):
		\[
				\left(B_k\right)^v=\Omega_{t_1}\times\dots\times\Omega_{t_{k-1}}\times B_k\times\Omega_{t_{k+1}}\times\dots\times\Omega_{t_n}
		.\]
		Now write, for each \(k\), \(A_k=\left\{X_{t_k}\in B_{k}\right\}=\left(B_{k}\right)_{\left\{t_k\right\}}\in\mathcal{F}\). Note that \(P(A_k)=P_{\left\{t_k\right\}}(B_k)=P_{t_k}(B_k)\). We can also regard \(A_k\) as having a higher base, common for all \(k\):
		\[
				A_k=\left(B_k\right)_{\left\{t_k\right\}}=\left( \left(B_k\right)^v\right)_v=\left(\Omega_{t_1}\times\dots\times\Omega_{t_{k-1}}\times B_k\times\Omega_{t_{k+1}}\times\dots\times\Omega_{t_n}\right)_v
		.\]
		It follows from \Cref{remark:algebraic propertis of cylinders} that \(A_1\cap\dots\cap A_n=\left( \left(B_1\right)^v\cap\dots\cap \left(B_n\right)^v\right)_v=\left(B_1\times\dots\times B_n\right)_v\).
		Therefore,
		\[
				P\left(A_1\cap\dots\cap A_n\right)=P_v(B_1\times\dots\times B_n)=P_{t_1}(B_1)\cdot\dotsc\cdot P_{t_n}(B_n)=P(A_1)\cdot\dotsc\cdot P(A_n)
		.\]
		Recall that each \(A_k=\left\{X_{t_k}\in B_k\right\}\), so this is the condition of independency.
\end{proof}
\begin{corl}
		Let \(F\) be a probabilistic distribution function. Then, there exists a random sample of \(F\).
\end{corl}
\begin{proof}
		In \cref{theorem:existence of stochastic}, take \(T=\mathbb{Z}^+\) and \(F_n=F\) for each \(n\in\mathbb{Z}^+\).
\end{proof}
\section*{Conclusions}\label{section:Conclusions}

In this work, we have developed many tools in measure theory which have allowed us to obtain results in real analysis and probability theory. In \Cref{chapter:elementary measure theory}, we introduced most of the necessary concepts regarding measure theory and integration. In \Cref{chapter:advanced results in measure theory}, we develped strong tools relating this theory to function spaces and topology. In \Cref{chapter:product spaces}, we studied measures in higher dimensions, and this led us to the \hyperref[theorem:Kolmogorov Extension]{Kolmogorov Extension Theorem}. Finally, in \Cref{chapter:an application to real analysis and probability}, we used the theory developed so far to gain some insight into real analysis and probability theory.


%%%%%%%% Fi cos del treball %%%%%%%%%%%
%
%% Si el vostre document no conté apèndixs 
%% comentau les dues línies següents
\appendix
%!TeX root=Final.tex

\chapter{Lemmas on switching limits}\label{chapter:Lemmas on switching limits}
\setcounter{section}{1}

\begin{lemm}\label{lemma:switching limits of increasing requences of real
numbers} Let \(a_{nm}\) be a sequence of real numbers that is increasing with
respect to both indices; that is, \(n\geq n'\) implies
\(\forall m\colon a_{nm}\geq a_{n'm}\), and \(m\geq m'\) implies
\(\forall n\colon a_{nm}\geq a_{nm'}\). Then,
	\[ \lim_{n}\lim_{m}a_{nm}=\lim_{m}\lim_{n}a_{nm}
	\]
\end{lemm}
\begin{proof} Since the sequence is increasing with respect to both indices, the
result is equivalent to
	\[ \sup_{n}\sup_{m}a_{nm}=\sup_{m}\sup_{n}a_{nm}
	\] We will first show that \(\sup_{n}\sup_{m}a_{nm}=\sup_{n,m}a_{nm}\). For
every \(n\), define \(k_{n}=\sup_{m}a_{nm}\). Also set \(k=\sup_{n,m}a_{nm}\).
It is clear that \(\forall n\colon k_{n}\leq k\), and thus
\(\sup_{n}k_{n}\leq k\). On the other side,
\(\forall n \forall m\colon k_{n}\geq a_{nm}\). Therefore, taking suprema with
respect to \(n,m\), we have \(\sup_{n,m}k_{n}\geq k\). But since \(k_{n}\) does
not depend on \(m\), \(\sup_{n,m}k_{n}=\sup_{n}k_{n}\). Thus,
\(\sup_{n}\sup_{m}a_{nm}=\sup_{n,m}a_{nm}\).
	
	Define \(b_{nm}=a_{mn}\). The sequence \(b_{nm}\) is also increasing with
respect to both indices. Hence,
	\[ \sup_{n}\sup_{m}a_{nm}=\sup_{n,m}a_{nm}=\sup_{n,m}b_{nm}=\sup_{n}\sup_{m}a_{mn}=\sup_{m}\sup_{n}a_{nm}
	\]
\end{proof}
\begin{corl}\label{corollary:switching double series} Let \(a_{ij}\) be a
sequence of nonnegative real numbers, where \(i,j\in\mathbb{Z}^+\). Then,
	\[\sum_i\sum_j a_{ij}=\sum_j\sum_i a_{ij},\] whether the expression is
finite or not.
\end{corl}
\begin{proof} Define \(S_{nm}=\sum_{i=1}^{n}\sum_{j=1}^{m}a_{ij}\). Then, the
expression \(\sum_i\sum_j a_{ij}\) corresponds to \(\lim_{n}\lim_{m}S_{nm}\),
and the expression \(\sum_j\sum_i a_{ij}\) corresponds to
\(\lim_{m}\lim_{n}S_{nm}\). Since \(\forall i\forall j\colon a_{ij}\geq 0\), the
sequence \(S_{nm}\) is increasing with respect to both indices. Applying
\cref{lemma:switching limits of increasing requences of real numbers} yields the
desired result.
\end{proof}
 
%!TeX root=Final.tex
\chapter{RELATIONS BEWTEEN MEASURES}\label{chapter:relations between measures}

In \Cref{chapter:elementary measure theory} we studied
measures mostly \textit{intrinsically}: we constructed measure spaces and
studied the properties of functions on fixed measure spaces. However, we did not
study measures \textit{extrinsically}, that is, how different measure spaces
relate to each other. This is one of the main goals of this chapter.

\section{Jordan-Hahn Decomposition Theorem}\label{section:Jordan-Hahn Decomposition}
In this section we will develop a
way of systematically studying \(\sigma\)-additive set functions defined on
\(\sigma\)-fields. This class of functions is interesting because it contains
all set functions of the kind \(\lambda(A)=\int_{A}f~d\mu\) for some Borel
measurable function \(f\) such that \(\int_{\Omega}f~d\mu\) exists. Every such function can be
written as the difference of two measures \(\lambda=\lambda^+-\lambda^-\):
simply take \(\lambda^+(A)=\int_{A}f^+~d\mu\) and
\(\lambda^-(A)=\int_{A}f^-~d\mu\).

The Jordan-Hahn Decomposition Theorem states that the same is true for the
entire class: any \(\sigma\)-additive set function on a \(\sigma\)-field can be
expressed as the difference of two measures. Before stating it, we need some to
develop one result that is very useful by itself, and will allow us to prove the
main theorem:

\begin{thrm}\label{theorem:sigma-additive function assumes extrema} Let
\(\lambda\) be a countably additive extended real-valued set function defined on
a \(\sigma\)-field \(\mathcal{F}\) of subsets of \(\Omega\). Then, \(\lambda\) assumes
its maximum and minimum, that is, there exist \(C,D\in\mathcal{F}\) such that
	\[\lambda(C)=\sup\left\{\lambda(A)\colon A\in\mathcal{F}\right\} \text{ and
} \lambda(D)=\inf\left\{\lambda(A)\colon A\in\mathcal{F}\right\}\]
\end{thrm}

\begin{proof} We will first consider the supremum. We may assume that
\(\lambda(A)<+\infty\) for every \(A\in\mathcal{F}\), for if \(\lambda(A_0)=\infty\), we
take \(C=A_0\). Take a sequence of sets \(A_n\in\mathcal{F}\) such that
\(\lambda(A_n)\to\sup\lambda\), and set \(A=\bigcup_nA_n\).
	
	The idea of the proof is to ``approximate'' \(A\) by finitely many subsets
\(A_1,\dots,A_n\) and take only the positive bits created by those subsets. More
concretely, consider some fixed \(n\in\mathbb{Z}^+\) and note that, for each \(k\leq n\)
and for each \(a\in A\), either \(a\in A_k\) or \(a\in A_k^c\). Thus, we may
write \(A=\bigcup_{m=1}^{2^n}A_{nm}\), where the sets \(A_{nm}\) comprise all
the different \(2^n\) sets of the form \(A_1^*\cap\dots\cap A_n^*\), where
\(A_k^*\) is either \(A_k\) or \(A_k^c\). From this construction, we can observe
that:
	
	\begin{enumerate}
		\item \label{theorem:sigma-additive function assumes extrema 1} The
sets \(A_{nm}\) are disjoint.
		\item \label{theorem:sigma-additive function assumes extrema 2} For
every \(k\leq n\), \(A_k\) is the finite union of some of the sets \(A_{nm}\).
		\item \label{theorem:sigma-additive function assumes extrema 3} If
\(n'>n\), each \(A_{nm}\) is a subset of some \(A_{n'm'}\).
	\end{enumerate}
	
	Define \(\displaystyle B_n\) as the union of all sets \(A_{nm}\) such that
\(\lambda(A_{nm})\geq0\). From \ref{theorem:sigma-additive function assumes
extrema 2}, it follows that \(\lambda(A_n)\leq\lambda(B_n)\) and from
\ref{theorem:sigma-additive function assumes extrema 3} we have that, for
\(k>n\), \(\displaystyle\bigcup_{k=n}^rB_k\) can be written as the disjoint
union of \(B_n\) and some sets \(A_{n'm'}\). Thus,
\(\lambda(\bigcup_{k=n}^{r}B_k)\geq\lambda(B_n)\). Therefore, we have
	\[ \sup\lambda=\lim_n\lambda(A_n)\leq\lim_n\lambda(B_n)\leq\lambda\left(\bigcup_{k=n}^{r}B_n\right)\uparrow_{r}\lambda\left(\bigcup_{k=n}^{+\infty}B_k\right).
	\]
	
	If we set \(C=\limsup_nB_n\), since
\(\lambda\left(\bigcup_{k=n}^{+\infty}B_k\right)\downarrow\lambda(C)\), we have
\(\sup\lambda\leq\lambda(C)\leq\sup\lambda\).
	
	The set \(D\) is obtained by applying the result proved so far to
\(-\lambda\).
\end{proof}

We now have all tools required to develop our theorem. Without further delay:

\begin{thrm}[Jordan-Hahn Decomposition Theorem]\label{theorem:Jordan-Hahn
Decomposition} Let \(\lambda\) be a countably additive set function defined on a
\(\sigma\)-field \(\mathcal{F}\). Define the set functions
	\[ \lambda^+(B)=\sup\{\lambda(A)\colon A\subseteq B, A\in\mathcal{F}\} \text{ and
} \lambda^-(B)=-\inf\{\lambda(A)\colon A\subseteq B, A\in\mathcal{F}\}.
	\]
	
	
	Then, \(\lambda^+\) and \(\lambda^-\) are measures on \(\mathcal{F}\) and
\(\lambda=\lambda^+-\lambda^-\).
\end{thrm}
\begin{proof} First, suppose, without loss of generality, that
\(\lambda<+\infty\) (by definition of countably additive set function,
\(+\infty\) and \(-\infty\) cannot both be in the range of \(\lambda\), and if
\(\lambda>-\infty\), we can apply the result to \(-\lambda\)).  Let \(C\in\mathcal{F}\)
be a set such that, for all \(A\in\mathcal{F}\),
	\[ \lambda(A\cap C)\geq0 \text{ and } \lambda(A\cap C^c)\leq0.
	\]
	
	Such a set exists because of \cref{theorem:sigma-additive function assumes
extrema}: take \(C\in\mathcal{F}\) satisfying \(\lambda(C)=\sup\lambda\). Then, if
\(\lambda(A\cap C)<0\), we have
\(\sup\lambda=\lambda(C)=\lambda(C\setminus (C\cap A))+\lambda(A\cap C)<\lambda(C\setminus (C\cap A))\);
and if \(\lambda(A\cap C^c)>0\), then
\(\lambda(C\cup (A\cap C^c))=\lambda(C)+\lambda(A\cap C^c)>\lambda(C)=\sup\lambda\).
	
	From this, we will see that \(\lambda^+(A)=\lambda(A\cap C)\) and
\(-\lambda^-(A)=\lambda(A\cap C^c)\):
	
	Inequalities
\(\lambda(A\cap C)\leq\lambda^+(A), \lambda(A\cap C^c)\geq\lambda^-(A)\) are
immediate from the definitions of \(\lambda^+\) and \(\lambda^-\). To see the
equality, note that, if \(B\subseteq A, B\in\mathcal{F}\),
	\[ \lambda(B)=\lambda(B\cap C)+\lambda(B\cap C^c)\leq\lambda(B\cap  C)\leq\lambda(B\cap C)+\lambda((A\setminus B)\cap C)=\lambda(A\cap C).
	\] Thus, \(\lambda^+(A)\leq\lambda(A\cap C)\). Similarly,
\(-\lambda^-(A)\geq \lambda(A\cap C^c)\).
	
	From here, it clearly follows that both \(\lambda^+\) and \(\lambda^-\) are
measures and that \(\lambda=\lambda^+-\lambda^-\).
\end{proof} Before finishing the section, we can extract a few additional
results.
\begin{corl} Let \(\lambda\) be a countably additive extended real-valued
set function on the \(\sigma\)-field \(\mathcal{F}\). Then,
	\begin{enumerate}
		\item \(\lambda\) is the difference of two measures, at least one of
which is finite.
		\item If \(\lambda\) is finite, then it is bounded.
		\item \label{corollary:positive set of a signed measure} There is a set
\(C\in\mathcal{F}\) such that \(\lambda(A\cap C)\geq0\) and \(\lambda(A\cap C^c)\leq0\)
for all \(A\in\mathcal{F}\).
		\item If \(E\in\mathcal{F}\) is another set satisfying that
\(\lambda(A\cap E)\geq0\) and \(\lambda(A\cap E^c)\leq0\) for all \(A\in\mathcal{F}\),
then \(\lambda^+(A)=\lambda(A\cap E)\) and \(\lambda^-(A)=\lambda(A\cap E^c)\)
for all \(A\in\mathcal{F}\).
		\item \label{corollary:positive set of a signed measure is its positive
part}If \(E\) is one such set, then \(\left|\lambda\right|(C\triangle E)=0\),
where \(\left|\lambda\right|=\lambda^++\lambda^-\).
	\end{enumerate}
\end{corl}
\begin{proof}
	\begin{enumerate}
		\item If \(\lambda<+\infty\), then \(\lambda^+<+\infty\), and if
\(\lambda>-\infty\), then \(\lambda^-<+\infty\).
		\item Consequence of the previous section and the fact that every finite
measure is bounded.
		\item Consequence of \Cref{theorem:sigma-additive function assumes
extrema}, as shown in the proof of \Cref{theorem:Jordan-Hahn Decomposition}.
		\item The set \(C\) in the proof of \ref{theorem:Jordan-Hahn
Decomposition} was imposed no other hypotheses except that \(C\in\mathcal{F}\) and
\(\lambda(A\cap C)\geq0, \lambda(A\cap C^c)\leq0\). Thus, any such set \(E\)
will satisfy \(\lambda^+(A)=\lambda(A\cap E), \lambda^-(A)=\lambda(A\cap E^c)\).
		\item First note that by the property satisfied by \(C\) and \(E\), we
have \(0\leq\lambda(C\cap E^c)\leq0\), and thus, by the previous section we have
\(\lambda^+(E^c)=\lambda^-(C)=0\).  Therefore,
\(0\leq\lambda^+(C\cap E^c)\leq\lambda^+(E^c)=0\), and
\(0\leq\lambda^-(C\cap E^c)\leq\lambda^-(C)=0\).
	\end{enumerate}
\end{proof}

The \href{theorem:Jordan-Hahn Decomposition}{Jordan-Hahn Decomposition Theorem}
motivates the use of the expression \textbf{signed measure} for denoting any
countably additive extended-real valued set function defined on a
\(\sigma\)-field.

Given a signed measure \(\lambda\), we call \(\lambda^+\) its \textbf{upper
variation}, \(\lambda^-\) its \textbf{lower variation}, and
\(\left|\lambda\right|=\lambda^++\lambda^-\) its \textbf{total variation}.

\begin{remk}\label{remark:characterisation of nullity of TV} Note that, for
every \(A\in\mathcal{F}\), \(\left|\lambda(A)\right|\leq \left|\lambda\right|(A)\):
	\[ \left|\lambda(A)\right|=\left|\lambda^+(A)-\lambda^-(A)\right|\leq\lambda^+(A)+\lambda^-(A)=\left|\lambda\right|(A) .\]
	
	Additionally, \(\left|\lambda\right|(A)=0\) if, and only if,
\(\lambda(B)=0\) for every \(B\subseteq A, B\in\mathcal{F}\).
\end{remk}

\begin{remk}\label{remark:triangular inequality for signed measures}
  For every signed measure \(\lambda\) and every \(\alpha\in \overline{\mathbb{R}}\), one has \(|\alpha\lambda|=|\alpha||\lambda|\). Additionally, if \(\tau\) is another signed measure such that \(\lambda+\tau\) is well defined, then \(|\lambda+\tau|(A)\leq|\lambda|(A)+|\tau|(A)\) for every measurable \(A\) .
\end{remk}
\section{Absolute continuity and singularity of signed measures}
We are now ready to study relations between measures. From this study will arise two
important results: the Radon-Nikodým Theorem and the Lebesgue Decomposition
Theorem.
% TODO: Blabla motivacio mesura singletons i densitats %TODO: Millorar simbol de continuitat absoluta "<<"

In this section, we are to introduce two important concepts, that can be regarded, in some sense, as opposite. A (signed) measure is said to be
\emph{absolutely continuous} with respect to another signed measure whenever it can have no effect on
sets that are null with respect to the former. We might also be interested on (signed) measures
that \emph{only} have effect on null sets with respect to the former, and these are called \emph{singular}.


\begin{defn} Let \(\mathcal{F}\) be a \(\sigma\)-field, \(\mu\) be a measure on \(\mathcal{F}\)  and \(\nu\) be a signed
measure on \(\mathcal{F}\). We say that \(\nu\) is \textbf{absolutely continuous} with
respect to \(\mu\) if \(\mu(A)=0\) implies \(\nu(A)=0\), and we denote it by
\(\nu\ll\mu\). If \(\lambda_{1}\) and \(\lambda_{2}\) are signed measures, we
say that \(\lambda_{1}\) is \textbf{absolutely continuous} with respect to
\(\lambda_{2}\) whenever
\(\lambda_{1}\ll\left|\lambda_{2}\right|\).
	
We say that \(\nu\) is \textbf{singular} with respect to \(\mu\) if there
exists some \(A\in\mathcal{F}\) such that \(\mu(A)=0\) and \(\nu(A^c)=0\), and we denote
it by \(\nu\perp\mu\). If \(\lambda_{1}\) and \(\lambda_{2}\) are signed
measures, we say that \(\lambda_{1}\) is \textbf{singular} with respect to
\(\lambda_{2}\) whenever
\(\left|\lambda_{1}\right|\perp\left|\lambda_{2}\right|\).
\end{defn}

It is clear that if \(\lambda_{1}\perp\lambda_{2}\), then \(\lambda_{2}\perp\lambda_{1}\). As
expected, there are relations between the two concepts. We capture this in the following result:
\begin{lemm}\label{lemma:properties of singularity and absolute continuity}
  Let \(\lambda_{1}, \lambda_{2}\) and \(\nu\) be signed measures defined on a \(\sigma\)-field \(\mathcal{F}\).
  \begin{enumerate}
	\item\label{lemma:properties of singularity and absolute continuity 1} If \(\lambda_{1}\perp\nu\) and \(\lambda_{2}\perp\nu\), then, for every \(\alpha_{1}, \alpha_{2}\in \overline{\mathbb{R}}\) such that \(\alpha_{1}\lambda_{1}+\alpha_{2}\lambda_{2}\) is well-defined, we have that \(\alpha\lambda_{1}+\beta\lambda_{2}\perp\nu\) too.
	\item\label{lemma:properties of singularity and absolute continuity 2} \(\lambda_{1}\ll\nu\) if, and only if, \(|\lambda_{1}|\ll\nu\).
	\item\label{lemma:properties of singularity and absolute continuity 3} If \(\lambda_{1}\ll\nu\) and \(\lambda_{2}\perp\nu\), then \(\lambda_{1}\perp\lambda_{2}\).
	\item\label{lemma:properties of singularity and absolute continuity 4} If \(\lambda_{1}\ll\nu\) and \(\lambda_{1}\perp\nu\), then \(\lambda_{1}\equiv 0\).
	\item\label{lemma:properties of singularity and absolute continuity 5} If \(\lambda_{1}\) is finite, then \(\lambda_{1}\ll\nu\) if, and only if, \(\lim_{|\nu|(A)\to0}\lambda_{1}(A)=0\).
  \end{enumerate}
\end{lemm}
\begin{proof}
  \begin{enumerate}
	\item Let \(A_{1}\) and \(A_{2}\) be sets in \(\mathcal{F}\) such that
	\(|\lambda_{1}|(A_{1})=|\lambda_{2}|(A_{2})=0\) and \(|\nu|(A_{1}^{c})=|\nu|(A_{2}^{c})=0\).
		  Let \(B=A_{1}\cap A_{2}\). It is clear that \(|\nu|(B^{c})=0\) and \(|\lambda_{1}|(B)=|\lambda_{2}|(B)=0\).
		  Then, by \Cref{remark:triangular inequality for signed measures}, we have \(|\alpha_{1}\lambda_{1}+\alpha_{2}\lambda_{2}|(B)\leq|\alpha_{1}\lambda_{1}|(B)+|\alpha_{2}\lambda_{2}|(B)=|\alpha_{1}|\cdot0+|\alpha_{2}|\cdot0=0\).
	\item Immediate by \Cref{remark:characterisation of nullity of TV}.
	\item Let \(A\in\mathcal{F}\) be a set such that \(|\lambda_{2}|(A)=|\nu|(A^{c})=0\). Since \(|\lambda_{1}|\ll|\nu|\) (by \ref{lemma:properties of singularity and absolute continuity 2}), it must be \(|\lambda_{1}|(A^{c})=0\). Thus, \(\lambda_{1}\perp\lambda_{2}\).
	\item By \ref{lemma:properties of singularity and absolute continuity 3}, we have \(\lambda_{1}\perp\lambda_{1}\). It follows that there exists some \(A\in\mathcal{F}\) such that  \(|\lambda_{1}|(\Omega)=|\lambda_{1}|(A)+|\lambda_{1}|(A^{c})=0+0=0\). Thus, \(|\lambda_{1}|\equiv0\), whence \(\lambda_{1}\equiv0\)
	\item Suppose that \(\lim_{\nu(A)\to0}\lambda_{1}(A)=0\). If \(B\in\mathcal{F}\) is a set such that \(|\nu|(B)=0\), it follows that \(|\lambda_{1}(B)|<\varepsilon\) for all \(\varepsilon>0\), whence \(\lambda_{1}(B)=0\).
		  Conversely, suppose that there exists some \(\varepsilon>0\) such that, for every \(n\in\mathbb{N}\) there exists some \(A_{n}\in\mathcal{F}\) with \(|\nu|(A_{n})<2^{-n}\) and
		  \(|\lambda_{1}(A_{n})|\geq\varepsilon\). Note that \(|\lambda_{1}|(A_{n})\geq|\lambda(A_{n})|\geq\varepsilon\). Thus, if we define \(A=\limsup_{n}A_{n}\), we have \(|\lambda_{1}|(\bigcup_{k\geq
		  n}A_{k})\geq|\lambda_{1}|(A_{n})\geq\varepsilon\), whence
		  \[|\lambda_{1}|(A)=\lim_{n\to+\infty}\left(\bigcup_{k\geq n}A_{k}\right)\geq\varepsilon.\]
			However, since \(\sum_{n}|\nu|(A_{n})<+\infty\), by the \hyperref[theorem:Borel-Cantelli Lemma]{Borel-Cantelli Lemma}, it must be \(|\nu|(A)=0\), a contradiction.
  \end{enumerate}
\end{proof}

Note that if \(\lambda\) can be written as the integral of some Borel measurable
function \(g\), that is, \(\lambda(A)=\int_{A}gd\mu\) for every \(A\in\mathcal{F}\), then
\(\lambda\ll\mu\). The Radon-Nikodým Theorem states the converse result, under
the hypothesis that \(\mu\) be \(\sigma\)-finite:

\begin{thrm}[Radon-Nikodým Theorem]\label{theorem:Radon-Nikodym} Let \(\mu\) be a \(\sigma\)-finite
measure defined on a measurable space \(\left(\Omega,\mathcal{F}\right)\). Let
\(\lambda\) be a signed measure that is absolutely continuous with respect to
\(\mu\). Then, there exists a Borel measurable function
\(g\colon\Omega\to\overline{\mathbb{R}}\) such that
	\[ \lambda(A)=\int_{A}g~d\mu \text{ ~~for every } A\in\mathcal{F}.
	\] If \(h\) is another such function, then \(g=h\) \(\mu\)-a.e.
\end{thrm}

\begin{proof} Uniqueness is a direct consequence of \Cref{corollary:equality of
integrals implies equality of functions}.  We will start the proof with strict
hypotheses to \(\mu\) and \(\lambda\) and work upwards.
	\begin{enumerate}
		\item \label{proof:Radon-Nikodym 1} Suppose \(\lambda\) and \(\mu\) are
finite measures.
		
This theorem states
the existence of a function satisfying certain ``abstract'' properties. Many
such theorems can be proved by using Zorn's lemma, and that is what we are going
to do.  Firstly, we need to construct the set and a partial order in it whose
maximal element is to be our candidate function. With this in mind, consider the
set
		\[ G=\left\{g\colon\Omega\to\overline{\mathbb{R}}\left| g \text{ is
Borel measurable}, g\geq0 \text{ and } \lambda(A)\geq\int_{A}g~d\mu \text{ for
every } A\in\mathcal{F}\right.\right\},
		\] and \(\mathcal{G}=G/{\sim}\), where \(\sim\) is the usual equivalence
relation identifying functions that coincide \(\mu\)-a.e. The choice for \(G\)
makes sense because it transforms our desired property into a more easily
``partially orderable'' one, and we need to take the quotient because
\Cref{theorem:inequality of integrals implies inequality of functions} only ensures
inequalities \(\mu\)-a.e. The set constructed is nonempty because
\(0\in\mathcal{G}\) (this is the reason why we choose \(\geq\) instead of
\(\leq\) in the definition of \(G\)). Now we need to partially order our set.
Keeping in mind that we want our maximal element \(g\) to be our candidate, and
that we have ensured that \(\lambda(A)\geq\int_{A}gd\mu\), we would like the
equality not to be strict. We want, then, integrals of \(g\) to be ``the
biggest'' as possible. Thus, a good candidate for partial ordering \(\leq^{*}\)
would be \(h_{1}\leq^{*} h_{2}\) whenever
\(\int_{A}h_{1}d\mu\leq\int_{A}h_{2}d\mu\) for all \(A\in\mathcal{F}\). However, by
\Cref{theorem:inequality of integrals implies inequality of functions}, this is equivalent to
simply \(h_{1}\leq h_{2}\) \(\mu\)-a.e. in the standard sense. Therefore we will
define our ordering in \(\mathcal{G}\) as \(h_{1}\leq^{*}h_{2}\) whenever
\(h_{1}\leq h_{2}\) \(\mu\)-almost everywhere on \(\Omega\).
		
		Having a candidate partially ordered set, we will now use Zorn's lemma
to see it has a maximal element. Let \(\mathcal{I}\) be a chain of
\(\mathcal{G}\). Let
\(M=\sup_{f\in\mathcal{I}}\left\{\int_{\Omega}fd\mu\right\}\) and
\(f_{1},f_{2},\dots\) a sequence of functions in \(\mathcal{I}\) such that
\(\int_{\Omega}f_{n}d\mu\uparrow M\). Since \(\mathcal{I}\) is a chain, the
sequence of functions is necessarily increasing \(\mu\)-a.e. Thus, we can define
\(f=\lim_{n}f_{n}\) almost everywhere and \(f=0\) on the set where the sequence
is not monotone. By the Extended Monotone Convergence Theorem,
\(\int_{\Omega}fd\mu=M\). This function is an upper bound of \(\mathcal{I}\):
Let \(g\in\mathcal{I}\). If \(g\geq^{*} f_{n}\) for all \(n\), then
\(g\geq^{*}f\), and thus \(\int_{\Omega}gd\mu=M=\int_{\Omega}fd\mu\), but then
\(g=f\) \(\mu\)-a.e. If \(g<^{*}f_{n}\), since \(f_{n}\uparrow f\) \(\mu\)-a.e.,
then \(g<^{*}f\).
		
		By Zorn's lemma, \(\mathcal{G}\) has a maximal element, \(g\). We will
now show that this is the function we were looking for. Define a measure on
\(\mathcal{F}\) by \(\nu(A)=\lambda(A)-\int_{A}gd\mu\). Since \(\lambda\) is absolutely
continuous with respect to \(\mu\), so is \(\nu\). Suppose, by way of
contradiction, that \(\nu(\Omega)>0\), and set \(k=2\mu(\Omega)/\nu(\Omega)>0\),
so that
		\[ k\nu(\Omega)-\mu(\Omega)>0
		\] Define a signed measure \(\eta=k\nu-\mu\), which is absolutely
continuous with respect to \(\mu\), and use \cref{corollary:positive set of a
signed measure} to obtain a set \(D\in\mathcal{F}\) such that \(\eta(A\cap D)\geq0\) for
all \(A\in\mathcal{F}\). By \cref{corollary:positive set of a signed measure is its
positive part}, we have that
\(\eta(D)=\eta(\Omega\cap D)=\eta^+(\Omega)\geq\eta(\Omega)>0\). Since \(\eta\)
is absolutely continuous with respect to \(\mu\), then \(\mu(D)>0\) (if
\(\mu(D)\) was \(0\), so would be \(\eta(D)\)).
		
		Finally, define the function \(h=g+\frac{1}{k}I_{D}\). Note that \(h\)
is Borel measurable, nonnegative and since \(\mu(D)>0\), \(h>^{*}g\). Now, for
any \(A\in\mathcal{F}\), we have
		\[ \lambda(A)-\int_{A}hd\mu=\nu(A)-\frac1k\mu(A\cap D)\geq\frac1k\left(k\nu(A\cap D)-\mu(A\cap D)\right)=\frac1k\eta(A\cap D)\geq0.
		\] Therefore, \(h\in G\), contradicting the maximality of \(g\). It must
be that \(\nu(\Omega)=0\), and thus \(\nu(A)=\lambda(A)-\int_{A}gd\mu=0\) for
every \(A\in\mathcal{F}\).
		\item \label{proof:Radon-Nikodym 2} Suppose \(\lambda\) is a
\(\sigma\)-finite measure and \(\mu\) is a finite measure.
		
		Decompose \(\lambda\) into countably many finite measures:
\(\lambda=\sum_{n}\lambda_{n}\). For every \(n\), \(\lambda_{n}\) is absolutely
continuous with respect to \(\mu\). Apply \ref{proof:Radon-Nikodym 1} to obtain
a function \(g_{n}\) that is nonnegative \(\mu\)-a.e. Then, \(g=\sum_{n}g_{n}\)
is the desired function: for every \(A\in\mathcal{F}\),
		\[ \lambda(A)=\sum_{n}\lambda_{n}(A)=\sum_{n}\int_{A}g_{n}d\mu=\int_{A}gd\mu,
		\] where in the last step we used \ref{corollary:exchange series and
integral}.
		
		\item \label{proof:Radon-Nikodym 3} Suppose \(\lambda\) is an arbitrary
measure and \(\mu\) is a finite measure.
		
		Given some \(C\in\mathcal{F}\), define the \(\sigma\)-field
\(\mathcal{F}_C=\left\{B\cap C\colon B\in\mathcal{F}\right\}\). Define the measures
\(\lambda_{C}\) and \(\mu_{C}\) over \(\mathcal{F}_{C}\) as
\(\lambda_{C}=\lambda|_{F_{C}}\), \(\mu_{C}=\mu|_{F_{C}}\). Note that
\(\lambda_{C}\) is absolutely continuous with respect to \(\mu_{C}\). Let
\(\mathcal{S}\) be the class of sets \(C\in\mathcal{F}\) such that \(\lambda_{C}\) is
\(\sigma\)-finite. \(\mathcal{S}\) is not empty since
\(\emptyset\in\mathcal{S}\). Note that \(\mathcal{S}\) is closed under countable
unions: if \(S_{1},S_{2},\dots\) is a sequence of sets in \(\mathcal{S}\) and
\(S=\bigcup_{n}S_{n}\), we can write \(S_{n}=\bigcup_{m}A_{nm}\), where
\(\lambda(A_{nm})<\infty\). Thus, \(S=\bigcup_{n,m}A_{nm}\), showing that
\(S\in\mathcal{S}\).
		
		Let \(M=\sup\left\{\mu(A)\colon A\in\mathcal{S}\right\}\), and consider
a sequence of sets \(B_{1},B_{2},\dotsc\in\mathcal{S}\) such that
\(\mu(S_{n})\uparrow M\). Let \(B=\bigcup_{n}S_{n}\in\mathcal{S}\). Then, for
all \(n\), \(M\geq\mu(B)\geq\mu(S_{n})\), whence \(\mu(B)=M\).
		
		Apply \ref{proof:Radon-Nikodym 2} to \(\lambda_{B}\) and \(\mu_{B}\) to
obtain a function \(g'\). Define \(g=g'+\infty I_{B^{c}}\). Now, for any set
\(A\in\mathcal{F}\),
		\[ \lambda(A)=\lambda(A\cap B)+\lambda(A\setminus B)=\int_{A\cap B}g'd\mu + \lambda(A\setminus B)=\int_{A\cap B}gd\mu+\lambda(A\cap B).
		\] We need only to show that
\(\lambda(A\setminus B)=\int_{A\setminus B}gd\mu\). If \(\mu(A\setminus B)=0\),
both values are clearly \(0\). If \(\mu(A\setminus B)>0\), clearly
\(\int_{A\setminus B}gd\mu=+\infty\), and it follows too that
\(\lambda(A\setminus B)=+\infty\): suppose \(\lambda(A\setminus B)<\infty\).
Then \(A\cup B=B\cup (A\setminus B)\in\mathcal{S}\), because we can decompose
\(B\) into countably many sets \(B_{1},B_{2},\dotsc\) such that
\(\lambda(B_{n})<\infty\), and thus
\(A\cup B=\bigcup_{n}B_{n}\cup \left(A\setminus B\right)\), which is a countable
union. However, \(M\geq\mu(A\cup B)=\mu(B)+\mu(A\setminus B)>\mu(B)=M\), a
contradiction.
		
		
		\item \label{proof:Radon-Nikodym 4} Suppose \(\lambda\) is an arbitrary
measure and \(\mu\) is a \(\sigma\)-finite measure.
		
		Since \(\mu\) is \(\sigma\)-finite, \(\Omega\) can be decomposed into
countably many disjoint sets \(A_{1},A_{2},\dots\) such that
\(\mu(A_{n})<\infty\) for every \(n\). Define \(\lambda_{n}=\lambda_{A_{n}}\),
\(\mu_{n}=\mu_{A_{n}}\) (with the notation established in \ref{proof:Radon-Nikodym 3}). It is clear that \(\lambda_{n}\) is \(\sigma\)-finite with
respect to \(\mu_{n}\) for every \(n\). Apply \ref{proof:Radon-Nikodym 3} to
obtain a function \(g_{n}\) that is nonnegative \(\mu\)-a.e. Then,
\(g=\sum_{n}g_{n}I_{A_{n}}\) is the desired function, as we will show:
		
		First, consider any \(A\in\mathcal{F}\), and
		\[ \int_{A}g_{n}I_{A_{n}}d\mu_{n}=\int_{A_{n}\cap A}g_{n}d\mu_{n}.
		\] Note that, if we regard \(\left(A_{n}\cap A,\mathcal{F}_{A_{n}\cap A}\right)\)
as a subspace of \(\Omega\) (as in \cref{remark:subspace of a measure space has
the same integral}), and write
\(\mathcal{F}_{n}=\mathcal{F}_{A_{n}\cap A}=\left\{B\cap (A_{n}\cap A)\colon B\in\mathcal{F}\right\}\),
then \(\mu|_{\mathcal{F}_{n}}\equiv\mu_{n}|_{\mathcal{F}_{n}}\). Therefore, their integrals
coincide; that is,
\(\int_{A_{n}\cap A}g_{n}d\mu_{n}=\int_{A_{n}\cap A}g_{n}d\mu\),  and this last
integral is simply \(\int_{A}g_{n}I_{A_{n}}d\mu\).
		
		It now follows that, by \cref{corollary:exchange series and integral},
		\[ \lambda(A)=\sum_{n}\lambda_{n}(A)=\sum_{n}\int_{A}g_{n}I_{n}d\mu=\int_{A}gd\mu
		\]
		\item Suppose \(\lambda\) is an arbitrary measure and \(\mu\) is a
\(\sigma\)-finite measure.
		
		Use Jordan-Hahn Decomposition Theorem to split \(\lambda\) as the
difference of two measures: \(\lambda=\lambda^+-\lambda^-\). Note that
\(\left|\lambda\right|\) is absolutely continuous with respect to \(\mu\) by
\Cref{remark:characterisation of nullity of TV} and, hence, so are \(\lambda^+\)
and \(\lambda^-\). Apply \ref{proof:Radon-Nikodym 4} to \(\lambda^+\) and
\(\lambda^-\), to obtain \(g^+\) and \(g^-\), respectively. Since at least one
of \(\lambda^+\) and \(\lambda^-\) is finite, so is at least one of
\(\int_{\Omega}g^+~d\mu\) or \(\int_{\Omega}g^-~d\mu\). Thus, if we define the
function \(g=g^+-g^-\), its integral \(\int_{\Omega}g~d\mu\) exists. Therefore,
for every \(A\in\mathcal{F}\), \(\int_{A}gd\mu\) exists and
		\[ \lambda(A)=\lambda^+(A)-\lambda^-(A)=\int_{A}g^+~d\mu-\int_{A}g^-~d\mu=\int_{A}g~d\mu.
		\]
		
	\end{enumerate}
\end{proof}

A different approach to proving the Radon-Nikodym Theorem is followed in \cite{kolmogorov1957elements}.

Finally, we are ready to give a second decomposition theorem, which relates the two concepts introduced in this section.

\begin{thrm}[Lebesgue Decomposition Theorem]\label{theorem:Lebesgue Decomposition}
  Let \(\mathcal{F}\) be a \(\sigma\)-field and \(\mu\) a \(\sigma\)-finite measure on \(\mathcal{F}\). Let \(\lambda\) be a signed measure on \(\mathcal{F}\) such that \(|\lambda|\) is \(\sigma\)-finite. Then, there exists a unique decomposition \(\lambda=\lambda_{1}+\lambda_{2}\), where \(\lambda_{1}\) and \(\lambda_{2}\) are signed measures such that \(\lambda_{1}\ll\mu\) and \(\lambda_{2}\perp\mu\).

  Additionally, if \(\lambda\) is a measure, then \(\lambda_{1}\) and \(\lambda_{2}\) are measures.
\end{thrm}
\begin{proof}
First, suppose that \(\lambda\) is a measure. Define \(m=\mu+\lambda\), which is a \(\sigma\)-finite measure (if \(\mu\) is finite on \(A_{1},A_{2},\dotsc\) and \(\lambda\) is finite on \(B_{1},B_{2},\dotsc\), then \(m\) is finite on the sets \(C_{nm}=A_{n}\cap B_{m}\), which there are countably many of, and cover \(\Omega\)). Additionally, both \(\mu\) and \(\lambda\) are absolutely continuous with respect to \(m\).

By the \href{theorem:Radon-Nikodym}{Radon-Nikodym Theorem}, there exist a nonnegative, Borel measurable function \(f\) such that
\[\mu(A)=\int_{A}f~dm.\]
Define the sets \(B=\{\omega\in\Omega|f(\omega)>0\}\) and \(C=B^{c}=\{\omega\in\Omega|f(\omega)=0\}\). Define the measures
\[\lambda_{1}(A)=\lambda(A\cap B), ~~~\lambda_{2}(A)=\lambda(A\cap C).\]
It is clear that \(\lambda=\lambda_{1}+\lambda_{2}\). Additionally, if \(\mu(A)=\int_Af~dm=
0\), it must be that \(f=0\) \(m\)-a.e on A. Since \(f>0\) on \(A\cap B\), it must be that \(m(A\cap B)=0\), whence \(\lambda_{1}(A)=\lambda(A\cap B)=0\). Thus, \(\lambda_{1}\ll\mu\).
Finally , \(\lambda_{2}(B)=\lambda(B\cap C)=\lambda(\emptyset)=0\), and \(\mu(B^{c})=\mu(C)=\int_Cf~dm=\int_C0~dm=0\).

For the general case, use \Cref{theorem:Jordan-Hahn Decomposition} to split \(\lambda=\lambda^{+}-\lambda^{-}\). Since \(|\lambda|\) is \(\sigma\)-finite, so are \(\lambda^{+}\) and \(\lambda^{-}\). Use the result proved so far to decompose \(\lambda^{+}=\lambda_{1}^{+}+\lambda_{2}^{+}\), \(\lambda^-=\lambda_1^-+\lambda_2^-\), with \(\lambda_{1}^{+}, \lambda_{1}^{-}\ll\mu\) and \(\lambda_{2}^{+},\lambda_{2}^{-}\perp\mu\).

Since at least one of \(\lambda^{+}, \lambda^{-}\) is finite, we can define the signed measures \(\lambda_{1}=\lambda_{1}^{+}-\lambda_{1}^{-}\) and
\(\lambda_{2}=\lambda_{2}^{+}-\lambda_{2}^{-}\) (the expression \(+\infty-\infty\) is never attained), so that \(\lambda=\lambda_{1}+\lambda_{2}\). It is easy to check that \(\lambda_{1}\ll\mu\) and, by \cref{lemma:properties of singularity and absolute continuity 1}, \(\lambda_{2}\perp\mu\).
\end{proof}

%!TeX root=Final.tex

\chapter{A Brief Application to Analysis}\label{chapter:A Brief Application to Analysis}
\setcounter{section}{1}
\begin{prop}\label{proposition:integral over a simple region}
    Let \(C\subseteq\mathbb{R}^{n-1}\) be a compact set (with the usual topology). Let \(l,u\colon C\to\mathbb{R}\) be continuous functions such that \(l(x)\leq u(x)~\forall x\in C\). Consider the set
    \[
        S=\{(x,t)\in\mathbb{R}^{n-1}\times\mathbb{R}\left|x\in C, l(x)\leq t\leq u(x)\right.\}
    .\]
    Let \(f\colon S\to \mathbb{R}\) be a bounded, continuous function. Then, \(f\) is integrable on \(S\) and 
    \[
        \int_S f = \int_{x\in C}\int_{t=l(x)}^{t=u(x)} f(x,t)
    ,\]
    
    where the integration is understood to be performed in the Riemann sense.  
\end{prop}
\begin{proof}
    We know, from \Cref{theorem:Riemann and Lebesgue}, that Riemann integration coincides with Lebesgue integration (with respect to the Lebesgue measure) on ``classical'' rectangles of \(\mathbb{R}^n\) (that is, cartesian products of intervals). Riemann integration is usually defined on classical rectangles and then extended to bounded sets via indicator functions (see \cite{apuntes_Manolo}). Thus, both kinds of integration will coincide on bounded sets.

    It follows from Heine-Borel and Weierstrass Theorems that \(S\) is a bounded subset of \(\mathbb{R}^n\) (Heine-Borel guarantees the boundedness of \(C\), and Weierstrass guarantees the boundedness of limit functions \(a(x),b(x)\)). Thus, we can apply all techniques regarding Lebesgue integration.

    Additionally, \(S\) is closed because of the following reasoning:
if \(g\colon X\to \mathbb{R}\) is a continuous function, then the epigraph \(E(f)=\{(x,y)\in X\times\mathbb{R}\left|g(x)\leq y\right.\}\) is a closed set of the product topological space \(X\times \mathbb{R}\). Similarly, so is the hypograph \(H(f)=\{(x,y)\in X\times\mathbb{R}\left|g(x)\geq y\right.\}\). Then, \(S=E(l)\cap H(u)\) is closed. By the Heine-Borel theorem, \(S\) is compact.

    Extend \(f\) to \(\mathbb{R}^n\) setting its value to \(0\) outside of \(C\). Now \(I_Sf\), regarded as a function from \(\mathbb{R}^n\) to \(\mathbb{R}\), is Borel measurable (since \(S\) is closed, it is Borel measurable) and bounded because \(f\) is (by Weierstrass Theorem). Since \(S\) is also bounded, \(I_Sf\) is integrable on \(\mathbb{R}^n\).

    Now split \(f=f^{+}-f^{-}\). Use \Cref{remark:two-dim Fubini's on subsets} to see that
    \[
        \int_{S}f^{+}=\int_{\mathbb{R}^{n-1}}\left(\int_{S^{x}}f^{+}(x,t)~dt\right)dx
    .\]
    Now note that \(S^x=[l(x),u(x)]\) if \(x\in C\) and \(S^x=\emptyset\) otherwise. Therefore,
    \[
        \int_{S}f^{+}=\int_{C}\int_{l(x)}^{u(x)}f^{+}(x,t)~dt~dx
    .\]
    A similar reasoning applied to \(f^{-}\) yields the desired result expanding and regrouping the expression \(\int_{S}f=\int_{S}f^{+}-\int_{S}f^{-}\).
\end{proof}


% En aquest cas sols hi ha un fitxer d'annexos,
% però podeu afegir tants \include com calgui. 

%%%%%%% Fi apèndix 

% No toqueu la línia següent 
\backmatter

% La comanda següent defineix l'estil bibliogràfic
\bibliographystyle{IEEEtran}

% La comanda següent defineix el fitxer que
% conté les referències bibliogràfiques.
% En aquest cas és el fitxer Bibliografia.bib
\bibliography{biblioTFG} 

\end{document}
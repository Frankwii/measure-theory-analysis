%!TeX root=Final.tex
\chapter{RELATIONS BEWTEEN MEASURES}\label{chapter:relations between measures}

In \Cref{chapter:elementary measure theory} we studied
measures mostly \textit{intrinsically}: we constructed measure spaces and
studied the properties of functions on fixed measure spaces. However, we did not
study measures \textit{extrinsically}, that is, how different measure spaces
relate to each other. This is one of the main goals of this chapter.

\section{Jordan-Hahn Decomposition Theorem}\label{section:Jordan-Hahn Decomposition}
In this section we will develop a
way of systematically studying \(\sigma\)-additive set functions defined on
\(\sigma\)-fields. This class of functions is interesting because it contains
all set functions of the kind \(\lambda(A)=\int_{A}f~d\mu\) for some Borel
measurable function \(f\) such that \(\int_{\Omega}f~d\mu\) exists. Every such function can be
written as the difference of two measures \(\lambda=\lambda^+-\lambda^-\):
simply take \(\lambda^+(A)=\int_{A}f^+~d\mu\) and
\(\lambda^-(A)=\int_{A}f^-~d\mu\).

The Jordan-Hahn Decomposition Theorem states that the same is true for the
entire class: any \(\sigma\)-additive set function on a \(\sigma\)-field can be
expressed as the difference of two measures. Before stating it, we need some to
develop one result that is very useful by itself, and will allow us to prove the
main theorem:

\begin{thrm}\label{theorem:sigma-additive function assumes extrema} Let
\(\lambda\) be a countably additive extended real-valued set function defined on
a \(\sigma\)-field \(\mathcal{F}\) of subsets of \(\Omega\). Then, \(\lambda\) assumes
its maximum and minimum, that is, there exist \(C,D\in\mathcal{F}\) such that
	\[\lambda(C)=\sup\left\{\lambda(A)\colon A\in\mathcal{F}\right\} \text{ and
} \lambda(D)=\inf\left\{\lambda(A)\colon A\in\mathcal{F}\right\}\]
\end{thrm}

\begin{proof} We will first consider the supremum. We may assume that
\(\lambda(A)<+\infty\) for every \(A\in\mathcal{F}\), for if \(\lambda(A_0)=\infty\), we
take \(C=A_0\). Take a sequence of sets \(A_n\in\mathcal{F}\) such that
\(\lambda(A_n)\to\sup\lambda\), and set \(A=\bigcup_nA_n\).
	
	The idea of the proof is to ``approximate'' \(A\) by finitely many subsets
\(A_1,\dots,A_n\) and take only the positive bits created by those subsets. More
concretely, consider some fixed \(n\in\mathbb{Z}^+\) and note that, for each \(k\leq n\)
and for each \(a\in A\), either \(a\in A_k\) or \(a\in A_k^c\). Thus, we may
write \(A=\bigcup_{m=1}^{2^n}A_{nm}\), where the sets \(A_{nm}\) comprise all
the different \(2^n\) sets of the form \(A_1^*\cap\dots\cap A_n^*\), where
\(A_k^*\) is either \(A_k\) or \(A_k^c\). From this construction, we can observe
that:
	
	\begin{enumerate}
		\item \label{theorem:sigma-additive function assumes extrema 1} The
sets \(A_{nm}\) are disjoint.
		\item \label{theorem:sigma-additive function assumes extrema 2} For
every \(k\leq n\), \(A_k\) is the finite union of some of the sets \(A_{nm}\).
		\item \label{theorem:sigma-additive function assumes extrema 3} If
\(n'>n\), each \(A_{nm}\) is a subset of some \(A_{n'm'}\).
	\end{enumerate}
	
	Define \(\displaystyle B_n\) as the union of all sets \(A_{nm}\) such that
\(\lambda(A_{nm})\geq0\). From \ref{theorem:sigma-additive function assumes
extrema 2}, it follows that \(\lambda(A_n)\leq\lambda(B_n)\) and from
\ref{theorem:sigma-additive function assumes extrema 3} we have that, for
\(k>n\), \(\displaystyle\bigcup_{k=n}^rB_k\) can be written as the disjoint
union of \(B_n\) and some sets \(A_{n'm'}\). Thus,
\(\lambda(\bigcup_{k=n}^{r}B_k)\geq\lambda(B_n)\). Therefore, we have
	\[ \sup\lambda=\lim_n\lambda(A_n)\leq\lim_n\lambda(B_n)\leq\lambda\left(\bigcup_{k=n}^{r}B_n\right)\uparrow_{r}\lambda\left(\bigcup_{k=n}^{+\infty}B_k\right).
	\]
	
	If we set \(C=\limsup_nB_n\), since
\(\lambda\left(\bigcup_{k=n}^{+\infty}B_k\right)\downarrow\lambda(C)\), we have
\(\sup\lambda\leq\lambda(C)\leq\sup\lambda\).
	
	The set \(D\) is obtained by applying the result proved so far to
\(-\lambda\).
\end{proof}

We now have all tools required to develop our theorem. Without further delay:

\begin{thrm}[Jordan-Hahn Decomposition Theorem]\label{theorem:Jordan-Hahn
Decomposition} Let \(\lambda\) be a countably additive set function defined on a
\(\sigma\)-field \(\mathcal{F}\). Define the set functions
	\[ \lambda^+(B)=\sup\{\lambda(A)\colon A\subseteq B, A\in\mathcal{F}\} \text{ and
} \lambda^-(B)=-\inf\{\lambda(A)\colon A\subseteq B, A\in\mathcal{F}\}.
	\]
	
	
	Then, \(\lambda^+\) and \(\lambda^-\) are measures on \(\mathcal{F}\) and
\(\lambda=\lambda^+-\lambda^-\).
\end{thrm}
\begin{proof} First, suppose, without loss of generality, that
\(\lambda<+\infty\) (by definition of countably additive set function,
\(+\infty\) and \(-\infty\) cannot both be in the range of \(\lambda\), and if
\(\lambda>-\infty\), we can apply the result to \(-\lambda\)).  Let \(C\in\mathcal{F}\)
be a set such that, for all \(A\in\mathcal{F}\),
	\[ \lambda(A\cap C)\geq0 \text{ and } \lambda(A\cap C^c)\leq0.
	\]
	
	Such a set exists because of \cref{theorem:sigma-additive function assumes
extrema}: take \(C\in\mathcal{F}\) satisfying \(\lambda(C)=\sup\lambda\). Then, if
\(\lambda(A\cap C)<0\), we have
\(\sup\lambda=\lambda(C)=\lambda(C\setminus (C\cap A))+\lambda(A\cap C)<\lambda(C\setminus (C\cap A))\);
and if \(\lambda(A\cap C^c)>0\), then
\(\lambda(C\cup (A\cap C^c))=\lambda(C)+\lambda(A\cap C^c)>\lambda(C)=\sup\lambda\).
	
	From this, we will see that \(\lambda^+(A)=\lambda(A\cap C)\) and
\(-\lambda^-(A)=\lambda(A\cap C^c)\):
	
	Inequalities
\(\lambda(A\cap C)\leq\lambda^+(A), \lambda(A\cap C^c)\geq\lambda^-(A)\) are
immediate from the definitions of \(\lambda^+\) and \(\lambda^-\). To see the
equality, note that, if \(B\subseteq A, B\in\mathcal{F}\),
	\[ \lambda(B)=\lambda(B\cap C)+\lambda(B\cap C^c)\leq\lambda(B\cap  C)\leq\lambda(B\cap C)+\lambda((A\setminus B)\cap C)=\lambda(A\cap C).
	\] Thus, \(\lambda^+(A)\leq\lambda(A\cap C)\). Similarly,
\(-\lambda^-(A)\geq \lambda(A\cap C^c)\).
	
	From here, it clearly follows that both \(\lambda^+\) and \(\lambda^-\) are
measures and that \(\lambda=\lambda^+-\lambda^-\).
\end{proof} Before finishing the section, we can extract a few additional
results.
\begin{corl} Let \(\lambda\) be a countably additive extended real-valued
set function on the \(\sigma\)-field \(\mathcal{F}\). Then,
	\begin{enumerate}
		\item \(\lambda\) is the difference of two measures, at least one of
which is finite.
		\item If \(\lambda\) is finite, then it is bounded.
		\item \label{corollary:positive set of a signed measure} There is a set
\(C\in\mathcal{F}\) such that \(\lambda(A\cap C)\geq0\) and \(\lambda(A\cap C^c)\leq0\)
for all \(A\in\mathcal{F}\).
		\item If \(E\in\mathcal{F}\) is another set satisfying that
\(\lambda(A\cap E)\geq0\) and \(\lambda(A\cap E^c)\leq0\) for all \(A\in\mathcal{F}\),
then \(\lambda^+(A)=\lambda(A\cap E)\) and \(\lambda^-(A)=\lambda(A\cap E^c)\)
for all \(A\in\mathcal{F}\).
		\item \label{corollary:positive set of a signed measure is its positive
part}If \(E\) is one such set, then \(\left|\lambda\right|(C\triangle E)=0\),
where \(\left|\lambda\right|=\lambda^++\lambda^-\).
	\end{enumerate}
\end{corl}
\begin{proof}
	\begin{enumerate}
		\item If \(\lambda<+\infty\), then \(\lambda^+<+\infty\), and if
\(\lambda>-\infty\), then \(\lambda^-<+\infty\).
		\item Consequence of the previous section and the fact that every finite
measure is bounded.
		\item Consequence of \Cref{theorem:sigma-additive function assumes
extrema}, as shown in the proof of \Cref{theorem:Jordan-Hahn Decomposition}.
		\item The set \(C\) in the proof of \ref{theorem:Jordan-Hahn
Decomposition} was imposed no other hypotheses except that \(C\in\mathcal{F}\) and
\(\lambda(A\cap C)\geq0, \lambda(A\cap C^c)\leq0\). Thus, any such set \(E\)
will satisfy \(\lambda^+(A)=\lambda(A\cap E), \lambda^-(A)=\lambda(A\cap E^c)\).
		\item First note that by the property satisfied by \(C\) and \(E\), we
have \(0\leq\lambda(C\cap E^c)\leq0\), and thus, by the previous section we have
\(\lambda^+(E^c)=\lambda^-(C)=0\).  Therefore,
\(0\leq\lambda^+(C\cap E^c)\leq\lambda^+(E^c)=0\), and
\(0\leq\lambda^-(C\cap E^c)\leq\lambda^-(C)=0\).
	\end{enumerate}
\end{proof}

The \href{theorem:Jordan-Hahn Decomposition}{Jordan-Hahn Decomposition Theorem}
motivates the use of the expression \textbf{signed measure} for denoting any
countably additive extended-real valued set function defined on a
\(\sigma\)-field.

Given a signed measure \(\lambda\), we call \(\lambda^+\) its \textbf{upper
variation}, \(\lambda^-\) its \textbf{lower variation}, and
\(\left|\lambda\right|=\lambda^++\lambda^-\) its \textbf{total variation}.

\begin{remk}\label{remark:characterisation of nullity of TV} Note that, for
every \(A\in\mathcal{F}\), \(\left|\lambda(A)\right|\leq \left|\lambda\right|(A)\):
	\[ \left|\lambda(A)\right|=\left|\lambda^+(A)-\lambda^-(A)\right|\leq\lambda^+(A)+\lambda^-(A)=\left|\lambda\right|(A) .\]
	
	Additionally, \(\left|\lambda\right|(A)=0\) if, and only if,
\(\lambda(B)=0\) for every \(B\subseteq A, B\in\mathcal{F}\).
\end{remk}

\begin{remk}\label{remark:triangular inequality for signed measures}
  For every signed measure \(\lambda\) and every \(\alpha\in \overline{\mathbb{R}}\), one has \(|\alpha\lambda|=|\alpha||\lambda|\). Additionally, if \(\tau\) is another signed measure such that \(\lambda+\tau\) is well defined, then \(|\lambda+\tau|(A)\leq|\lambda|(A)+|\tau|(A)\) for every measurable \(A\) .
\end{remk}
\section{Absolute continuity and singularity of signed measures}
We are now ready to study relations between measures. From this study will arise two
important results: the Radon-Nikodým Theorem and the Lebesgue Decomposition
Theorem.
% TODO: Blabla motivacio mesura singletons i densitats %TODO: Millorar simbol de continuitat absoluta "<<"

In this section, we are to introduce two important concepts, that can be regarded, in some sense, as opposite. A (signed) measure is said to be
\emph{absolutely continuous} with respect to another signed measure whenever it can have no effect on
sets that are null with respect to the former. We might also be interested on (signed) measures
that \emph{only} have effect on null sets with respect to the former, and these are called \emph{singular}.


\begin{defn} Let \(\mathcal{F}\) be a \(\sigma\)-field, \(\mu\) be a measure on \(\mathcal{F}\)  and \(\nu\) be a signed
measure on \(\mathcal{F}\). We say that \(\nu\) is \textbf{absolutely continuous} with
respect to \(\mu\) if \(\mu(A)=0\) implies \(\nu(A)=0\), and we denote it by
\(\nu\ll\mu\). If \(\lambda_{1}\) and \(\lambda_{2}\) are signed measures, we
say that \(\lambda_{1}\) is \textbf{absolutely continuous} with respect to
\(\lambda_{2}\) whenever
\(\lambda_{1}\ll\left|\lambda_{2}\right|\).
	
We say that \(\nu\) is \textbf{singular} with respect to \(\mu\) if there
exists some \(A\in\mathcal{F}\) such that \(\mu(A)=0\) and \(\nu(A^c)=0\), and we denote
it by \(\nu\perp\mu\). If \(\lambda_{1}\) and \(\lambda_{2}\) are signed
measures, we say that \(\lambda_{1}\) is \textbf{singular} with respect to
\(\lambda_{2}\) whenever
\(\left|\lambda_{1}\right|\perp\left|\lambda_{2}\right|\).
\end{defn}

It is clear that if \(\lambda_{1}\perp\lambda_{2}\), then \(\lambda_{2}\perp\lambda_{1}\). As
expected, there are relations between the two concepts. We capture this in the following result:
\begin{lemm}\label{lemma:properties of singularity and absolute continuity}
  Let \(\lambda_{1}, \lambda_{2}\) and \(\nu\) be signed measures defined on a \(\sigma\)-field \(\mathcal{F}\).
  \begin{enumerate}
	\item\label{lemma:properties of singularity and absolute continuity 1} If \(\lambda_{1}\perp\nu\) and \(\lambda_{2}\perp\nu\), then, for every \(\alpha_{1}, \alpha_{2}\in \overline{\mathbb{R}}\) such that \(\alpha_{1}\lambda_{1}+\alpha_{2}\lambda_{2}\) is well-defined, we have that \(\alpha\lambda_{1}+\beta\lambda_{2}\perp\nu\) too.
	\item\label{lemma:properties of singularity and absolute continuity 2} \(\lambda_{1}\ll\nu\) if, and only if, \(|\lambda_{1}|\ll\nu\).
	\item\label{lemma:properties of singularity and absolute continuity 3} If \(\lambda_{1}\ll\nu\) and \(\lambda_{2}\perp\nu\), then \(\lambda_{1}\perp\lambda_{2}\).
	\item\label{lemma:properties of singularity and absolute continuity 4} If \(\lambda_{1}\ll\nu\) and \(\lambda_{1}\perp\nu\), then \(\lambda_{1}\equiv 0\).
	\item\label{lemma:properties of singularity and absolute continuity 5} If \(\lambda_{1}\) is finite, then \(\lambda_{1}\ll\nu\) if, and only if, \(\lim_{|\nu|(A)\to0}\lambda_{1}(A)=0\).
  \end{enumerate}
\end{lemm}
\begin{proof}
  \begin{enumerate}
	\item Let \(A_{1}\) and \(A_{2}\) be sets in \(\mathcal{F}\) such that
	\(|\lambda_{1}|(A_{1})=|\lambda_{2}|(A_{2})=0\) and \(|\nu|(A_{1}^{c})=|\nu|(A_{2}^{c})=0\).
		  Let \(B=A_{1}\cap A_{2}\). It is clear that \(|\nu|(B^{c})=0\) and \(|\lambda_{1}|(B)=|\lambda_{2}|(B)=0\).
		  Then, by \Cref{remark:triangular inequality for signed measures}, we have \(|\alpha_{1}\lambda_{1}+\alpha_{2}\lambda_{2}|(B)\leq|\alpha_{1}\lambda_{1}|(B)+|\alpha_{2}\lambda_{2}|(B)=|\alpha_{1}|\cdot0+|\alpha_{2}|\cdot0=0\).
	\item Immediate by \Cref{remark:characterisation of nullity of TV}.
	\item Let \(A\in\mathcal{F}\) be a set such that \(|\lambda_{2}|(A)=|\nu|(A^{c})=0\). Since \(|\lambda_{1}|\ll|\nu|\) (by \ref{lemma:properties of singularity and absolute continuity 2}), it must be \(|\lambda_{1}|(A^{c})=0\). Thus, \(\lambda_{1}\perp\lambda_{2}\).
	\item By \ref{lemma:properties of singularity and absolute continuity 3}, we have \(\lambda_{1}\perp\lambda_{1}\). It follows that there exists some \(A\in\mathcal{F}\) such that  \(|\lambda_{1}|(\Omega)=|\lambda_{1}|(A)+|\lambda_{1}|(A^{c})=0+0=0\). Thus, \(|\lambda_{1}|\equiv0\), whence \(\lambda_{1}\equiv0\)
	\item Suppose that \(\lim_{\nu(A)\to0}\lambda_{1}(A)=0\). If \(B\in\mathcal{F}\) is a set such that \(|\nu|(B)=0\), it follows that \(|\lambda_{1}(B)|<\varepsilon\) for all \(\varepsilon>0\), whence \(\lambda_{1}(B)=0\).
		  Conversely, suppose that there exists some \(\varepsilon>0\) such that, for every \(n\in\mathbb{N}\) there exists some \(A_{n}\in\mathcal{F}\) with \(|\nu|(A_{n})<2^{-n}\) and
		  \(|\lambda_{1}(A_{n})|\geq\varepsilon\). Note that \(|\lambda_{1}|(A_{n})\geq|\lambda(A_{n})|\geq\varepsilon\). Thus, if we define \(A=\limsup_{n}A_{n}\), we have \(|\lambda_{1}|(\bigcup_{k\geq
		  n}A_{k})\geq|\lambda_{1}|(A_{n})\geq\varepsilon\), whence
		  \[|\lambda_{1}|(A)=\lim_{n\to+\infty}\left(\bigcup_{k\geq n}A_{k}\right)\geq\varepsilon.\]
			However, since \(\sum_{n}|\nu|(A_{n})<+\infty\), by the \hyperref[theorem:Borel-Cantelli Lemma]{Borel-Cantelli Lemma}, it must be \(|\nu|(A)=0\), a contradiction.
  \end{enumerate}
\end{proof}

Note that if \(\lambda\) can be written as the integral of some Borel measurable
function \(g\), that is, \(\lambda(A)=\int_{A}gd\mu\) for every \(A\in\mathcal{F}\), then
\(\lambda\ll\mu\). The Radon-Nikodým Theorem states the converse result, under
the hypothesis that \(\mu\) be \(\sigma\)-finite:

\begin{thrm}[Radon-Nikodým Theorem]\label{theorem:Radon-Nikodym} Let \(\mu\) be a \(\sigma\)-finite
measure defined on a measurable space \(\left(\Omega,\mathcal{F}\right)\). Let
\(\lambda\) be a signed measure that is absolutely continuous with respect to
\(\mu\). Then, there exists a Borel measurable function
\(g\colon\Omega\to\overline{\mathbb{R}}\) such that
	\[ \lambda(A)=\int_{A}g~d\mu \text{ ~~for every } A\in\mathcal{F}.
	\] If \(h\) is another such function, then \(g=h\) \(\mu\)-a.e.
\end{thrm}

\begin{proof} Uniqueness is a direct consequence of \Cref{corollary:equality of
integrals implies equality of functions}.  We will start the proof with strict
hypotheses to \(\mu\) and \(\lambda\) and work upwards.
	\begin{enumerate}
		\item \label{proof:Radon-Nikodym 1} Suppose \(\lambda\) and \(\mu\) are
finite measures.
		
This theorem states
the existence of a function satisfying certain ``abstract'' properties. Many
such theorems can be proved by using Zorn's lemma, and that is what we are going
to do.  Firstly, we need to construct the set and a partial order in it whose
maximal element is to be our candidate function. With this in mind, consider the
set
		\[ G=\left\{g\colon\Omega\to\overline{\mathbb{R}}\left| g \text{ is
Borel measurable}, g\geq0 \text{ and } \lambda(A)\geq\int_{A}g~d\mu \text{ for
every } A\in\mathcal{F}\right.\right\},
		\] and \(\mathcal{G}=G/{\sim}\), where \(\sim\) is the usual equivalence
relation identifying functions that coincide \(\mu\)-a.e. The choice for \(G\)
makes sense because it transforms our desired property into a more easily
``partially orderable'' one, and we need to take the quotient because
\Cref{theorem:inequality of integrals implies inequality of functions} only ensures
inequalities \(\mu\)-a.e. The set constructed is nonempty because
\(0\in\mathcal{G}\) (this is the reason why we choose \(\geq\) instead of
\(\leq\) in the definition of \(G\)). Now we need to partially order our set.
Keeping in mind that we want our maximal element \(g\) to be our candidate, and
that we have ensured that \(\lambda(A)\geq\int_{A}gd\mu\), we would like the
equality not to be strict. We want, then, integrals of \(g\) to be ``the
biggest'' as possible. Thus, a good candidate for partial ordering \(\leq^{*}\)
would be \(h_{1}\leq^{*} h_{2}\) whenever
\(\int_{A}h_{1}d\mu\leq\int_{A}h_{2}d\mu\) for all \(A\in\mathcal{F}\). However, by
\Cref{theorem:inequality of integrals implies inequality of functions}, this is equivalent to
simply \(h_{1}\leq h_{2}\) \(\mu\)-a.e. in the standard sense. Therefore we will
define our ordering in \(\mathcal{G}\) as \(h_{1}\leq^{*}h_{2}\) whenever
\(h_{1}\leq h_{2}\) \(\mu\)-almost everywhere on \(\Omega\).
		
		Having a candidate partially ordered set, we will now use Zorn's lemma
to see it has a maximal element. Let \(\mathcal{I}\) be a chain of
\(\mathcal{G}\). Let
\(M=\sup_{f\in\mathcal{I}}\left\{\int_{\Omega}fd\mu\right\}\) and
\(f_{1},f_{2},\dots\) a sequence of functions in \(\mathcal{I}\) such that
\(\int_{\Omega}f_{n}d\mu\uparrow M\). Since \(\mathcal{I}\) is a chain, the
sequence of functions is necessarily increasing \(\mu\)-a.e. Thus, we can define
\(f=\lim_{n}f_{n}\) almost everywhere and \(f=0\) on the set where the sequence
is not monotone. By the Extended Monotone Convergence Theorem,
\(\int_{\Omega}fd\mu=M\). This function is an upper bound of \(\mathcal{I}\):
Let \(g\in\mathcal{I}\). If \(g\geq^{*} f_{n}\) for all \(n\), then
\(g\geq^{*}f\), and thus \(\int_{\Omega}gd\mu=M=\int_{\Omega}fd\mu\), but then
\(g=f\) \(\mu\)-a.e. If \(g<^{*}f_{n}\), since \(f_{n}\uparrow f\) \(\mu\)-a.e.,
then \(g<^{*}f\).
		
		By Zorn's lemma, \(\mathcal{G}\) has a maximal element, \(g\). We will
now show that this is the function we were looking for. Define a measure on
\(\mathcal{F}\) by \(\nu(A)=\lambda(A)-\int_{A}gd\mu\). Since \(\lambda\) is absolutely
continuous with respect to \(\mu\), so is \(\nu\). Suppose, by way of
contradiction, that \(\nu(\Omega)>0\), and set \(k=2\mu(\Omega)/\nu(\Omega)>0\),
so that
		\[ k\nu(\Omega)-\mu(\Omega)>0
		\] Define a signed measure \(\eta=k\nu-\mu\), which is absolutely
continuous with respect to \(\mu\), and use \cref{corollary:positive set of a
signed measure} to obtain a set \(D\in\mathcal{F}\) such that \(\eta(A\cap D)\geq0\) for
all \(A\in\mathcal{F}\). By \cref{corollary:positive set of a signed measure is its
positive part}, we have that
\(\eta(D)=\eta(\Omega\cap D)=\eta^+(\Omega)\geq\eta(\Omega)>0\). Since \(\eta\)
is absolutely continuous with respect to \(\mu\), then \(\mu(D)>0\) (if
\(\mu(D)\) was \(0\), so would be \(\eta(D)\)).
		
		Finally, define the function \(h=g+\frac{1}{k}I_{D}\). Note that \(h\)
is Borel measurable, nonnegative and since \(\mu(D)>0\), \(h>^{*}g\). Now, for
any \(A\in\mathcal{F}\), we have
		\[ \lambda(A)-\int_{A}hd\mu=\nu(A)-\frac1k\mu(A\cap D)\geq\frac1k\left(k\nu(A\cap D)-\mu(A\cap D)\right)=\frac1k\eta(A\cap D)\geq0.
		\] Therefore, \(h\in G\), contradicting the maximality of \(g\). It must
be that \(\nu(\Omega)=0\), and thus \(\nu(A)=\lambda(A)-\int_{A}gd\mu=0\) for
every \(A\in\mathcal{F}\).
		\item \label{proof:Radon-Nikodym 2} Suppose \(\lambda\) is a
\(\sigma\)-finite measure and \(\mu\) is a finite measure.
		
		Decompose \(\lambda\) into countably many finite measures:
\(\lambda=\sum_{n}\lambda_{n}\). For every \(n\), \(\lambda_{n}\) is absolutely
continuous with respect to \(\mu\). Apply \ref{proof:Radon-Nikodym 1} to obtain
a function \(g_{n}\) that is nonnegative \(\mu\)-a.e. Then, \(g=\sum_{n}g_{n}\)
is the desired function: for every \(A\in\mathcal{F}\),
		\[ \lambda(A)=\sum_{n}\lambda_{n}(A)=\sum_{n}\int_{A}g_{n}d\mu=\int_{A}gd\mu,
		\] where in the last step we used \ref{corollary:exchange series and
integral}.
		
		\item \label{proof:Radon-Nikodym 3} Suppose \(\lambda\) is an arbitrary
measure and \(\mu\) is a finite measure.
		
		Given some \(C\in\mathcal{F}\), define the \(\sigma\)-field
\(\mathcal{F}_C=\left\{B\cap C\colon B\in\mathcal{F}\right\}\). Define the measures
\(\lambda_{C}\) and \(\mu_{C}\) over \(\mathcal{F}_{C}\) as
\(\lambda_{C}=\lambda|_{F_{C}}\), \(\mu_{C}=\mu|_{F_{C}}\). Note that
\(\lambda_{C}\) is absolutely continuous with respect to \(\mu_{C}\). Let
\(\mathcal{S}\) be the class of sets \(C\in\mathcal{F}\) such that \(\lambda_{C}\) is
\(\sigma\)-finite. \(\mathcal{S}\) is not empty since
\(\emptyset\in\mathcal{S}\). Note that \(\mathcal{S}\) is closed under countable
unions: if \(S_{1},S_{2},\dots\) is a sequence of sets in \(\mathcal{S}\) and
\(S=\bigcup_{n}S_{n}\), we can write \(S_{n}=\bigcup_{m}A_{nm}\), where
\(\lambda(A_{nm})<\infty\). Thus, \(S=\bigcup_{n,m}A_{nm}\), showing that
\(S\in\mathcal{S}\).
		
		Let \(M=\sup\left\{\mu(A)\colon A\in\mathcal{S}\right\}\), and consider
a sequence of sets \(B_{1},B_{2},\dotsc\in\mathcal{S}\) such that
\(\mu(S_{n})\uparrow M\). Let \(B=\bigcup_{n}S_{n}\in\mathcal{S}\). Then, for
all \(n\), \(M\geq\mu(B)\geq\mu(S_{n})\), whence \(\mu(B)=M\).
		
		Apply \ref{proof:Radon-Nikodym 2} to \(\lambda_{B}\) and \(\mu_{B}\) to
obtain a function \(g'\). Define \(g=g'+\infty I_{B^{c}}\). Now, for any set
\(A\in\mathcal{F}\),
		\[ \lambda(A)=\lambda(A\cap B)+\lambda(A\setminus B)=\int_{A\cap B}g'd\mu + \lambda(A\setminus B)=\int_{A\cap B}gd\mu+\lambda(A\cap B).
		\] We need only to show that
\(\lambda(A\setminus B)=\int_{A\setminus B}gd\mu\). If \(\mu(A\setminus B)=0\),
both values are clearly \(0\). If \(\mu(A\setminus B)>0\), clearly
\(\int_{A\setminus B}gd\mu=+\infty\), and it follows too that
\(\lambda(A\setminus B)=+\infty\): suppose \(\lambda(A\setminus B)<\infty\).
Then \(A\cup B=B\cup (A\setminus B)\in\mathcal{S}\), because we can decompose
\(B\) into countably many sets \(B_{1},B_{2},\dotsc\) such that
\(\lambda(B_{n})<\infty\), and thus
\(A\cup B=\bigcup_{n}B_{n}\cup \left(A\setminus B\right)\), which is a countable
union. However, \(M\geq\mu(A\cup B)=\mu(B)+\mu(A\setminus B)>\mu(B)=M\), a
contradiction.
		
		
		\item \label{proof:Radon-Nikodym 4} Suppose \(\lambda\) is an arbitrary
measure and \(\mu\) is a \(\sigma\)-finite measure.
		
		Since \(\mu\) is \(\sigma\)-finite, \(\Omega\) can be decomposed into
countably many disjoint sets \(A_{1},A_{2},\dots\) such that
\(\mu(A_{n})<\infty\) for every \(n\). Define \(\lambda_{n}=\lambda_{A_{n}}\),
\(\mu_{n}=\mu_{A_{n}}\) (with the notation established in \ref{proof:Radon-Nikodym 3}). It is clear that \(\lambda_{n}\) is \(\sigma\)-finite with
respect to \(\mu_{n}\) for every \(n\). Apply \ref{proof:Radon-Nikodym 3} to
obtain a function \(g_{n}\) that is nonnegative \(\mu\)-a.e. Then,
\(g=\sum_{n}g_{n}I_{A_{n}}\) is the desired function, as we will show:
		
		First, consider any \(A\in\mathcal{F}\), and
		\[ \int_{A}g_{n}I_{A_{n}}d\mu_{n}=\int_{A_{n}\cap A}g_{n}d\mu_{n}.
		\] Note that, if we regard \(\left(A_{n}\cap A,\mathcal{F}_{A_{n}\cap A}\right)\)
as a subspace of \(\Omega\) (as in \cref{remark:subspace of a measure space has
the same integral}), and write
\(\mathcal{F}_{n}=\mathcal{F}_{A_{n}\cap A}=\left\{B\cap (A_{n}\cap A)\colon B\in\mathcal{F}\right\}\),
then \(\mu|_{\mathcal{F}_{n}}\equiv\mu_{n}|_{\mathcal{F}_{n}}\). Therefore, their integrals
coincide; that is,
\(\int_{A_{n}\cap A}g_{n}d\mu_{n}=\int_{A_{n}\cap A}g_{n}d\mu\),  and this last
integral is simply \(\int_{A}g_{n}I_{A_{n}}d\mu\).
		
		It now follows that, by \cref{corollary:exchange series and integral},
		\[ \lambda(A)=\sum_{n}\lambda_{n}(A)=\sum_{n}\int_{A}g_{n}I_{n}d\mu=\int_{A}gd\mu
		\]
		\item Suppose \(\lambda\) is an arbitrary measure and \(\mu\) is a
\(\sigma\)-finite measure.
		
		Use Jordan-Hahn Decomposition Theorem to split \(\lambda\) as the
difference of two measures: \(\lambda=\lambda^+-\lambda^-\). Note that
\(\left|\lambda\right|\) is absolutely continuous with respect to \(\mu\) by
\Cref{remark:characterisation of nullity of TV} and, hence, so are \(\lambda^+\)
and \(\lambda^-\). Apply \ref{proof:Radon-Nikodym 4} to \(\lambda^+\) and
\(\lambda^-\), to obtain \(g^+\) and \(g^-\), respectively. Since at least one
of \(\lambda^+\) and \(\lambda^-\) is finite, so is at least one of
\(\int_{\Omega}g^+~d\mu\) or \(\int_{\Omega}g^-~d\mu\). Thus, if we define the
function \(g=g^+-g^-\), its integral \(\int_{\Omega}g~d\mu\) exists. Therefore,
for every \(A\in\mathcal{F}\), \(\int_{A}gd\mu\) exists and
		\[ \lambda(A)=\lambda^+(A)-\lambda^-(A)=\int_{A}g^+~d\mu-\int_{A}g^-~d\mu=\int_{A}g~d\mu.
		\]
		
	\end{enumerate}
\end{proof}

A different approach to proving the Radon-Nikodym Theorem is followed in \cite{kolmogorov1957elements}.

Finally, we are ready to give a second decomposition theorem, which relates the two concepts introduced in this section.

\begin{thrm}[Lebesgue Decomposition Theorem]\label{theorem:Lebesgue Decomposition}
  Let \(\mathcal{F}\) be a \(\sigma\)-field and \(\mu\) a \(\sigma\)-finite measure on \(\mathcal{F}\). Let \(\lambda\) be a signed measure on \(\mathcal{F}\) such that \(|\lambda|\) is \(\sigma\)-finite. Then, there exists a unique decomposition \(\lambda=\lambda_{1}+\lambda_{2}\), where \(\lambda_{1}\) and \(\lambda_{2}\) are signed measures such that \(\lambda_{1}\ll\mu\) and \(\lambda_{2}\perp\mu\).

  Additionally, if \(\lambda\) is a measure, then \(\lambda_{1}\) and \(\lambda_{2}\) are measures.
\end{thrm}
\begin{proof}
First, suppose that \(\lambda\) is a measure. Define \(m=\mu+\lambda\), which is a \(\sigma\)-finite measure (if \(\mu\) is finite on \(A_{1},A_{2},\dotsc\) and \(\lambda\) is finite on \(B_{1},B_{2},\dotsc\), then \(m\) is finite on the sets \(C_{nm}=A_{n}\cap B_{m}\), which there are countably many of, and cover \(\Omega\)). Additionally, both \(\mu\) and \(\lambda\) are absolutely continuous with respect to \(m\).

By the \href{theorem:Radon-Nikodym}{Radon-Nikodym Theorem}, there exist a nonnegative, Borel measurable function \(f\) such that
\[\mu(A)=\int_{A}f~dm.\]
Define the sets \(B=\{\omega\in\Omega|f(\omega)>0\}\) and \(C=B^{c}=\{\omega\in\Omega|f(\omega)=0\}\). Define the measures
\[\lambda_{1}(A)=\lambda(A\cap B), ~~~\lambda_{2}(A)=\lambda(A\cap C).\]
It is clear that \(\lambda=\lambda_{1}+\lambda_{2}\). Additionally, if \(\mu(A)=\int_Af~dm=
0\), it must be that \(f=0\) \(m\)-a.e on A. Since \(f>0\) on \(A\cap B\), it must be that \(m(A\cap B)=0\), whence \(\lambda_{1}(A)=\lambda(A\cap B)=0\). Thus, \(\lambda_{1}\ll\mu\).
Finally , \(\lambda_{2}(B)=\lambda(B\cap C)=\lambda(\emptyset)=0\), and \(\mu(B^{c})=\mu(C)=\int_Cf~dm=\int_C0~dm=0\).

For the general case, use \Cref{theorem:Jordan-Hahn Decomposition} to split \(\lambda=\lambda^{+}-\lambda^{-}\). Since \(|\lambda|\) is \(\sigma\)-finite, so are \(\lambda^{+}\) and \(\lambda^{-}\). Use the result proved so far to decompose \(\lambda^{+}=\lambda_{1}^{+}+\lambda_{2}^{+}\), \(\lambda^-=\lambda_1^-+\lambda_2^-\), with \(\lambda_{1}^{+}, \lambda_{1}^{-}\ll\mu\) and \(\lambda_{2}^{+},\lambda_{2}^{-}\perp\mu\).

Since at least one of \(\lambda^{+}, \lambda^{-}\) is finite, we can define the signed measures \(\lambda_{1}=\lambda_{1}^{+}-\lambda_{1}^{-}\) and
\(\lambda_{2}=\lambda_{2}^{+}-\lambda_{2}^{-}\) (the expression \(+\infty-\infty\) is never attained), so that \(\lambda=\lambda_{1}+\lambda_{2}\). It is easy to check that \(\lambda_{1}\ll\mu\) and, by \cref{lemma:properties of singularity and absolute continuity 1}, \(\lambda_{2}\perp\mu\).
\end{proof}
